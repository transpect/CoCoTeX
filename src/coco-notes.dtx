% \chapter{coco-notes.dtx}
% This file contains the code for foot-  and endnote handling. It provides a
% switch between endnotes and footnotes as well as options to handle
% the resetting of footnote/endnote counters.
%    \begin{macrocode}[gobble=1]
%<*endnotes>
%    \end{macrocode}
%
%    \begin{macrocode}
%%
%% module for CoCoTeX that handles footnote/endnote switching.
%%
%% Maintainer: p.schulz@le-tex.de
%%
%% lualatex  -  texlive > 2019
%%
\NeedsTeXFormat{LaTeX2e}[2018/12/01]
\ProvidesPackage{coco-notes}
    [\filedate \fileversion le-tex coco notes module]
%    \end{macrocode}
% internal switch for endnotes (\lstinline{\endnotestrue}) or footnotes (\lstinline{\endnotesfalse}, default).
%    \begin{macrocode}
\newif\ifendnotes \endnotesfalse
\newif\ifendnotelinks \endnotelinksfalse
%    \end{macrocode}
% package options:
% \begin{itemize}
% \item \lstinline{endnotes} activates endnotes.
% \item \lstinline{ennotoc} prevents chapter headings in the Notes
%   section from creating toc entries.
% \item \lstinline{resetnotesperchapter} resets foot- and endnotes at
%   the start of each chapter level heading. If omitted (default)
%   foot- or endnotes are numbered throughout the whole document
% \item \lstinline{endnotesperchapter} implies \lstinline{endnotes}
%   and allows the output of all collected endnotes at the end of each
%   chapter. It also sets the note's heading to section level
%   (otherwise it is chapter level).
% \end{itemize}
%    \begin{macrocode}
\DeclareOption{endnotes}{\global\endnotestrue}
\DeclareOption{ennotoc}{\global\let\tp@ennotoc\relax}
\DeclareOption{resetnotesperchapter}{\global\let\reset@notes@per@chapter\relax}
\DeclareOption{endnoteswithchapters}{\global\endnotestrue\global\let\endnotes@with@chapters\relax}
\DeclareOption{endnotelinks}{\global\endnotelinkstrue}
\ProcessOptions
%    \end{macrocode}
% footnote package is mandatory since it provides the \lstinline{\savenotes} and \lstinline{\spewnotes} macros:
%    \begin{macrocode}
\RequirePackage{footnote}
%    \end{macrocode}
% Handling of endnotes:
%    \begin{macrocode}
\newif\if@enotesopen
\AtBeginDocument{\edef\tpfn@parindent{\the\parindent}}
\ifendnotes
  \RequirePackage{endnotes}
  \@ifpackageloaded{coco-headings}{\let\tp@useTeXHeading\relax}{}
  % Allow linking endnotes to their respective occurrence in the document.
  \ifendnotelinks
    % Counters are global across all chapters; no reset.
    \newcount\endnoteanchor \global\endnoteanchor\z@
    \newcount\endnoteref \global\endnoteref\z@
    \def\endnote{%
      \@ifnextchar[%
        \@xendnote{%
          \stepcounter{endnote}%
          \phantomsection%
          \label{endnote-\the\endnoteanchor}%
          \global\advance\endnoteanchor\@ne%
          \protected@xdef\@theenmark{\theendnote}%
          \hyperref[endnotetext-\the\endnoteanchor]{\@endnotemark}\@endnotetext%
        }%
    }
  \fi
  \let\footnote=\endnote
  \def\enotesize{\normalsize}%
  \def\enoteformat{%
    \noindent
    \leavevmode
    \hskip-2em\hb@xt@2em{%
      \ifendnotelinks
        \hyperref[endnote-\the\endnoteref]{\@theenmark}\hss%
      \else
        \@theenmark\hss%
      \fi%
    }%
    \ifendnotelinks%
      \phantomsection%
      \label{endnotetext-\the\endnoteref}%
      \global\advance\endnoteref\@ne%
    \fi%
    \expandafter\parindent\tpfn@parindent\relax\expandafter}%
  \gdef\enoteheading{%
    \leftskip2em
  }%
  \def\printnotes{%
    \ifx\endnotes@with@chapters\relax
      \ifnum\c@endnote>\z@
        \expandafter\global\expandafter\let\csname enotes@in@\the\realchap\endcsname\@empty
      \fi
    \fi
    \if@enotesopen
      \global\c@endnote\z@%
      \bgroup
      %\parindent\z@
      \parskip\z@
      \theendnotes
      \egroup
    \fi}
\else
  \newcount\c@endnote \c@endnote\z@
  \let\printnotes\relax
\fi

\newcount\realchap \realchap\z@
\ifx\endnotes@with@chapters\relax
  \AtBeginDocument{%
    \tpAddToHook[heading]{before-hook-chapter}{%
      \ifnum\c@endnote>\z@\relax
        \expandafter\global\expandafter\let\csname enotes@in@\the\realchap\endcsname\@empty
      \fi
      \global\advance\realchap\@ne
      \global\c@endnote\z@
      \def\tp@par@title{\tpIfComp{TocTitle}{\tpUseComp{TocTitle}}{\tpUseComp{Title}}}%
      \def\tp@par@runtitle{\tpIfComp{RunTitle}{\tpUseComp{RunTitle}}{\tpUseComp{Title}}}%
      \addtoendnotes{%
        \noexpand\expandafter\noexpand\ifx\noexpand\csname enotes@in@\the\realchap\noexpand\endcsname\noexpand\@empty
          \bgroup
            \noexpand\leftskip\noexpand\z@
            \noexpand\begin{heading}\ifx\tp@ennotoc\relax[notoc]\fi{section}%
              \noexpand\tpTitle{\tp@par@title}%
              \noexpand\tpRunTitle{\tp@par@runtitle}%
            \noexpand\end{heading}%
          \egroup
        \noexpand\fi}%
    }%
  }
\fi

\ifx\reset@notes@per@chapter\relax
  \AtBeginDocument{%
    \tpAddToHook[heading]{before-hook-chapter}{%
      \global\c@footnote\z@
      \global\c@endnote\z@
    }%
  }%
\fi

%    \end{macrocode}
% Here we make a small adjustment to the \lstinline{\fn@fntext} macro
% from the \lstinline{footnote} package by making it \lstinline{\long}
% and therefore allowing \lstinline{\par} inside it's argument.
%    \begin{macrocode}
\long\def\fn@fntext#1{%
  \ifx\ifmeasuring@\@@undefined%
    \expandafter\@secondoftwo\else\expandafter\@iden%
  \fi%
  {\ifmeasuring@\expandafter\@gobble\else\expandafter\@iden\fi}%
  {%
    \global\setbox\fn@notes\vbox{%
      \unvbox\fn@notes%
      \fn@startnote%
      \@makefntext{%
        \rule\z@\footnotesep%
        \ignorespaces%
        #1%
        \@finalstrut\strutbox%
      }%
      \fn@endnote%
    }%
  }%
}
%    \end{macrocode}
% Re-definition of \lstinline{footnote} package's footnote mark
% retriever to allow non-numeric values in the optional argument of
% \lstinline{\footnote}.
%    \begin{macrocode}
\def\fn@getmark@i#1[#2]{%
  \sbox\z@{\@tempcnta0#2\relax}%
  \ifdim\wd\z@>0\p@\relax
    \def\thempfn{#2}%
    \fn@getmark@iii%
  \else
    \csname c@\@mpfn\endcsname#2%
    \fn@getmark@ii%
  \fi
}
\def\fn@getmark@iii#1{%
  \unrestored@protected@xdef\@thefnmark{\thempfn}%
  \endgroup%
  #1%
}
%    \end{macrocode}
% And the same for plain \LaTeX:
%    \begin{macrocode}
\def\@xfootnote[#1]{%
   \begingroup
     \sbox\z@{\@tempcnta0#1\relax}%
     \ifdim\wd\z@>0\p@\relax
       \unrestored@protected@xdef\@thefnmark{#1}%
     \else
       \csname c@\@mpfn\endcsname #1\relax
       \unrestored@protected@xdef\@thefnmark{\thempfn}%
     \fi
   \endgroup
   \@footnotemark\@footnotetext}
%    \end{macrocode}
%    \begin{macrocode}[gobble=1]
%</endnotes>
%    \end{macrocode}
