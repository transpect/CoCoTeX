%    \end{macrocode}
% \chapter{transpect-floats.dtx}
% This module provides handlers for floating objects like tables and
% figires common to all Transpect projects
%
%    \begin{macrocode}
%%
%% module for le-tex transpect.cls that extends floating objects.
%%
%% Maintainer: p.schulz@le-tex.de
%%
%% lualatex  -  texlive > 2019
%%
\NeedsTeXFormat{LaTeX2e}[2019/01/01]
\ProvidesPackage{transpect-floats}
    [2020/22/04 0.90 le-tex transpect floats module]
%    \end{macrocode}
% Hard dependencies
%    \begin{macrocode}
\usepackage{tabularx}
\usepackage{multirow}
\usepackage{caption}
%    \end{macrocode}
% Additional column types for \lstinline{tabular} and
% \lstinline{tabularx} environments:
% \begin{description}
% \item[Q] Like tabularx's X but left aligned
% \item[W] Like tabularx's X but right aligned
% \item[L] Like p but left aligned
% \item[P] Like p but left aligned
% \item[C] Like tabularx's X but centered
% \item[Z] Like p but centered
% \item[e] Used to expand whitespaces evenly between columns. Used
%   between the descriptor for the first and the second physical
%   column and does itself not provide a column!
% \end{description}
%
%    \begin{macrocode}
\newcolumntype{Q}{>{\raggedright\arraybackslash\unskip}X}
\newcolumntype{W}{>{\raggedleft\arraybackslash\unskip}X}
\newcolumntype{L}{>{\raggedright\arraybackslash\unskip}p}
\newcolumntype{P}{>{\RaggedRight\arraybackslash\unskip}p}
\newcolumntype{R}{>{\raggedleft\arraybackslash\unskip}p}
\newcolumntype{C}{>{\centering\arraybackslash\unskip}X}
\newcolumntype{Z}{>{\centering\arraybackslash\unskip}p}
\newcolumntype{e}{!{\extracolsep{\fill}}}
%    \end{macrocode}
% for automatic typesetting and float positioning, we use very high tolerances from standard \LaTeX:
%    \begin{macrocode}
\def\topfraction{0.9}
\def\textfraction{0.1}
\def\bottomfraction{0.8}
\def\totalnumber{8}
\def\topnumber{8}
\def\bottomnumber{8}
\def\floatpagefraction{.95}
%    \end{macrocode}
% some reserved registers
%    \begin{macrocode}
\newbox \tp@subfltbox
\newcount\tp@subflts          \tp@subflts=\z@\relax
\newdimen\tp@subflt@maxheight \tp@subflt@maxheight=\z@\relax
\newdimen\tp@subflt@sep       \tp@subflt@sep=\fboxsep\relax
\newdimen\tp@total@flt@width     \tp@total@flt@width=\textwidth\relax
%    \end{macrocode}
% This macro simplifies the definition of caption type specific
% setups. \#1 is the property name; \#2 is the global default value.
%    \begin{macrocode}
\def\tp@provide@setting#1#2{%
  \expandafter\def\csname @#1@internal@\endcsname{#2}%
  \expandafter\def\csname #1\endcsname{\@ifnextchar[{\csname @#1\endcsname}{\csname @#1\endcsname[]}}%]
  \expandafter\def\csname @#1\endcsname[##1]##2{%
    \edef\tp@opt@arg{##1}%
    \ifx\tp@opt@arg\@empty
      \ltx@LocalExpandAfter\gdef\csname @#1@internal@\endcsname{##2}%
    \else
      \ltx@LocalExpandAfter\gdef\csname @#1@internal@##1\endcsname{##2}%
    \fi
  }%
}
%    \end{macrocode}
% This macro simplifies the definition of switches with no
% arguments. They are always \lstinline{\relax} by default, and
% \lstinline{\@empty} if set by the user.
%    \begin{macrocode}
\def\tp@provide@switch#1{%
  \expandafter\let\csname @#1@internal@\endcsname\relax%
  \expandafter\def\csname #1\endcsname{\@ifnextchar[{\csname @#1\endcsname}{\csname @#1\endcsname[]}}%]
  \expandafter\def\csname @#1\endcsname[##1]{%
    \edef\tp@opt@arg{##1}%
    \ifx\tp@opt@arg\@empty
      \ltx@LocalExpandAfter\global\expandafter\let\csname @#1@internal@\endcsname\@empty%
    \else
      \ltx@LocalExpandAfter\global\expandafter\let\csname @#1@internal@##1\endcsname\@empty%
    \fi
  }%
}
%    \end{macrocode}
% Macro to expand the float settings and select between (global)
% default and type specific values.
%    \begin{macrocode}
\def\tp@use@setting#1{%
  \expandafter\ifx\csname @#1@internal@\tp@captype\endcsname\relax
    \csname @#1@internal@\endcsname
  \else
    \csname @#1@internal@\tp@captype\endcsname
  \fi}
%    \end{macrocode}
% built in float settings
%    \begin{macrocode}
\tp@provide@setting{tpCaptionLabelSep}{}
\tp@provide@setting{tpCaptionLabelFont}{}
\tp@provide@setting{tpCaptionSourceSep}{}
\tp@provide@setting{tpCaptionSourceFont}{}
\tp@provide@setting{tpCaptionLegendSep}{}
\tp@provide@setting{tpCaptionLegendFont}{}
\tp@provide@setting{tpCaptionFormat}{}
\tp@provide@setting{tpCaptionFormatBelow}{}
\tp@provide@setting{tpCaptionFormatTop}{}
\tp@provide@setting{tpCaptionFont}{}
%    \end{macrocode}
% Space between caption apparatus (top and below) and float apparatus
%    \begin{macrocode}
\tp@provide@setting{tpCaptionTopSep}{\z@}
\tp@provide@setting{tpCaptionBelowSep}{\z@}
\tp@provide@setting{tpCaptionOrderTop}{}
\tp@provide@setting{tpCaptionOrderBelow}{}
\tp@provide@setting{tpCaptionVAlignTop}{t}
\tpCaptionVAlignTop[table]{b}
\tp@provide@setting{tpCaptionVAlignBottom}{t}
\tpCaptionOrderTop[figure]{}
\tpCaptionOrderBelow[figure]{label,caption,source,legend}
\tpCaptionOrderTop[table]{label,caption,source}
\tpCaptionOrderBelow[table]{legend}
\tp@provide@setting{tpIntextSkipAbove}{\z@}
\tp@provide@setting{tpIntextSkipBelow}{\z@}
 %\tp@provide@setting{tpCaptionSep}{\z@}
\tp@provide@setting{tpAboveFloatSkip}{\z@}
\tp@provide@setting{tpBelowFloatSkip}{\z@}
\tp@provide@setting{tpSubFloatSep}{\tp@subflt@sep}
\tp@provide@setting{tpLeftSkip}{\z@}
\tp@provide@setting{tpRightSkip}{\z@}
%    \end{macrocode}
% Resetting the subfloats
%    \begin{macrocode}
\def\tp@reset@subflts{%
  \global\tp@subflts=\z@
  \global\tp@subflt@maxheight=\z@\relax
}

%    \end{macrocode}
% User macro to (re-)declare a new tpFloat
% environment. \lstinline{#1}: inherit from, \lstinline{#2}:
% environment name, \lstinline{#3}: caption name, \lstinline{#4}:
% listof, \lstinline{#5} parameters.
%    \begin{macrocode}
\def\tpDeclareFloat{\@ifnextchar[{\@tpDeclareFloat}{\@tpDeclareFloat[]}}%]
\def\@tpDeclareFloat[#1]#2#3#4#5{%
  \def\tp@float@name{#2}%
  \expandafter\ifx\csname\tp@float@name\endcsname\relax
    \newenvironment{#2}[1][]{\tp@float[##1]{#3}{#4}}{\endtp@float}%
    \newenvironment{#2*}[1][]{\let\tp@do@dblfloat\relax\tp@float[##1]{#3}{#4}}{\endtp@dblfloat}%
  \fi
  #5}

%    \end{macrocode}
% User macros. The optional argument corresponds to the placement
% parameter in standard \LaTeX\ environments.
%    \begin{macrocode}
\tpDeclareFloat{tpFigure}{figure}{lof}{%
}
\tpDeclareFloat{tpTable}{table}{lot}{%
}


%    \end{macrocode}
% Internal macros for generalized floating environments. They provide
% the macros used to compose floats to make sure they are used only
% within the \lstinline{tp@float}-derived environments.
%
% \#1: options, xml attribute syntax (\lstinline{attr="value"}),
% \begin{description}
% \item[float-pos] \LaTeX's float position, e.g. \lstinline{h}, \lstinline{t}, etc.\\
% \item[class] Class\\
% \end{description}
%    \begin{macrocode}
\let\ltx@label\label
\def\tp@str@landscape{landscape}
\def\tp@str@default{default}
\AtBeginDocument{\@ifpackageloaded{htmltabs}{\global\let\tp@uses@htmltabs\relax}{}}%

\def\tp@float[#1]#2#3{%
  \savenotes
  \begingroup
    \tp@reset@subflts
    \tp@reset@caption{0}%
    \def\tp@captype{#2}%
    \def\tp@caplisttype{#3}%
    \global\let\tp@current@class\relax
    \if!#1!\else
      \cssParseAttributes{#2}{#1}%
      \expandafter\ifx\csname css@#2@attr@class\endcsname\relax\else
        \expandafter\global\expandafter\let\expandafter\tp@current@class\csname css@#2@attr@class\endcsname
        \message{^^Jtranspect-floats.sty: Class of float \tp@captype\space is: `\csname css@#2@attr@class\endcsname'.}%
        \cssUseClass{default}%
        \expandafter\cssUseClass\expandafter{\csname css@#2@attr@class\endcsname}%
      \fi
    \fi
    \expandafter\ifx\csname css@#2@attr@float-pos\endcsname\relax
      \let\tp@fps\@empty
    \else
      \expandafter\let\expandafter\tp@fps\csname css@#2@attr@float-pos\endcsname
    \fi
    \def\tp@bcnthack{\bgroup\ifx\tp@is@subflt\relax\else\tp@subflts=\z@\relax\fi}%
    \def\tp@ecnthack{\egroup}%
    \tpProvideMacro{tpFig}{\tp@bcnthack}{\tp@ecnthack}{tp@subflt@fig@\the\tp@subflts}%
    \tpProvideMacro{tpCaption}{\tp@bcnthack}{\tp@ecnthack}{tp@subflt@cpt@\the\tp@subflts}%
    \tpProvideMacro{tpLegend}{\tp@bcnthack}{\tp@ecnthack}{tp@subflt@lgd@\the\tp@subflts}%
    \tpProvideMacro{tpSource}{\tp@bcnthack}{\tp@ecnthack}{tp@subflt@src@\the\tp@subflts}%
    \tpProvideMacro{tpNumber}{\tp@bcnthack}{\tp@ecnthack}{tp@subflt@lbl@\the\tp@subflts}%
    \tpProvideMacro{tpLabel}{\tp@bcnthack}{\tp@ecnthack}{tp@subflt@ref@\the\tp@subflts}%
    \parindent\z@
    \ifx\tp@do@dblfloat\relax\linewidth\dimexpr2\columnwidth+\columnsep\relax\fi
    \expandafter\ifx\csname css@#2@attr@orientation\endcsname\tp@str@landscape
      \linewidth\textheight
      \def\tp@fps{p}%
    \else
      \ifx\tp@fps\@empty
        \expandafter\addvspace\expandafter{\tp@use@setting{tpAboveFloatSkip}}%
      \else
        \expandafter\addvspace\expandafter{\tp@use@setting{tpIntextSkipAbove}}%
      \fi
    \fi
    \ifx\tp@uses@htmltabs\relax
      \global\setbox\htTableBox\box\voidb@x
      \let\htOutputTable\relax
    \fi
  }

%    \end{macrocode}
% 1st iteration: only reading. This step happens when the contents of the float environment are read.
%    \begin{macrocode}
\RequirePackage{rotating}
\def\endtp@float{%
  \def\@tempa{h!}\ifx\tp@fps\@tempa\let\tp@fps\@empty\fi
  \ifx\tp@fps\@empty\else
    \def\@tempa{h}\ifx\tp@fps\@tempa\def\tp@fps{ht!}\fi
    \let\tp@b@flt\tp@captype%
    \expandafter\ifx\csname css@\tp@captype @attr@orientation\endcsname\tp@str@landscape
      \edef\tp@b@flt{sideways\tp@captype}%
    \fi
    \ifx\tp@do@dblfloat\relax
      \expandafter\def\expandafter\tp@b@flt\expandafter{\tp@b@flt*}%
    \fi
    \expandafter\expandafter\expandafter\begin\expandafter\expandafter\expandafter{\expandafter\tp@b@flt\expandafter}\expandafter[\tp@fps]%
  \fi
  \csname tp@\tp@captype @float\endcsname
  \ifx\tp@fps\@empty
    \expandafter\vskip\tp@use@setting{tpBelowFloatSkip}\relax%
  \else
    \expandafter\vskip\tp@use@setting{tpIntextSkipBelow}\relax%
    \expandafter\end\expandafter{\tp@b@flt}%
  \fi
  \endgroup
  \global\let\tp@current@class\relax
  \spewnotes}
\def\endtp@dblfloat{\let\tp@do@dblfloat\relax\endtp@float}%
%    \end{macrocode}
% Sub-routine for tables
%    \begin{macrocode}
\def\tp@table@float{%
  \expandafter\ifx\csname tp@subflt@cpt@0\endcsname\relax
  \else
    \tp@make@caption{0}{Top}%
  \fi
  \ifx\htTableBox\@undefined
  \else
    \ifvoid\htTableBox\else
      \box\htTableBox
    \fi
  \fi
  \expandafter\ifx\csname tp@subflt@cpt@0\endcsname\relax\else
    \tp@make@caption{0}{Below}%
  \fi}

%    \end{macrocode}
% This macro adds the entry for the float to the corresponding list.
%    \begin{macrocode}
\def\tpListEntry{\tp@lo@cpt\space\tp@lo@src}
\def\tp@add@list#1{%
  \expandafter\expandafter\expandafter\def\expandafter\expandafter\expandafter\tp@lo@cpt\expandafter\expandafter\expandafter{\csname tp@subflt@cpt@#1\endcsname}%
  \expandafter\expandafter\expandafter\def\expandafter\expandafter\expandafter\tp@lo@src\expandafter\expandafter\expandafter{\csname tp@subflt@src@#1\endcsname}%
  \expandafter\expandafter\expandafter\def\expandafter\expandafter\expandafter\tp@lo@lbl\expandafter\expandafter\expandafter{\csname tp@subflt@lbl@#1\endcsname}%
  \expandafter\expandafter\expandafter\def\expandafter\expandafter\expandafter\tp@lo@ref\expandafter\expandafter\expandafter{\csname tp@subflt@ref@#1\endcsname}%
  \expandafter\ifx\tp@lo@src\relax\def\tp@lo@src{}\fi
  \expandafter\ifx\tp@lo@cpt\relax\else
    \expandafter\ifx\tp@lo@lbl\relax
      \addcontentsline{\tp@caplisttype}{\tp@captype}{\tpListEntry}%
    \else
      \tp@make@anchor%
      \addcontentsline{\tp@caplisttype}{\tp@captype}{\string\numberline{\tp@lo@lbl}\tpListEntry}%
    \fi
    \tp@reset@caption{#1}%
  \fi}

%    \end{macrocode}
% This is a rundimentary implementation of hyperref's anchor mechanism
% to make labels work:
%    \begin{macrocode}
\def\tp@make@anchor{%
  \expandafter\ifx\tp@lo@ref\relax\else
    \let\@currentlabel\tp@lo@lbl
    \expandafter\H@refstepcounter\expandafter{\tp@captype}%
    \expandafter\hyper@makecurrent\expandafter{\tp@captype}%
    \let\Hy@tempa\Hy@float@caption
    \expandafter\hyper@@anchor\expandafter{%
      \@currentHref
    }{\relax}%
    \expandafter\ltx@label\expandafter{\tp@lo@ref}%
  \fi}


\let\ltx@includegraphics\includegraphics
\def\tp@includesubgraphics{\@ifnextchar [\@tp@includesubgraphics{\@tp@includesubgraphics[]}}%]
\def\@tp@includesubgraphics[#1]#2{%
  \def\@igopts{#1,width=\hsize}%
  \expandafter\ltx@includegraphics\expandafter[\@igopts]{#2}%
}

%    \end{macrocode}
% Sub-routine for figures
%    \begin{macrocode}
\def\tp@figure@float{%
  \tp@total@flt@width=\hsize\relax
  \expandafter\@tempdima\tp@use@setting{tpRightSkip}\relax
  \ifdim\@tempdima>\z@\relax\advance\tp@total@flt@width-\@tempdima\relax\fi
  \expandafter\@tempdima\tp@use@setting{tpLeftSkip}\relax
  \ifdim\@tempdima>\z@\relax\advance\tp@total@flt@width-\@tempdima\relax\fi
  \hskip\@tempdima
  \vtop{%
    \hsize\tp@total@flt@width
    \expandafter\ifx\csname tp@subflt@cpt@0\endcsname\relax\else
      \tp@make@caption{0}{Top}%
    \fi
    \bgroup
      \expandafter\tp@subflt@sep=\tp@use@setting{tpSubFloatSep}\relax%
      \@tempdima=\z@\relax
      \ifnum\tp@subflts=\z@\relax
        \ifx\tp@current@class\relax\else\let\includegraphics\tp@includesubgraphics\fi
        \tp@subflts\@ne
        \expandafter\def\csname tp@subflt@res@width@1\endcsname{\tp@total@flt@width}%
        \tp@create@mp{\csname tp@subflt@fig@0\endcsname\nobreak}{t}%
        \tp@subflts\z@
      \else
%    \end{macrocode}
% 2nd iteration: calculate the ratio between each subfigure's height and the height of the largest subfigure
%    \begin{macrocode}
        \sh@iterate{\@tempcnta}{\@ne}{\tp@subflts}{%
          \edef\@tempa{\CalcRatio{\tp@subflt@maxheight}{\csname tp@subflt@height@\the\@tempcnta\endcsname}}%
          \ifnum\@tempcnta>\@ne
            \advance\tp@total@flt@width-\tp@subflt@sep\relax%
          \fi
          \expandafter\@tempdimc\csname tp@subflt@width@\the\@tempcnta\endcsname\relax
          \@tempdimb=\@tempa\@tempdimc\relax
          \expandafter\edef\csname  tp@subflt@adj@width@\the\@tempcnta\endcsname{\the\@tempdimb}%
          \advance\@tempdima\@tempdimb
        }% /sh@iterate
        \@tempcnta\z@
%    \end{macrocode}
% 3rd iteration: Calculate width ratio of all adjusted subfigures against total width of all figures plus separators and print:
%    \begin{macrocode}
        \sh@iterate{\@tempcnta}{\@ne}{\tp@subflts}{%
          \edef\@tempa{\CalcRatio{\csname tp@subflt@adj@width@\the\@tempcnta\endcsname}{\@tempdima}}%
          \expandafter\edef\csname tp@subflt@res@width@\the\@tempcnta\endcsname{\dimexpr\@tempa\tp@total@flt@width\relax}%
          \expandafter\ifx\csname tp@subflt@cpt@\the\@tempcnta\endcsname\relax\else\let\tp@has@subcapt\relax\fi
        }%
        \ifx\tp@has@subcapt\relax\tp@create@mp{\tp@make@caption{\the\@tempcnta}{Top}}{\tp@use@setting{tpCaptionVAlignTop}}\par\fi%
        \tp@create@mp{\csname tp@subflt@fig@\the\@tempcnta\endcsname}{t}%
        \ifx\tp@has@subcapt\relax \par\tp@create@mp{\tp@make@caption{\the\@tempcnta}{Below}}{\tp@use@setting{tpCaptionVAlignBottom}}\par\fi%
        \par
      \fi
    \egroup
    \expandafter\ifx\csname tp@subflt@cpt@0\endcsname\relax\else
      \tp@make@caption{0}{Below}%
    \fi
  }% /vbox
}
%    \end{macrocode}
% creates a minipage of a fixed width for each part of a sub figure
%    \begin{macrocode}
\def\tp@create@mp#1#2{%
  \sh@iterate{\@tempcnta}{\@ne}{\tp@subflts}{%
    \ifnum\@tempcnta>\@ne\relax\hfill\fi
    \begin{minipage}[#2][][b]{\the\csname tp@subflt@res@width@\the\@tempcnta\endcsname}%
      \ifx\tp@current@class\relax\else\let\includegraphics\tp@includesubgraphics\fi
      #1
    \end{minipage}%
  }}

%    \end{macrocode}
% prints a sub figure without a minipage
%    \begin{macrocode}

%    \end{macrocode}
% Mechanism to determine the correct order of caption constituents
%    \begin{macrocode}
\def\tp@use@label#1{%
  \expandafter\ifx\csname tp@subflt@lbl@#1\endcsname\relax\else
    \bgroup
      \expandonce{\tp@use@setting{tpCaptionLabelFont}}%
      \csname tp@subflt@lbl@#1\endcsname%
    \egroup
    \tp@use@setting{tpCaptionLabelSep}%
  \fi
}
\def\tp@use@legend#1{%
  \expandafter\ifx\csname tp@subflt@lgd@#1\endcsname\relax\else
    \tp@use@setting{tpCaptionLegendSep}%
    \bgroup
      \expandonce{\tp@use@setting{tpCaptionLegendFont}}%
      \expandonce{\csname tp@subflt@lgd@#1\endcsname}%
    \egroup
  \fi
}
\def\tp@use@source#1{%
  \expandafter\ifx\csname tp@subflt@src@#1\endcsname\relax\else
    \tp@use@setting{tpCaptionSourceSep}%
    \bgroup
      \expandonce{\tp@use@setting{tpCaptionSourceFont}}%
      \expandonce{\csname tp@subflt@src@#1\endcsname}%
    \egroup
  \fi
}
\def\tp@use@caption#1{%
  \bgroup
    \expandonce{\tp@use@setting{tpCaptionFont}}%
    \expandonce{\csname tp@subflt@cpt@#1\endcsname}%
  \egroup}

%    \end{macrocode}
% split comma separated list of constituents given by
% \lstinline{\tpCaptionOrderTop} and
% \lstinline{\tpCaptionOrderBottom}, respectively
%    \begin{macrocode}
\def\tp@traverse@cap@order #1,#2,\@nil#3{%
  \if!#1!\else
    \expandafter\ifx\csname tp@use@#1\endcsname\relax\else%
      \csname tp@use@#1\endcsname{#3}%
    \fi
    \if!#2!\else
      \tp@traverse@cap@order#2,\@nil{#3}%
    \fi
  \fi
}
\def\tp@caption@order#1#2{%
  \edef\@tp@cap@odr{\tp@use@setting{tpCaptionOrder#2}}%
  \expandafter\tp@traverse@cap@order\@tp@cap@odr,,\@nil{#1}%
}

%    \end{macrocode}
% prints the caption
%    \begin{macrocode}
\def\tp@make@caption#1#2{%
  \def\@argii{#2}\def\@tempa{Below}%
  \ifx\@tempa\@argii
    \vskip\tp@use@setting{tpCaptionBelowSep}%
  \fi
  \bgroup%
    \expandonce{\tp@use@setting{tpCaptionFormat}}%
    \expandonce{\tp@use@setting{tpCaptionFormat#2}}%
    \strut\tp@caption@order{#1}{#2}\strut%
    \nobreak\par
  \egroup
  \ifx\@tempa\@argii
    \tp@add@list{#1}%
  \else
    \vskip\tp@use@setting{tpCaptionTopSep}%
  \fi
}
%    \end{macrocode}
% Resetting caption parameters
%    \begin{macrocode}
\def\tp@reset@caption#1{%
  \ltx@LocalExpandAfter\global\expandafter\let\csname tp@subflt@fig@#1\endcsname\relax%
  \ltx@LocalExpandAfter\global\expandafter\let\csname tp@subflt@cpt@#1\endcsname\relax%
  \ltx@LocalExpandAfter\global\expandafter\let\csname tp@subflt@src@#1\endcsname\relax%
  \ltx@LocalExpandAfter\global\expandafter\let\csname tp@subflt@lbl@#1\endcsname\relax%
  \ltx@LocalExpandAfter\global\expandafter\let\csname tp@subflt@lgd@#1\endcsname\relax%
  \ltx@LocalExpandAfter\global\expandafter\let\csname tp@subflt@ref@#1\endcsname\relax%
  }
%    \end{macrocode}
% shell for sub floats
%    \begin{macrocode}
\def\tpSubFloat{%
  \let\tp@is@subflt\relax
  \global\advance\tp@subflts\@ne
  \ignorespaces}

\def\endtpSubFloat{%
  \setbox\tp@subfltbox\hbox{\csname tp@subflt@fig@\the\tp@subflts\endcsname}%
  \expandafter\xdef\csname tp@subflt@width@\the\tp@subflts\endcsname{\the\wd\tp@subfltbox}%
  \expandafter\xdef\csname tp@subflt@height@\the\tp@subflts\endcsname{\the\ht\tp@subfltbox}%
  \expandafter\xdef\csname tp@subflt@depth@\the\tp@subflts\endcsname{\the\dp\tp@subfltbox}%
  \expandafter\ifdim\csname tp@subflt@height@\the\tp@subflts\endcsname>\tp@subflt@maxheight\relax
    \expandafter\global\expandafter\tp@subflt@maxheight=\csname tp@subflt@height@\the\tp@subflts\endcsname\relax
  \fi\ignorespaces}


%    \end{macrocode}
% Hooks to change the text font of a standard \LaTeX\ tabular
%    \begin{macrocode}
\def\tablefont{\small}
\def\arraystretch{1.3}
\def\@tabular{%
  \leavevmode
  \hbox \bgroup\tablefont $\col@sep\tabcolsep \let\d@llarbegin\begingroup%$
                                    \let\d@llarend\endgroup
  \ifx\ST@tableformat\@undefined\gdef\@tablefont{\tablefont}\fi
  \@tabarray}
\let\@classzold\@classz
\def\@classz{%
   \expandafter\ifx\d@llarbegin\begingroup
     \toks \count@ =
     \expandafter{\expandafter\@tablefont\the\toks\count@}%
   \fi
   \@classzold}
\def\endtabular{%
  \endarray
  $\egroup}%$
\expandafter\let\csname endtabular*\endcsname=\endtabular
%    \end{macrocode}
%
% \section{Class handler}
%
% % TODO: in eigene csstex.sty auslagern und dann hier und in htmltabs einbinden
%
% Declaring float classes.
% \begin{description}
% \item[\#1] scope: \lstinline{figure} or \lstinline{table}.
% \item[\#2] class name.  If left empty, \lstinline{default} is used
%   instead.
% \item[\#3] properties
% \end{description}
%    \begin{macrocode}
\def\CSSDeclareClass{\@ifnextchar [{\css@set@class}{\css@set@class[]}}%]

\gdef\css@@class@default{}
\gdef\css@figure@class@default{}
\gdef\css@table@class@default{}

\def\css@set@class[#1]#2#3{%
  \def\@argii{#2}\ifx\@argii\@empty\let\@argii\tp@str@default\fi%
  \expandafter\def\csname css@#1@class@\@argii\endcsname{%
    #3%
  }%
}

\def\cssUseClass#1{%
  \expandafter\ifx\csname css@\tp@captype @class@#1\endcsname\relax
    \expandafter\ifx\csname css@@class@#1\endcsname\relax
      \PackageError{transpect-flaots.sty}{Class `#1' not defined!}{Please declare the class `#1'!}%
    \fi
  \fi
  \let\ht@cur@element\tp@captype
  \csname css@@class@#1\endcsname%
  \csname css@\tp@captype @class@#1\endcsname%
}




% %%%%%%%%%%
% % LEGACY %
% %%%%%%%%%%
% % \renewcommand \thefigure
% %      {\@arabic\c@figure}
% % %% TODO: Projektspezifisch!
% % \@removefromreset{figure}{chapter}
% % \renewcommand \thetable
% %      {\@arabic\c@table}
% % %% TODO: Projektspezifisch!
% % \@removefromreset{table}{chapter}
% % \captionsetup{%
% %    format=plain
% %   ,labelformat=empty
% %   ,font+=it
% %   ,singlelinecheck=false
% %   ,justification=RaggedRight
% %   ,listformat=empty
% % }
% %    \end{macrocode}
% % \section{Sources}
% % Many Transpect projects use different markup for the caption of a floating object and its source.
% %    \begin{macrocode}
% % \newlength\aboveSourceSkip \aboveSourceSkip0mm
% % \newcommand\transpectBild[3][]{\tr@nspectFloat{#1}{#2}{#3}}
% % \newcommand\transpectTab[3][]{\aboveSourceSkip1mm\let\use@depth\relax\tr@nspectFloat{#1}{#2}{#3}}
% % \let\transcriptBild\transpectBild
% % \let\transcriptTab\transpectTab
% % \newbox\tr@nspectFlt
% % \newdimen\tr@nspectFltWd
% % \newdimen\tr@nspectWd
% % \newdimen\tr@nspectHt
% % \newdimen\tr@nspectSep
% %    \end{macrocode}
% % \section{Scaling graphics to a common height}
% % The \lstinline{\sameheight} macro is used to scale all figures
% % incorporated via \lstinline{\includesubgraphics} to a common height
% % such that the line of graphics fills \lstinline{\shhsize} (\lstinline{\hsize} by default).
% %    \begin{macrocode}
% % \newbox\@includesubgraphicsbox
% % \newcount\c@includesubgraphics \c@includesubgraphics\z@
% % \newdimen\includesubgraphics@maxheight
% % \newdimen\subgraphicssep \subgraphicssep\fboxsep
% % \newdimen\sh@margins
% % \RequirePackage{ltxcmds}
% % \newcommand*\includesubgraphics[2][]{%
% %   \global\advance\c@includesubgraphics\@ne
% %   \ltx@LocalExpandAfter\gdef\csname subgraphics@caption@\the\c@includesubgraphics\endcsname{#1}%
% %   \ltx@LocalExpandAfter\gdef\csname subgraphics@name@\the\c@includesubgraphics\endcsname{#2}%
% %   \setbox\@includesubgraphicsbox\hbox{\includegraphics{#2}}%
% %   \ltx@LocalExpandAfter\xdef\csname subgraphics@width@\the\c@includesubgraphics\endcsname{\the\wd\@includesubgraphicsbox}%
% %   \ltx@LocalExpandAfter\xdef\csname subgraphics@height@\the\c@includesubgraphics\endcsname{\the\ht\@includesubgraphicsbox}%
% %   \expandafter\ifdim\csname subgraphics@height@\the\c@includesubgraphics\endcsname>\includesubgraphics@maxheight\relax
% %     \ltx@LocalExpandAfter\global\expandafter\includesubgraphics@maxheight=\csname subgraphics@height@\the\c@includesubgraphics\endcsname\relax
% %   \fi}
% %    \end{macrocode}
% %
% %    \begin{macrocode}
% % \def\sameheight#1{%
% %   \bgroup
% %     \tp@reset@subflts
% % %    \end{macrocode}
% % %% measurement
% % %    \begin{macrocode}
% %     \setbox\@tempboxa\hbox{#1}%
% % %    \end{macrocode}
% % %% 1st iteration: Calculate widths of all subfigures when scaled up to the highest subfigure's height:
% % %% \lstinline{\@tempdima}: total sum of those widths
% % %    \begin{macrocode}
% %     \@tempdima=\z@\relax
% %     \sh@iterate{\@tempcnta}{\@ne}{\tp@subflts}{%
% %       \edef\@tempa{\CalcRatio{\tp@subflt@maxheight}{\csname tp@subflt@height@\the\@tempcnta\endcsname}}%
% %       \ifnum\@tempcnta>\@ne\global\advance\@shhsize-\tp@subflt@sep\relax\fi
% %       \expandafter\@tempdimc\csname tp@subflt@width@\the\@tempcnta\endcsname\relax
% %       \@tempdimb=\@tempa\@tempdimc\relax
% %       \expandafter\edef\csname  tp@subflt@adj@width@\the\@tempcnta\endcsname{\the\@tempdimb}%
% %       \global\advance\@tempdima\@tempdimb
% %   }%
% % %    \end{macrocode}
% % %% 2nd iteration: Calculate width ratio of all adjusted subfigures against total width of all figures plus separators and output:
% % %    \begin{macrocode}
% %   \@tempcnta\z@
% %   \sh@iterate{\@tempcnta}{\@ne}{\tp@subflts}{%
% %     \edef\@tempa{\CalcRatio{\csname tp@subflt@adj@width@\the\@tempcnta\endcsname}{\@tempdima}}%
% %     \@tempdimb\@tempa\@shhsize\relax
% %     %\edef\@tempb{\csname subgraphics@name@\the\@tempcnta\endcsname}%
% %     \ifnum\@tempcnta>\@ne\hskip\subgraphicssep\fi
% %     \def\@igopts{width=\linewidth}%
% %     \begin{minipage}[b]{\@tempdimb}%
% %       \captionsetup{margin=\z@}%
% %       %\csname subgraphics@caption@\the\@tempcnta\endcsname%
% %       \csname tp@subflt@fig@\the\@tempcnta\endcsname
% %       %\expandafter\expandafter\expandafter\includegraphics\expandafter\expandafter\expandafter[\expandafter\@igopts\expandafter]\expandafter{\@tempb}%
% %       % \ifx\c@psource\@undefined\else
% %       %   \vskip\dimexpr-\topskip+\aboveSourceSkip+.5mm\relax
% %       %   \vtop{\captionsetup{font=footnotesize}\caption*{\c@psource}}%
% %       %   \global\let\c@psource\@undefined
% %       % \fi
% %     \end{minipage}%
% %   }%
% %   \egroup
% % }
% %    \end{macrocode}
% % END: sameheight
% %
% % \#1: layout variant, \#2 subufigures
% %    \begin{macrocode}
% % \newcommand\transriptFixedFigure[3][]{%
% %   \bgroup
% %   \shhsize\hsize
% %   \subgraphicssep2mm
% %   \ifx#2A\relax
% %     \@tempdima\z@
% %   \else
% %     \ifx#2B\relax
% %       \@tempdima20mm
% %     \else
% %       \ifx#2C\relax
% %         \@tempdima30mm
% %       \else
% %         \ifx#2D\relax
% %           \@tempdima50mm
% %         \fi
% %       \fi
% %     \fi
% %   \fi
% %   \advance\shhsize-\@tempdima
% %   \sh@margins.5\@tempdima
% %   \vskip1\baselineskip
% %   \captionsetup{margin=\sh@margins\relax}%
% %   \if!#1!\else#1\fi
% %   \hskip\sh@margins\sameheight{#3}%
% %   \ifx\c@psource\@undefined\else
% %     \vskip\aboveSourceSkip
% %     \captionsetup{font=footnotesize,skip=-1mm}%
% %     \caption*{\c@psource}%
% %     \global\let\c@psource\@undefined
% %   \fi
% %   \egroup}
% % \def\capsource#1{\def\c@psource{#1}}%
% % \let\oldfigure\figure
% % \let\oldendfigure\endfigure
% % \renewenvironment{figure}{\savenotes\oldfigure}{\oldendfigure\spewnotes}
% % \long\def\tr@nspectFloat#1#2#3{%
% %   \def\@rgi{#1}%
% %   \setbox\tr@nspectFlt\hbox{#3}%
% %   \tr@nspectFltWd=\wd\tr@nspectFlt\relax
% %   \tr@nspectHt=\dimexpr\ht\tr@nspectFlt\ifx\use@depth\relax+\dp\tr@nspectFlt\fi\relax
% %   \tr@nspectSep\dimexpr(\textwidth-\tr@nspectFltWd)/2\relax
% %   \vskip\baselineskip
% %   \ifdim\tr@nspectFltWd<.5\textwidth\relax
% %     \bgroup
% %       %\tr@nspectSep4mm
% %       \tr@nspectWd\dimexpr\textwidth-\tr@nspectFltWd\relax
% %       \noindent\begin{minipage}[\ifx\use@depth\relax t\else b\fi][\tr@nspectHt][t]{\tr@nspectWd}%
% %         \captionsetup{justification=RaggedRight}%
% %         \captionof{figure}{#2}%
% %         \ifx\@rgi\@empty\else
% %           \vfill
% %         \captionsetup{font=footnotesize,skip=\z@}%
% %         \caption*{#1}%
% %         \vspace*{\dimexpr-\dp\tr@nspectFlt-1mm}%
% %        \fi
% %       \end{minipage}\hfill
% %       \begin{minipage}[t][\tr@nspectHt]{\tr@nspectFltWd}\centering
% %         \unhbox\tr@nspectFlt%
% %       \end{minipage}%
% %     \egroup
% %   \else
% %     \noindent\hskip\tr@nspectSep
% %     \begin{minipage}{\tr@nspectFltWd}\centering%
% %       \captionof{figure}{#2}%
% %       \unhbox\tr@nspectFlt\par
% %       \ifx\@rgi\@empty\else
% %         \captionsetup{font=footnotesize,skip=-1mm}\caption*{#1}%
% %         \vspace*{\dimexpr-\dp\tr@nspectFlt-1mm}%
% %       \fi
% %     \end{minipage}%
% %   \fi
% %   \vskip\baselineskip
% %   \global\let\use@depth\@undefined
% %   \tr@nspectWd\z@% \tr@nspectSep\z@
% % }
