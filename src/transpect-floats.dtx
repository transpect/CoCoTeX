% \chapter{transpect-floats.dtx}
% This module provides handlers for floating objects like tables and
% figures common to all Transpect projects
%
%    \begin{macrocode}[gobble=1]
%<*floats>
%    \end{macrocode}
%    \begin{macrocode}
%%
%% module for le-tex transpect.cls that extends floating objects.
%%
%% Maintainer: p.schulz@le-tex.de
%%
%% lualatex  -  texlive > 2019
%%
\NeedsTeXFormat{LaTeX2e}[2018/12/01]
\ProvidesPackage{transpect-floats}
    [\filedate \fileversion le-tex transpect floats module]
\DeclareOptionX{nofigs}{\global\let\tp@nofigs\relax}
\ProcessOptionsX
%    \end{macrocode}
% Hard requirements:
%    \begin{macrocode}
\RequirePackage{transpect-common}
\RequirePackage{rotating}
\RequirePackage{grffile}
\RequirePackage[Export]{adjustbox}
\usepackage{stfloats}
\setcounter{dblbotnumber}{5}
%    \end{macrocode}
% for automatic typesetting and float positioning, we use very high tolerances from standard \LaTeX:
%    \begin{macrocode}
\def\topfraction{0.9}
\def\textfraction{0.1}
\def\bottomfraction{0.8}
\def\totalnumber{8}
\def\topnumber{8}
\def\bottomnumber{8}
\def\floatpagefraction{0.8}
\@fptop\z@
\@fpbot\@flushglue
%    \end{macrocode}
% some reserved registers
%    \begin{macrocode}
\newbox  \tp@floatbox
\newbox  \tp@subfltbox
\newcount\tpSubFloatCnt       \tpSubFloatCnt=\z@\relax
\newcount\tp@subfloat@row     \tp@subfloat@row\z@
\newcount\tp@subfloat@col     \tp@subfloat@col\z@
\newcount\tp@int@flt@cnt \tp@int@flt@cnt\z@

\newdimen\tp@subflt@maxheight \tp@subflt@maxheight=\z@\relax
\newdimen\tp@subflt@sep       \tp@subflt@sep=\fboxsep\relax
\newdimen\tp@total@flt@width  \tp@total@flt@width=\textwidth\relax
\newdimen\tp@total@flt@height \tp@total@flt@height=\textwidth\relax
\newdimen\tp@total@flt@depth  \tp@total@flt@depth=\textwidth\relax

\newdimen\tp@flt@marg@r       \tp@flt@marg@r=\z@\relax
\newdimen\tp@flt@marg@l       \tp@flt@marg@l=\z@\relax

\def\tp@str@landscape{landscape}
\def\tp@str@figure{figure}
\def\tp@str@table{table}
\def\tp@str@bottom{bottom}
\def\tp@str@top{top}

%    \end{macrocode}
% Storing the final definition of \lstinline{\label}:
%    \begin{macrocode}
\AtBeginDocument{%
  \global\let\tpltx@label\label
  \ifx\tp@nofigs\relax
    \renewcommand\includegraphics[2][]{}%
  \fi
  \global\let\tpltx@includegraphics\includegraphics
  \@ifpackageloaded{htmltabs}
    {\global\let\tp@uses@htmltabs\relax
     \def\ht@adjust@linewidth{%
       \advance\ht@h@offset\leftskip
       \advance\ht@h@offset\@totalleftmargin
       %\advance\linewidth-\leftskip
       \advance\linewidth-\rightskip
     }%
   }{}%
}%

%    \end{macrocode}
% Resetting the subfloats. \#1 is the caption family, \#2 is the
% abbreviation of the caption list (e.g., standard {\LaTeX} uses
% \lstinline{lof} for the List of Figures, \lstinline{lot} for the
% List of Tables).
%    \begin{macrocode}

\def\tp@reset@components#1{%
  \@tempcnta=#1\relax
  \@tp@reset@components
  \global\let\@tp@reset@components\relax
}
\global\let\@tp@reset@components\relax

\def\tp@reset@defaults#1#2{%
  \global\tpSubFloatCnt=\z@
  \global\let\tp@has@capt@top\@undefined
  \global\let\tp@has@capt@bottom\@undefined
  \global\let\tp@has@subcapt@top\@undefined
  \global\let\tp@has@subcapt@bottom\@undefined
  \global\let\tp@@savecontentsline\@empty
  \global\tp@subfloat@row=\z@
  \global\tp@subfloat@col=\z@
  \global\tp@subflt@maxheight=\z@\relax
  \tp@reset@components{0}%
  \def\tp@captype{#1}%
  \def\tp@caplisttype{#2}%
  \let\ht@cur@element\tp@captype
  \global\let\tp@current@class\relax
  \global\let\tp@subfloat@parent=f\relax
}
%    \end{macrocode}
% Handles the optional argument of float environments. \#1 is the
% content of the optional argument, \#2 is the caption family.
%    \begin{macrocode}
\def\tp@get@flt@attr#1#2{%
  \if!#1!\else
    \tpParseAttributes{#2}{#1}%
    \expandafter\ifx\csname tp@#2@attr@class\endcsname\relax\else
      \expandafter\global\expandafter\let\expandafter\tp@current@class\csname tp@#2@attr@class\endcsname
      \message{^^Jtranspect-floats.sty: Class of float \tp@captype\space is: `\csname tp@#2@attr@class\endcsname'.}%
      \tpUseClass{default}{\tp@captype}%
      \expandafter\tpUseClass\expandafter{\csname tp@#2@attr@class\endcsname}{\tp@captype}%
    \fi
  \fi
  \tp@get@flt@pos{#2}}
%    \end{macrocode}
% Handler for determining the floating position
%    \begin{macrocode}
\def\tp@get@flt@pos#1{%
  \expandafter\ifx\csname tp@#1@attr@float-pos\endcsname\relax
    \let\tp@fps\@empty
  \else
    \expandafter\let\expandafter\tp@fps\csname tp@#1@attr@float-pos\endcsname
  \fi
  \def\@tempa{h!}\ifx\tp@fps\@tempa\let\tp@fps\@empty\fi
  \def\@tempa{h}\ifx\tp@fps\@tempa\def\tp@fps{htbp!}\fi
  \ifx\tp@do@dblfloat\relax
    \ifx\tp@fps\@empty\def\tp@fps{htpb!}\fi% 11514
    \linewidth\dimexpr2\columnwidth+\columnsep\relax
  \fi
  \expandafter\ifx\csname tp@#1@attr@orientation\endcsname\tp@str@landscape\relax
    \linewidth\textheight
    \def\tp@fps{p}%
  \fi}
%    \end{macrocode}
% In order to cathc the actual dimensions of the float box, we need to
% hook into \LaTeX's \lstinline{\@endfloatbox} macro:
%    \begin{macrocode}
\AtBeginDocument{%
  \let\ltx@endfloatbox\@endfloatbox
\def \@endfloatbox{%
  \ltx@endfloatbox
  \global\tp@total@flt@height=\ht\@currbox\relax%
  \global\tp@total@flt@depth=\dp\@currbox\relax%
}}

\def\tp@set@flt@pos{%
  \ifx\tp@fps\@empty
    \let\tp@b@float\relax
    \let\tp@e@float\relax
  \else
    \let\tp@b@float\tp@captype%
    \expandafter\ifx\csname tp@\tp@captype @attr@orientation\endcsname\tp@str@landscape\relax
      \edef\@tp@b@float{sideways\tp@b@float}%
      \edef\tp@b@float{\noexpand\begin{\@tp@b@float\ifx\tp@do@dblfloat\relax*\fi}}
      \edef\tp@e@float{\noexpand\end{\@tp@b@float\ifx\tp@do@dblfloat\relax*\fi}}
    \else
      \edef\tp@flt@env{\ifx\tp@do@dblfloat\relax dbl\fi}
      \edef\tp@b@float{\expandafter\noexpand\csname @x\tp@flt@env float\endcsname {\tp@captype}[\tp@fps]}
      \edef\tp@e@float{\expandafter\noexpand\csname end@\tp@flt@env float\endcsname}
    \fi
  \fi}
%    \end{macrocode}
% debug output
%    \begin{macrocode}
\def\tp@flt@debug#1{%
  \expandafter\ifx\csname tp@#1@attr@debug\endcsname\relax\else
    \message{^^Jtranspect-float debug:^^J
        Textheight:\space\the\textheight^^J
        Type:\space\space\space\space\space\space\space\tp@namespace^^J
\ifx\tp@captype\tp@str@figure
        Path: \space\space\space\space\space\space\@tp@fig@path^^J
\fi
        Class:\space\space\space\space\space\space\tp@current@class^^J
        Floatpos:\space\space\space\tp@fps^^J
        Environ:\space\space\space\space\expandafter\noexpand\tp@b@float...\expandafter\noexpand\tp@e@float^^J
        Subfloat:\space\space\space               \the\tpSubFloatCnt^^J
\ifnum\tpSubFloatCnt=\z@
        Width:\space\space\space\space\space\space\the\tp@total@flt@width^^J
        Height:\space\space\space\space\space\the\tp@total@flt@height^^J
        Depth:\space\space\space\space\space\space\the\tp@total@flt@depth^^J
\else
        Width:\space\space\space\space\space\space\expandafter\meaning\csname tp@\tp@namespace @width-\the\tpSubFloatCnt\endcsname^^J
        Height:\space\space\space\space\space     \expandafter\meaning\csname tp@\tp@namespace @height-\the\tpSubFloatCnt\endcsname^^J
        Depth:\space\space\space\space\space\space\expandafter\meaning\csname tp@\tp@namespace @depth-\the\tpSubFloatCnt\endcsname^^J
\fi
}%
  \fi
}
%    \end{macrocode}
% depending on the float position, apply the top separator
%    \begin{macrocode}
\def\tp@set@top@sep{%
  \ifx\tp@fps\@empty
    \expandafter\addvspace\expandafter{\tpUseProperty{intext-skip-top}}%
  \else
    \expandafter\addvspace\expandafter{\tpUseProperty{float-skip-top}}%
  \fi}

\def\tp@set@bot@sep{%
  \expandafter\ifx\csname tp@\tp@captype @attr@orientation\endcsname\tp@str@landscape
  \else
    \ifx\tp@fps\@empty
      \expandafter\addvspace\expandafter{\tpUseProperty{intext-skip-bottom}}%
    \else
      \expandafter\addvspace\expandafter{\tpUseProperty{float-skip-bottom}}%
    \fi
  \fi}


%    \end{macrocode}
% Declarations for caption related Components in float environments.
%    \begin{macrocode}

\def\tpMakeFltComp#1#2#3{%
  \tpProvideCountedComp{#1}{#2}{#3}{\tpSubFloatCnt}{\ifx\tp@is@subflt\relax\else\tpSubFloatCnt=\z@\relax\fi}{}%
}

\def\tpMakeFltCompL#1#2{%
  \tpMakeFltComp{tp#1}{#2}{#1}%
  \tpMakeFltComp{tpListof#1}{list-of-#2}{Listof#1}}

\def\tp@set@caption@comps{%
  \tpMakeFltCompL{Caption}{caption}%
  \tpMakeFltCompL{Legend}{legend}%
  \tpMakeFltCompL{Source}{source}%
  \tpMakeFltCompL{Number}{number}%
  \tpMakeFltComp{label}{label}{Label}%
}

%    \end{macrocode}
% \section{Caption mechanism}
%
% This sub-routine cycles through the subfloats and creates boxes for
% each component. \#1 is the content, \#2 is the vertical alignment
% within the box, \#3 is the height of the row.
%    \begin{macrocode}

\def\tp@iterate@mp#1#2#3{%
  \@tempcnta=\z@\relax
  \tp@iterate{\@tempcnta}{\@ne}{\tpSubFloatCnt}{%
    \ifnum\@tempcnta>\@ne\relax\hfill\fi
    \tp@flt@create@block{#1}{#2}{\csname tp@\tp@namespace @res@width-\the\@tempcnta\endcsname}{#3}%
  }}

%    \end{macrocode}
% creates a vbox of a fixed width and height for each component
% (block) of a subfloat. \#1 is the source of the subfloat component
% (block), \#2 is the vertical alignment of the figure within the
% vbox, and \#3 is the projected overall width of the vbox, \#4 its height.
%    \begin{macrocode}
\def\tp@flt@create@block#1#2#3#4{%
  \ifx\tp@fps\@empty\savenotes\fi
  \leavevmode
  \ifdim#4>\z@\relax
    \def\@tempa{\vbox to #4}%
  \else
    \def\@tempa{\vbox}%
  \fi
  \edef\@argii{#2}% ???
  \@tempa{%
    \hsize=#3\relax
    \leavevmode
    \ifx\@argii\tp@str@top\else\vss\fi
    \let\includegraphics\tp@includesubgraphics
    \strut#1\strut%
    \ifx\@argii\tp@str@bottom\else\vss\fi
  }%
  % \smash{\llap{\tiny\strut(#2,\the\dimexpr\ht\@tempboxa\relax+\the\dimexpr\dp\@tempboxa\relax, \the\dimexpr#4\relax)\vrule \@height\ht\@tempboxa\@depth\dp\@tempboxa\@width 1bp\relax\hskip1bp}}%
  % \box\@tempboxa
  \ifx\tp@fps\@empty\spewnotes\fi\nopagebreak
}

%    \end{macrocode}
% prints the caption
%    \begin{macrocode}
\def\tp@make@caption#1#2{%
  \bgroup
    \tpSubFloatCnt#1\relax
    \def\@argii{#2}%
    \tpUseProperty{caption-format}%
    \tpUseProperty{caption-format-#2}%
    \ifx\@argii\tp@str@bottom
      \expandafter\@tempskipa\tpUseProperty{caption-sep-bottom}\relax%
      \advance\@tempskipa\dimexpr-\topskip\relax
      \vskip\@tempskipa\relax
    \fi
    \leavevmode%
    \strut\tpUseProperty{caption-#2}\strut%
    \ifx\@argii\tp@str@bottom
    \else
      \expandafter\@tempskipa\tpUseProperty{caption-sep-top}%
      \vspace*{\@tempskipa}%
    \fi
  \egroup
}

%    \end{macrocode}
% Calculate the available maximum width for the float contents and
% captions according to the values of the \lstinline{margin-right} and
% the \lstinline{margin-left} properties.
%    \begin{macrocode}
\def\tp@flt@set@hsize{%
  \tp@total@flt@width=\hsize\relax
  \expandafter\tp@flt@marg@r\tpUseProperty{margin-right}\relax
  \ifdim\tp@flt@marg@r>\z@\relax\advance\tp@total@flt@width-\tp@flt@marg@r\relax\fi
  \expandafter\tp@flt@marg@l\tpUseProperty{margin-left}\relax
  \ifdim\tp@flt@marg@l>\z@\relax\advance\tp@total@flt@width-\tp@flt@marg@l\relax\fi
  }

%    \end{macrocode}
% High level macro to (re-)declare a new \lstinline{tpFloat}
% environment. \lstinline{#1}: inherit from, \lstinline{#2}:
% environment name, \lstinline{#3}: caption name, \lstinline{#4}:
% list-of, \lstinline{#5} parameters.
%    \begin{macrocode}
\def\tpDeclareFloat{\@ifnextchar[{\@tpDeclareFloat}{\@tpDeclareFloat[]}}%]
\def\@tpDeclareFloat[#1]#2#3#4#5{%
  \def\tp@float@name{#2}%
  \expandafter\ifx\csname #2\endcsname\relax
    \PackageInfo{transpect}{Declaring #2 environment}%
    \newenvironment{#2}[1][]{\tp@float[##1]{#3}{#4}{#2}}{\endtp@float}%
    \newenvironment{#2*}[1][]{\let\tp@do@dblfloat\relax\tp@float[##1]{#3}{#4}{#2}}{\let\tp@do@dblfloat\relax\endtp@float}%
  \else
    \PackageWarning{transpect}{#2 environment already exists. Re-Declaring.}%
    \renewenvironment{#2}[1][]{\tp@float[##1]{#3}{#4}{#2}}{\endtp@float}%
    \renewenvironment{#2*}[1][]{\let\tp@do@dblfloat\relax\tp@float[##1]{#3}{#4}{#2}}{\let\tp@do@dblfloat\relax\endtp@float}%
  \fi
  \if!#1!\else
    \expandafter\def\csname tp@float@#2@parent\endcsname{#1}%
  \fi
  \expandafter\ifx\csname tp@float@#2@properties\endcsname\relax%
    \expandafter\def\csname tp@float@#2@properties\endcsname{#5}%
  \else
    \expandafter\expandafter\expandafter\def
    \expandafter\expandafter\expandafter\@tempa
    \expandafter\expandafter\expandafter{\csname tp@float@#2@properties\endcsname}%
    \expandafter\expandafter\expandafter\def
    \expandafter\csname tp@float@#2@properties\expandafter\endcsname
    \expandafter{\@tempa #5}%
  \fi
  \tp@flt@generate@listof@handlers{#4}{#3}{#2}%
  \bgroup
    \def\tp@namespace{#2}%
    \tp@init@l@[list-of]{#4}{0}{#3}% Generate listof-Entries for first level floats
    \tp@restore@init{#4}{number-0-maxwd}%
    \tp@init@l@[list-of]{#4}{1}{sub#3}% Generate listof-Entries for sub floats
    \tp@restore@init{#4}{number-1-maxwd}%
  \egroup
}
%    \end{macrocode}
% Generate handlers for listof-entries. \#1 is the file ending, \#2 is the caption name, \#3 is the namespace (env name)
%    \begin{macrocode}
\def\tp@flt@generate@listof@handlers#1#2#3{%
%    \end{macrocode}
% The first macro that is dynamicly defined, is the Component
% collector. \#\#1 is a numeric level that represents the order of the
% listof-entries; \#\#2 is the caption type; \#\#3 is the
% content of the \lstinline{l@<level>} macro; and \#\#4 is the page
% number associated with that entry.
%    \begin{macrocode}
  \expandafter\gdef\csname tp@#1@extract@data\endcsname##1##2##3##4{%
    \tpNamespace{#3}%
    \tpCascadeProps{#3}{float}%
    \tpProvideComp{tpListofCaption}{}{}{ListofCaption}%
    \tpProvideComp{tpListofLegend}{}{}{ListofLegend}%
    \tpProvideComp{tpListofSource}{}{}{ListofSource}%
    \tpProvideComp{tpListofNumber}{}{}{ListofNumber}%
    \tpProvideComp{tpListofPage}{}{}{ListofPage}%
    \tpListofPage{\tpUseProperty{list-of-page-format}##4}%
    %\let\tpTocLink\Hy@tocdestname%
    \tp@expand@l@contents{##3}{#2}{Listof}{Caption}%%
    \tp@format@number{list-of-}{Listof}{##1}%
  }%
%    \end{macrocode}
% The second dynamically defined macro is the entry renderer. It
% applies the Listof properties and selects the components to be
% printed. \#\#1 is the caption name of the float.
%    \begin{macrocode}
  \expandafter\gdef\csname tp@#1@print@entry\endcsname##1{%
    \bgroup
      \tpUseHook{list-of-before-hook-##1}%
      \tpUseProperty{list-of-before-entry}%
      \tpUseProperty{list-of-block}%
      \tpUseHook{list-of-after-hook-##1}%
      \tpUseProperty{list-of-after-entry}%
    \egroup}%
%    \end{macrocode}
% The last macro to be defined here is the listof writer. This macro
% is responsible to write the entry into TeX's auxiliary file
% system. \#\#1 is the name of the ``level'' for the entry.
%    \begin{macrocode}
  \expandafter\gdef\csname tp@make@listof@#2\endcsname##1{%
    \tp@flt@check@empty{Number}{number}%
    \tp@flt@check@empty{Caption}{caption}%
    \tp@flt@check@empty{Legend}{legend}%
    \tp@flt@check@empty{Source}{source}%
    \expandafter\ifx\csname tp@#2@attr@nolist\endcsname\@empty\else
      \let\@tp@listof@entry\relax
      \tpIfComp{ListofCaption}{\csgappto{@tp@listof@entry}{\string\tpListofCaption{\tpUseComp{ListofCaption}}}}{}%
      \tpIfComp{ListofNumber}{\csgappto{@tp@listof@entry}{\string\tpListofNumber{\tpUseComp{ListofNumber}}}}{}%
      \tpIfComp{ListofLegend}{\csgappto{@tp@listof@entry}{\string\tpListofLegend{\tpUseComp{ListofLegend}}}}{}%
      \tpIfComp{ListofSource}{\csgappto{@tp@listof@entry}{\string\tpListofSource{\tpUseComp{ListofSource}}}}{}%
      \ifx\@tp@listof@entry\relax\else
        \protected@edef\tp@listof@entry{\@tp@listof@entry}%
        \ifx\tp@is@subflt\relax
          \tp@savecontentsline{#1}{\tp@captype}{\expandonce{\tp@listof@entry}}%
        \else
          \addcontentsline{#1}{\tp@captype}{\expandonce{\tp@listof@entry}}%
          \tp@@savecontentsline
        \fi
      \fi
    \fi
  }%
}

\def\tp@savecontentsline#1#2#3{\protected@xdef\tp@@savecontentsline{\expandonce{\tp@@savecontentsline}\noexpand\addcontentsline{#1}{#2}{#3}\relax}}

\def\tp@flt@check@empty#1#2{%
  \ifx\tp@is@subflt\relax\else\tpSubFloatCnt\z@\fi
  \tpIfComp{Listof#1}
    {}
    {\tpIfComp{#1}
      {\expandafter\expandafter\expandafter\let\expandafter\csname tp@\tp@namespace @list-of-#2-\the\tpSubFloatCnt\expandafter\endcsname\csname tp@\tp@namespace @#2-\the\tpSubFloatCnt\endcsname}
      {\csname Listof#1\endcsname{}}}}


%    \end{macrocode}
% In order to account for sub-floats, we need to re-define some of the transpect-common Component conditionals:
%    \begin{macrocode}
\long\def\tp@flt@cond{%
  \long\def\tpIfComp##1##2##3{%
    \expandafter\let\expandafter\@tempa\csname tp@\tp@namespace @##1\endcsname\relax
    \expandafter\expandafter\expandafter\ifx\@tempa\relax##3\else##2\fi}%
  \long\def\tpIfCompEmpty##1##2##3{%
    \long\def\@tempa{}%
    \expandafter\let\expandafter\@tempb\csname tp@\tp@namespace @##1\endcsname\relax
    \expandafter\expandafter\expandafter\ifx\@tempb\@tempa ##2\else ##3\fi}}
%    \end{macrocode}
% Faculity to create auto-numbered counters
%    \begin{macrocode}
\def\tp@flt@create@counters{%
  \expandafter\ifx\csname tp@\tp@captype @attr@nonumber\endcsname\relax
    \tpIfPropVal{numbering}{auto}{%
      \tpIfComp{number-0}
        {}%
        {\expandafter\global\expandafter\advance\csname c@\tp@captype\endcsname\@ne\relax
         \tp@set@label{0}}%
      \ifnum\tpSubFloatCnt=\z@\relax\else
        \@tempcnta\z@
        \tp@iterate{\@tempcnta}{\@ne}{\tpSubFloatCnt}{%
          \tpIfComp{number-\the\@tempcnta}
            {}%
            {\expandafter\ifx\csname tp@\tp@captype @attr@subfloat\endcsname\relax
               \expandafter\global\expandafter\advance\csname c@\tp@captype\endcsname\@ne\relax
               \tp@set@label{\the\@tempcnta}%
             \else
               \tp@set@sublabel{\the\@tempcnta}%
             \fi
            }}%
      \fi}{}%
  \fi}

\def\tp@set@label#1{%
  \expandafter\expandafter\expandafter\edef\expandafter\csname tp@\tp@namespace @number-#1\expandafter\endcsname\expandafter{\csname the\tp@captype\endcsname}%
}
\def\tp@set@sublabel#1{%
  \tpSetValProp{float-number}{\csname tp@\tp@namespace @number-0\endcsname}%
  \tpSetValProp{sub-number}{%
    \begingroup
      \expandonce{\tpUseProperty{sub-number-format}}%
      \relax\tpUseProperty{sub-number-before}%
      \csname @\tpUseProperty{sub-number-style}\endcsname{#1}%
      \tpUseProperty{sub-number-after}%
    \endgroup}%
  \expandafter\expandafter\expandafter\edef\expandafter\csname tp@\tp@namespace @number-#1\expandafter\endcsname\expandafter{\tpUseProperty{sub-number-block}}%
}

%    \end{macrocode}
% Mid-level Macro that provides the common floating {\LaTeX}
% environment.
%    \begin{macrocode}

\def\tp@float[#1]#2#3#4{%
  \begingroup
    \tpNamespace{#4}%
    \tp@reset@defaults{#2}{#3}%
    \tp@flt@cond
    \expandafter\tpCascadeProps\expandafter{#4}{float}%
    \tp@get@flt@attr{#1}{#2}%
    \tp@set@caption@comps
    \tpUseProperty{content-handler}%
    \tpUseProperty{before-float}%
    \tp@flt@set@hsize
    \tp@set@flt@pos
    \ifx\tp@fps\@empty\else\savenotes\fi
}

\def\endtp@float{%
    \tp@b@float
    \tp@set@top@sep
    \tp@test@caption{0}{capt}{top}%
    \tp@test@caption{0}{capt}{bottom}%
    \tp@flt@create@counters%
    \tp@flt@compose
    \tp@set@bot@sep
    \tp@e@float
    \tp@flt@debug{\tp@captype}%
    \ifx\tp@fps\@empty\else\spewnotes\fi
  \endgroup
  \global\let\tp@current@class\relax
}
%    \end{macrocode}
% Test if the current (sub)float has any top or bottom captions that
% needed printed
%    \begin{macrocode}

\def\tp@test@caption#1#2#3{%
  \setbox\tp@tempboxa\hbox{\tpGobble\tpSubFloatCnt#1\relax\tpUseProperty{caption-#3}\relax}%
  \setbox\tp@tempboxb\hbox{\tpGobble\tpSubFloatCnt\m@ne\relax\tpUseProperty{caption-#3}\relax}%
  \ifdim\wd\tp@tempboxa>\wd\tp@tempboxb\relax
    \expandafter\global\expandafter\let\csname tp@has@#2@#3\endcsname\relax
  \fi}

%    \end{macrocode}
% Calculate the ratio between each subfloat's height and the height of the largest subfloat
%    \begin{macrocode}
\def\tp@flt@calc@sameheight{%
  \@tempdima=\z@\relax
  \@tempcnta=\z@\relax
  \tp@iterate{\@tempcnta}{\@ne}{\tpSubFloatCnt}{%
    \edef\@tempa{\CalcRatio{\tp@subflt@maxheight}{\csname tp@\tp@namespace @height-\the\@tempcnta\endcsname}}%
    \ifnum\@tempcnta>\@ne
      \advance\tp@total@flt@width-\tp@subflt@sep\relax%
    \fi
    \expandafter\@tempdimc\csname tp@\tp@namespace @width-\the\@tempcnta\endcsname\relax
    \@tempdimb=\@tempa\@tempdimc\relax
    \expandafter\edef\csname  tp@\tp@namespace @adj@width-\the\@tempcnta\endcsname{\the\@tempdimb}%
    \advance\@tempdima\@tempdimb
  }%
  \@tempcnta=\z@\relax
  \@tempdimb=\z@\relax
  \@tempdimc=\z@\relax
  \tp@iterate{\@tempcnta}{\@ne}{\tpSubFloatCnt}{%
    \edef\@tempa{\CalcRatio{\csname tp@\tp@namespace @adj@width-\the\@tempcnta\endcsname}{\@tempdima}}%
    \expandafter\edef\csname tp@\tp@namespace @res@width-\the\@tempcnta\endcsname{\dimexpr\@tempa\tp@total@flt@width\relax}%
    \@tempdimc\dimexpr\csname tp@\tp@namespace @height-\the\@tempcnta\endcsname\relax
    \@tempdimc\dimexpr\@tempa\@tempdimc\relax
    \ifdim\@tempa\@tempdimb<\@tempdimc\@tempdimb\@tempdimc\relax\fi
    \tp@test@caption{\the\@tempcnta}{subcapt}{top}%
    \tp@test@caption{\the\@tempcnta}{subcapt}{bottom}%
  }%
  \expandafter\edef\csname tp@\tp@namespace @res@height\endcsname{\the\@tempdimb}%
}

%    \end{macrocode}
% Process the contents of a subfloat
%    \begin{macrocode}
\def\tp@flt@process@sameheight{%
  \tp@flt@calc@sameheight
  \hskip\tp@flt@marg@l
  \vbox{%
    \tpUseProperty{subfloat-block}%
  }%
}

\def\tp@flt@process@sub{%
  \hskip\tp@flt@marg@l
  \tpUseProperty{subfloat-block}%
}

%    \end{macrocode}
% Process the contents of a float environment
%    \begin{macrocode}

\def\tp@flt@process{%
  \bgroup
    \ifnum\tpSubFloatCnt=\z@\relax
      \hskip\tp@flt@marg@l%
      \tpUseProperty{float-block}%
    \else
      \let\tp@is@subflt\relax
      \expandafter\tp@subflt@sep=\tpUseProperty{sub-float-sep}\relax%
      \tpIfProp{subfloat-same-height}
        {\tp@flt@process@sameheight}
        {\tp@flt@process@sub}%
      \let\tp@is@subflt\@undefined
    \fi
  \egroup
}
%    \end{macrocode}
% Composes the full caption.
%    \begin{macrocode}
\def\tp@make@full@caption#1{%
  \hskip\tp@flt@marg@l
  \tp@flt@create@block{\tp@make@caption{0}{#1}}{top}{\tp@total@flt@width}{\z@}%
}
%    \end{macrocode}
% This macro finally composes the entire float object.
%    \begin{macrocode}
\def\tp@flt@compose{%
  \hsize\tp@total@flt@width
  \ifx\tp@has@capt@top\relax
    \expandafter\ifx\csname tp@\tp@captype @attr@sidecap\endcsname\relax\relax
      \tp@make@full@caption{top}%
    \else
      %% TODO: Sidecap handler: Besser nach \tp@flt@process ?
    \fi
  \fi
  %\ifhmode\@@par\nobreak\fi
  \tp@flt@process
  \ifhmode\@@par\nobreak\fi
  \ifx\tp@has@capt@bottom\relax
    \expandafter\ifx\csname tp@\tp@captype @attr@sidecap\endcsname\relax
      \tp@make@full@caption{bottom}%
    \fi
  \fi
  \tp@make@anchors%
  \csname tp@make@listof@\tp@captype\endcsname{\tp@captype}% single float
  \ifhmode\@@par\fi}


%    \end{macrocode}
% shell for sub floats
%    \begin{macrocode}
\def\tpSubFloat{%
  \ifx\tp@is@subflt\relax
    \PackageError{transpect-floats.sty}{Nested tpSubFloats detected!}{You cannot nest a `tpSubFloat' environment into another `tpSubFloat' environment! Use the `tpSubFloatRow' and `tpSubFloatCol' environments to create more complex structures, instead.}%
  \else
    \let\tp@is@subflt\relax
    \global\advance\tpSubFloatCnt\@ne
    \ignorespaces
  \fi}

\def\tp@flt@create@natural{\tpUseComp{Content}}

\def\endtpSubFloat{%
  \ifx\tp@nofigs\relax
    \setbox\tp@subfltbox\hbox{\rule{0pt}{1pt}\rule{1pt}{0pt}}%
  \else
    \setbox\tp@subfltbox\hbox{\tp@flt@create@natural}%\tpUseComp{Content}
  \fi
  \expandafter\xdef\csname tp@\tp@namespace @width-\the\tpSubFloatCnt\endcsname{\the\wd\tp@subfltbox}%
  \expandafter\xdef\csname tp@\tp@namespace @height-\the\tpSubFloatCnt\endcsname{\the\ht\tp@subfltbox}%
  \expandafter\xdef\csname tp@\tp@namespace @depth-\the\tpSubFloatCnt\endcsname{\the\dp\tp@subfltbox}%
  \expandafter\xdef\csname tp@\tp@namespace @col-\the\tpSubFloatCnt\endcsname{\the\tp@subfloat@col}%
  \expandafter\xdef\csname tp@\tp@namespace @row-\the\tpSubFloatCnt\endcsname{\the\tp@subfloat@row}%
  \@tempdima=\dimexpr\the\ht\tp@subfltbox+\the\dp\tp@subfltbox\relax
  \@tempdimb=\dimexpr\the\wd\tp@subfltbox\relax
  \ifx\tp@subfloat@parent r\relax
    \expandafter\expandafter\expandafter\edef\expandafter\csname tp@flt@max@row@width@\the\tp@subfloat@row\expandafter\endcsname\expandafter
      {\dimexpr\csname tp@flt@max@row@width@\the\tp@subfloat@row\endcsname+\the\wd\tp@subfltbox\relax}%
    \expandafter\expandafter\expandafter\ifdim\expandafter\@tempdima>\csname tp@flt@max@row@height@\the\tp@subfloat@row\endcsname\relax
      \expandafter\edef\csname tp@flt@max@row@height@\the\tp@subfloat@row\endcsname{\the\@tempdima}%
    \fi
  \else
    \ifx\tp@subfloat@parent c\relax
      \expandafter\expandafter\expandafter\edef\expandafter\csname tp@flt@max@col@height@\the\tp@subfloat@col\expandafter\endcsname\expandafter
        {\dimexpr\csname tp@flt@max@col@height@\the\tp@subfloat@col\endcsname+\the\ht\tp@subfltbox+\the\dp\tp@subfltbox\relax}%
      \expandafter\expandafter\expandafter\ifdim\expandafter\@tempdimb>\csname tp@flt@max@col@width@\the\tp@subfloat@col\endcsname\relax
        \expandafter\edef\csname tp@flt@max@col@width@\the\tp@subfloat@col\endcsname{\the\@tempdimb}%
      \fi
    \else
      \ifx\tp@subfloat@parent f\relax
        \ifdim\@tempdima>\tp@subflt@maxheight\relax
          \global\tp@subflt@maxheight=\@tempdima\relax
        \fi
      \fi
    \fi
  \fi
  \ignorespaces
  \expandafter\ifx\csname tp@\tp@captype @attr@subfloat\endcsname\relax
    \csname tp@make@listof@\tp@captype\endcsname{\tp@captype}% subfloats are counted separately
  \else
    \csname tp@make@listof@\tp@captype\endcsname{sub\tp@captype}% real subfloats
  \fi
  \setbox\tp@subfltbox\box\voidb@x
  \let\tp@is@subflt\@undefined
}


\def\tpSubFloatRow{%
  \advance\tp@subfloat@row\@ne
  \let\tp@subfloat@prev@parent\tp@subfloat@parent
  \let\tp@subfloat@parent=r\relax
  \expandafter\def\csname tp@flt@max@row@height@\the\tp@subfloat@row\endcsname{\z@}%
  \expandafter\def\csname tp@flt@max@row@width@\the\tp@subfloat@row\endcsname{\z@}%
}
\def\endtpSubFloatRow{%
  \let\tp@subfloat@parent\tp@subfloat@prev@parent
}

\def\tpSubFloatCol{%
  \advance\tp@subfloat@col\@ne
  \let\tp@subfloat@prev@parent\tp@subfloat@parent
  \let\tp@subfloat@parent=c\relax
  \expandafter\let\csname tp@flt@max@col@width@\the\tp@subfloat@col\endcsname=\z@\relax
}
\def\endtpSubFloatCol{%
  \let\tp@subfloat@parent\tp@subfloat@prev@parent
}


%    \end{macrocode}
% \section{Handlers for figures}
%
% The \lstinline{\tpFigureHandler} tells the float module the name,
% main namespace, and main content container of figure floats.
%    \begin{macrocode}
\def\tpFigureHandler{\tpMakeFltComp{tpFig}{figure}{Content}}
%    \end{macrocode}
% The \lstinline{\tpFigureFloat} macro tells the module how figures
% are to be rendered.
%    \begin{macrocode}
\def\tpFigureFloat{%
  \let\includegraphics\tp@includesubgraphics
  \tp@flt@create@block{%
    % \vskip-\tpUseProperty{float-v-offset}%
    \csname tp@\tp@namespace @figure-0\endcsname
    \nobreak}{top}{\the\tp@total@flt@width}{\z@}%
}
%    \end{macrocode}
% The \lstinline{\tpFigureSubFloat} macro tells the module how
% subfloats of figure floats are to be rendered.
%    \begin{macrocode}
\def\tpFigureSubFloat{%
  \ifx\tp@has@subcapt@top\relax
    % \vskip-\tpUseProperty{sub-float-v-offset}\relax
    \tp@flt@calc@row@ht{\tp@make@caption{\the\@tempcnta}{top}}%
    \tp@iterate@mp{\tp@make@caption{\the\@tempcnta}{top}}{\tpUseProperty{sub-caption-valign-top}}{\tp@flt@curr@row@height}%
    \par\nobreak
  \fi%
  \tp@iterate@mp{\csname tp@\tp@namespace @figure-\the\@tempcnta\endcsname}{bottom}{\csname tp@\tp@namespace @res@height\endcsname}%
  \ifx\tp@has@subcapt@bottom\relax
    \par\nobreak
    \tp@flt@calc@row@ht{\tp@make@caption{\the\@tempcnta}{bottom}}%
    \tp@iterate@mp{\tp@make@caption{\the\@tempcnta}{bottom}}{\tpUseProperty{sub-caption-valign-bottom}}{\tp@flt@curr@row@height}%%
  \fi
}
\def\tp@flt@calc@row@ht#1{%
  \@tempcnta\z@
  \@tempdima\z@
  \tp@iterate{\@tempcnta}{\@ne}{\tpSubFloatCnt}{%
    \setbox\z@\vbox{%
      \expandafter\hsize\dimexpr\csname tp@\tp@namespace @res@width-\the\@tempcnta\endcsname\relax
      \leavevmode
      \tpGobble
      \strut#1\strut
      \par}%
    \ifdim\ht\z@>\@tempdima \@tempdima\ht\z@\relax\fi
  }%
  \edef\tp@flt@curr@row@height{\the\@tempdima}%
}
\def\tp@includesubgraphics{\@ifnextchar [\@tp@includesubgraphics{\@tp@includesubgraphics[]}}%]
\def\@tp@includesubgraphics[#1]#2{%
  \ifx\tp@current@class\relax
    \def\@igopts{max width=\hsize,max height=\vsize}%
  \else
    \def\@igopts{width=\hsize}%
  \fi
  \if!#1!\else
    \def\@igopts{#1,width=\hsize}%
  \fi
  \gdef\@tp@fig@path{#2}%
  \expandafter\tpltx@includegraphics\expandafter[\@igopts]{#2}%
}


%    \end{macrocode}
%
%\section{Handlers for tables}
%
%
%    \begin{macrocode}

\def\tp@reserve@tabular{%
  %\@tp@reserve@tab{}% TODO: tabularx(and y) use tabular, so we can't re-define it
  \@tp@reserve@tab{x}%
  \@tp@reserve@tab{y}%
}

\def\@tp@reserve@tab#1{%
  \expandafter\expandafter\expandafter\let\expandafter\csname orig@tabular#1\expandafter\endcsname\csname tabular#1\endcsname
  \expandafter\expandafter\expandafter\let\expandafter\csname orig@endtabular#1\expandafter\endcsname\csname endtabular#1\endcsname
  \expandafter\def\csname tabular#1\endcsname{\global\setbox\tp@floatbox\vbox\bgroup\csname orig@tabular#1\endcsname}%
  \expandafter\def\csname endtabular#1\endcsname{\csname orig@endtabular#1\endcsname\egroup}%
}

\def\tpTableHandler{%
  \tpMakeFltComp{tpContent}{content}{Content}%
  \tp@reserve@tabular
  \ifx\tp@uses@htmltabs\relax
    \ifx\tp@fps\@empty\else\htAddStyle{break-table:false;}\fi
    \global\setbox\htTableBox\box\voidb@x
    \let\htOutputTable\relax
  \fi}

\def\tpTableFloat{%
  \ifx\htTableBox\@undefined
    \tpContent{\unvbox\tp@floatbox}%
  \else
    \ifvoid\htTableBox
      \tpContent{\unvbox\tp@floatbox}%
    \else
      \tpContent{\unvbox\htTableBox}%
    \fi
  \fi
  % \vskip-\tpUseProperty{float-v-offset}%
  \tpUseComp{Content}%
  % \vskip\tpUseProperty{float-v-offset}%
}



%    \end{macrocode}
%
% \section{Handlers for generic floats}
%
%
%    \begin{macrocode}

\def\tpGenericFloat{\tpUseComp{Content}}

\def\tpGenericHandler{\tpMakeFltComp{tpContent}{content}{Content}}


%    \end{macrocode}
%
% \section{Default Settings}
%
% These declarations also provide the default values for the
% properties of the respective environments.
%    \begin{macrocode}
\tpAddToDefault{float}{%
  \tpSetProperty{intext-skip-top}{\z@}%% non-float sep top
  \tpSetProperty{intext-skip-bottom}{\z@}%% non-float sep bottom
  \tpSetProperty{float-skip-top}{\z@}%% float sep top
  \tpSetProperty{float-skip-bottom}{\z@}%% float sep bottom
  \tpSetProperty{float-v-offset}{\topskip}% vertical offset for floats with no depth
  \tpSetProperty{sub-float-v-offset}{\topskip}% vertical offset for sub-floats with no depth
  \tpSetProperty{sub-float-sep}{\tp@subflt@sep}%% space between sub-floats
  \tpSetProperty{margin-left}{\z@}%% left margin
  \tpSetProperty{margin-right}{\z@}%% right margin
  \tpSetProperty{numbering}{auto}%% automatic numbering for missing Number component
  \tpSetProperty{number-sep}{\enskip}% Separator between label and caption
  \tpSetProperty{before-float}{\parindent\z@}%% executed before content is evaluated
  \tpSetProperty{content-handler}{\tpGenericHandler}% Alias for the caption type specific content handler
  \tpSetProperty{float-block}{\tpGenericFloat}% Alias for the caption type specific content printer
  \tpSetProperty{number-format}{\bfseries}% Format of number, additional to caption-format
  \tpSetProperty{source-format}{}% Format of source, additional to caption-format
  \tpSetProperty{legend-format}{}% Format of legend, additional to caption-format
  \tpSetProperty{subfloat-block}{\tpGenericFloat}% Alias for the caption type specific content printer for sub-floats
  \tpSetProperty{subfloat-same-height}{}% if true, the subfloat must/can be adjusted to the same heights
  \tpSetProperty{caption-top}{%
    \tpIfComp{Number}{\tpUseComp{Number}\tpUseProperty{number-sep}}{}%
    \tpUseComp{Caption}%
  }%
  \tpSetProperty{caption-bottom}{%
    \tpIfComp{Legend}{\tpUseComp{Legend}}{}%
    \tpIfComp{Source}{%
      \tpIfComp{Legend}{\par}{}%
      \tpUseComp{Source}}{}}%
  \tpSetProperty{caption-format}{}% style applied to top and bottom captions
  \tpSetProperty{caption-format-top}{}%% style applied to top captions
  \tpSetProperty{caption-format-bottom}{}%% style applied to bottom captions
  \tpSetProperty{caption-sep-top}{\z@}%% vertical space between top caption and content
  \tpSetProperty{caption-sep-bottom}{\z@}%% vertical space between content and bottom caption
  \tpSetProperty{sub-caption-top}{}% in case, sub-float captions diverge from main caption TODO
  \tpSetProperty{sub-caption-bottom}{}% in case, sub-float captions diverge from main caption TODO
  \tpSetProperty{sub-caption-valign-top}{top}%% vertical alignment of top sub-captions
  \tpSetProperty{sub-caption-valign-bottom}{top}%% vertical alignment of bottom sub-captions
  \tpSetProperty{sub-number-sep}{\,}%% when sub-captions, this is placed between the float counter and the sub-float counter
  \tpSetProperty{sub-number-style}{alph}%% counting style of subcaption counters
  \tpSetProperty{sub-number-format}{}%% format of subcaption counters
  \tpSetProperty{sub-number-before}{(}% stuff that is put immediately before the sub counter
  \tpSetProperty{sub-number-after}{)}% stuff that is put immediately after the sub counter
  \tpSetProperty{sub-number-block}{\tpUseProperty{float-number}\tpUseProperty{sub-number-sep}\tpUseProperty{sub-number}}%% Format of sub numbers
  %% List-of entries
  \tpSetProperty{list-of-page-sep}{\dotfill}%
  \tpSetProperty{list-of-number-sep}{\enskip}%
  \tpSetProperty{list-of-number-align}{left}%
  \tpSetProperty{list-of-parfillskip}{-\rightskip}%
  \tpSetProperty{list-of-margin-right}{\@pnumwidth}%
  \tpSetProperty{list-of-margin-left}{auto}%
  \tpSetProperty{list-of-indent}{auto}% list-of-float appearance
  \tpSetProperty{list-of-block}{%
    \tpUseProperty{list-of-caption-format}%
    \tpIfComp{ListofNumber}
      {\tpUseComp{list-of-hang-number}}
      {\leftskip0pt}%
    \tpUseComp{ListofCaption}%
    \tpUseProperty{list-of-page-sep}\tpUseComp{ListofPage}%
  }% list-of-float appearance
  \tpSetProperty{list-of-before-entry}{%
    \tpGobble
    \leftskip\tpUseProperty{list-of-margin-left}\relax%
    \rightskip \tpUseProperty{list-of-margin-right}\relax
    \parfillskip \tpUseProperty{list-of-parfillskip}\relax
    \parindent\z@
    \@afterindenttrue
    \interlinepenalty\@M
    \leavevmode
    \null\nobreak
  }% list-of-float appearance
  \tpSetProperty{list-of-after-entry}{\par}% list-of-float appearance
}

\tpDeclareFloat{tpFigure}{figure}{lof}{%
  \tpSetProperty{subfloat-same-height}{true}% if true, the subfloat must/can be adjusted to the same heights
  \tpSetProperty{content-handler}{\tpFigureHandler}%
  \tpSetProperty{float-block}{\tpFigureFloat}%
  \tpSetProperty{subfloat-block}{\tpFigureSubFloat}%
}

\tpDeclareFloat{tpTable}{table}{lot}{%
  \tpSetProperty{sub-caption-valign-top}{bottom}%
  \tpSetProperty{content-handler}{\tpTableHandler}%
  \tpSetProperty{float-block}{\tpTableFloat}%
  \tpSetProperty{subfloat-block}{\tpTableFloat}%
}
%    \end{macrocode}
% This is an implementation of hyperref's anchor mechanism to make
% labels work. If no explicit label is given, the mechanism generates
% one, unique to each (sub)float.
%    \begin{macrocode}
\def\tp@make@anchors{\@tempcnta\z@\tp@iterate{\@tempcnta}{\z@}{\tpSubFloatCnt}{\tp@make@anchor{\the\@tempcnta}}}
\def\tp@make@anchor#1{%
  \bgroup
    \global\advance\tp@int@flt@cnt\@ne
    \tpSubFloatCnt#1\relax
    \tpIfComp{Label}
      {\expandafter\let\expandafter\@currentlabel\csname tp@\tp@namespace @number-\the\tpSubFloatCnt\endcsname}
      {\edef\@currentlabel{tp-\tp@namespace-number-\the\tp@int@flt@cnt}}%
    \expandafter\H@refstepcounter\expandafter{\tp@captype}%
    \expandafter\hyper@makecurrent\expandafter{\tp@captype}%
    \let\Hy@tempa\Hy@float@caption
    \expandafter\hyper@@anchor\expandafter{\@currentHref}{\relax}%
    \tpIfComp{Label}
      {\expandafter\let\expandafter\@currentlabel\csname tp@\tp@namespace @number-\the\tpSubFloatCnt\endcsname
       \edef\@tempa{\tpUseComp{Label}}%
       \expandafter\tpltx@label\expandafter{\@tempa}}{\relax}%
  \egroup}
%    \end{macrocode}
% LEGACY: Hooks to change the text font of a standard \LaTeX\ tabular
%    \begin{macrocode}
\def\tablefont{\small}
\def\arraystretch{1.3}
\def\@tabular{%
  \leavevmode
  \hbox \bgroup\tablefont $\col@sep\tabcolsep \let\d@llarbegin\begingroup%$
                                    \let\d@llarend\endgroup
  \ifx\ST@tableformat\@undefined\gdef\@tablefont{\tablefont}\fi
  \@tabarray}
\let\@classzold\@classz
\def\@classz{%
   \expandafter\ifx\d@llarbegin\begingroup
     \toks \count@ =
     \expandafter{\expandafter\@tablefont\the\toks\count@}%
   \fi
   \@classzold}
\def\endtabular{%
  \endarray
  $\egroup}%$
\expandafter\let\csname endtabular*\endcsname=\endtabular
%    \begin{macrocode}[gobble=1]
%</floats>
%    \end{macrocode}
