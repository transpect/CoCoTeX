%    \end{macrocode}
% \chapter{transpect-floats.dtx}
% This module provides handlers for floating objects like tables and
% figires common to all Transpect projects
%
%    \begin{macrocode}
%%
%% module for le-tex transpect.cls that extends floating objects.
%%
%% Maintainer: p.schulz@le-tex.de
%%
%% lualatex  -  texlive > 2019
%%
\NeedsTeXFormat{LaTeX2e}[2019/01/01]
\ProvidesPackage{transpect-floats}
    [2020/22/04 0.90 le-tex transpect floats module]
%    \end{macrocode}
% Hard dependencies
%    \begin{macrocode}
\usepackage{tabularx}
\usepackage{multirow}
\usepackage{caption}
%    \end{macrocode}
% Additional column types for \lstinline{tabular} and
% \lstinline{tabularx} environments:
% \begin{description}
% \item[Q] Like tabularx's X but left aligned
% \item[W] Like tabularx's X but right aligned
% \item[L] Like p but left aligned
% \item[P] Like p but left aligned
% \item[C] Like tabularx's X but centered
% \item[Z] Like p but centered
% \item[e] Used to expand whitespaces evenly between columns. Used
%   between the descriptor for the first and the second physical
%   column and does itself not provide a column!
% \end{description}
%
%    \begin{macrocode}
\newcolumntype{Q}{>{\raggedright\arraybackslash\unskip}X}
\newcolumntype{W}{>{\raggedleft\arraybackslash\unskip}X}
\newcolumntype{L}{>{\raggedright\arraybackslash\unskip}p}
\newcolumntype{P}{>{\RaggedRight\arraybackslash\unskip}p}
\newcolumntype{R}{>{\raggedleft\arraybackslash\unskip}p}
\newcolumntype{C}{>{\centering\arraybackslash\unskip}X}
\newcolumntype{Z}{>{\centering\arraybackslash\unskip}p}
\newcolumntype{e}{!{\extracolsep{\fill}}}
%    \end{macrocode}
% some reserved registers
%    \begin{macrocode}
\newbox \tp@subfltbox
\newcount\tp@subflts          \tp@subflts=\z@\relax
\newdimen\tp@subflt@maxheight \tp@subflt@maxheight=\z@\relax
\newdimen\tp@subflt@sep       \tp@subflt@sep=\fboxsep\relax
\newdimen\tp@subflt@hsize     \tp@subflt@hsize=\textwidth\relax
%    \end{macrocode}
% This macro simplifies the definition of caption type specific
% setups. \#1 is the property name; \#2 is the global default value.
%    \begin{macrocode}
\def\tp@provide@setting#1#2{%
  \expandafter\def\csname @#1@internal@\endcsname{#2}%
  \expandafter\def\csname #1\endcsname{\@ifnextchar[{\csname @#1\endcsname}{\csname @#1\endcsname[]}}%]
  \expandafter\def\csname @#1\endcsname[##1]##2{%
    \edef\tp@opt@arg{##1}%
    \ifx\tp@opt@arg\@empty
      \ltx@LocalExpandAfter\gdef\csname @#1@internal@\endcsname{##2}%
    \else
      \ltx@LocalExpandAfter\gdef\csname @#1@internal@##1\endcsname{##2}%
    \fi
  }%
}
%    \end{macrocode}
% Macro to expand the float settings and select between (global)
% default and type specific values.
%    \begin{macrocode}
\def\tp@use@setting#1{%
  \expandafter\ifx\csname @#1@internal@\tp@captype\endcsname\relax
    \csname @#1@internal@\endcsname
  \else
    \csname @#1@internal@\tp@captype\endcsname
  \fi}
%    \end{macrocode}
% built in float settings
%    \begin{macrocode}
\tp@provide@setting{tpCaptionLabelSep}{}
\tp@provide@setting{tpCaptionLabelFont}{}
\tp@provide@setting{tpCaptionSourceSep}{}
\tp@provide@setting{tpCaptionLegendSep}{}
\tp@provide@setting{tpCaptionFont}{}
\tp@provide@setting{tpCaptionSep}{\z@}
\tp@provide@setting{tpAboveFloatSkip}{\@fptop}
\tp@provide@setting{tpBelowFloatSkip}{\@fpbot}
\tp@provide@setting{tpSubFloatSep}{\tp@subflt@sep}
%    \end{macrocode}
% Resetting the subfloats
%    \begin{macrocode}
\def\tp@reset@subflts{%
  \global\tp@subflts=\z@
  \global\tp@subflt@maxheight=\z@\relax
}
%    \end{macrocode}
% User macros. The optional argument corresponds to the placement
% parameter in standard \LaTeX\ environments.
%    \begin{macrocode}
\newenvironment{tpFigure}[1][]{\tp@float[#1]{figure}{lof}}{\endtp@float}
\newenvironment{tpFigure*}[1][]{\tp@dblfloat[#1]{figure}{lof}}{\endtp@dblfloat}
\newenvironment{tpTable}[1][]{\tp@float[#1]{table}{lot}}{\endtp@float}
\newenvironment{tpTable*}[1][]{\tp@dblfloat[#1]{table}{lot}}{\endtp@dblfloat}

%    \end{macrocode}
% Internal macros for generalized floating environments. They provide
% the macros used to compose floats to make sure they are used only
% within the \lstinline{tp@float}-derived environments.
%
% \#1: float pos, e.g. \lstinline{h}, \lstinline{t}, etc.\\
% \#2: caption type, e.g. \lstinline{figure} or \lstinline{table}
%    \begin{macrocode}
\def\tp@float[#1]#2#3{%
  \begingroup
    \def\tp@fps{#1}%
    \tp@reset@subflts
    \tp@reset@caption{0}%
    \def\tp@captype{#2}%
    \def\tp@caplisttype{#3}%
    \def\tpFig##1{\ltx@LocalExpandAfter\gdef\csname tp@subflt@fig@\the\tp@subflts\endcsname{##1}}%
    \def\tpCaption##1{\ltx@LocalExpandAfter\gdef\csname tp@subflt@cpt@\the\tp@subflts\endcsname{##1}}%
    \def\tpLegend##1{\ltx@LocalExpandAfter\gdef\csname tp@subflt@lgd@\the\tp@subflts\endcsname{##1}}%
    \def\tpSource##1{\ltx@LocalExpandAfter\gdef\csname tp@subflt@src@\the\tp@subflts\endcsname{##1}}%
    \def\tpNumber##1{\ltx@LocalExpandAfter\gdef\csname tp@subflt@lbl@\the\tp@subflts\endcsname{##1}}%
    \parindent\z@
    \ifx\tp@fps\@empty
      \expandafter\addvspace\expandafter{\tp@use@setting{tpAboveFloatSkip}}%
    \fi
    \let\htOutputTable\relax
  }
\let\tp@dblfloat\tp@float
%    \end{macrocode}
% 1st iteration: only reading. This step happens when the contents of the float environment are read.
%    \begin{macrocode}
  \def\endtp@float{%
    \ifx\tp@fps\@empty\else
      \ifx\tp@do@dblfloat\relax
        \expandafter\@dblfloat\expandafter{\expandafter\tp@captype\expandafter}\expandafter[\tp@fps]%
      \else
        \expandafter\@float\expandafter{\expandafter\tp@captype\expandafter}\expandafter[\tp@fps]%
      \fi
    \fi
    \csname tp@\tp@captype @float\endcsname
    \ifx\tp@fps\@empty
      \expandafter\vskip\tp@use@setting{tpBelowFloatSkip}\relax%
    \else
      \ifx\tp@do@dblfloat\relax
        \end@dblfloat
        \global\let\tp@do@dblfloat\@undefined
      \else
        \end@float
      \fi
    \fi
  \endgroup
}
\def\endtp@dblfloat{\let\tp@do@dblfloat\relax\endtp@float}%
%    \end{macrocode}
% Sub-routine for tables
%    \begin{macrocode}

\def\tp@table@float{%
  \expandafter\ifx\csname tp@subflt@cpt@0\endcsname\relax
  \else
    \tp@make@caption{0}%
  \fi
  \vskip\tp@use@setting{tpCaptionSep}%
  \box\htTableBox
}

%    \end{macrocode}
% Sub-routine for figures
%    \begin{macrocode}

\def\tp@figure@float{%
    \tp@subflt@hsize=\hsize\relax
    \vbox{%
      \bgroup
        \expandafter\tp@subflt@sep=\tp@use@setting{tpSubFloatSep}%
        \@tempdima=\z@\relax
        \ifnum\tp@subflts=\z@\relax
          \tp@usefloat0%
        \else
%    \end{macrocode}
% 2nd iteration: calculate the ratio between each subfigure's height and the height of the largest subfigure
%    \begin{macrocode}
          \sh@iterate{\@tempcnta}{\@ne}{\tp@subflts}{%
            \edef\@tempa{\CalcRatio{\tp@subflt@maxheight}{\csname tp@subflt@height@\the\@tempcnta\endcsname}}%
            \ifnum\@tempcnta>\@ne
              \expandafter\global\expandafter\advance\expandafter\tp@subflt@hsize\expandafter-\tp@subflt@sep\relax%
            \fi
            \expandafter\@tempdimc\csname tp@subflt@width@\the\@tempcnta\endcsname\relax
            \@tempdimb=\@tempa\@tempdimc\relax
            \expandafter\edef\csname  tp@subflt@adj@width@\the\@tempcnta\endcsname{\the\@tempdimb}%
            \advance\@tempdima\@tempdimb
          }%
          \@tempcnta\z@
%    \end{macrocode}
% 3rd iteration: Calculate width ratio of all adjusted subfigures against total width of all figures plus separators and print:
%    \begin{macrocode}
          \sh@iterate{\@tempcnta}{\@ne}{\tp@subflts}{%
            \edef\@tempa{\CalcRatio{\csname tp@subflt@adj@width@\the\@tempcnta\endcsname}{\@tempdima}}%
            \@tempdimb\@tempa\tp@subflt@hsize\relax
            \ifnum\@tempcnta>\@ne\relax
              \hskip\tp@subflt@sep\relax
            \fi
            \begin{minipage}[t]{\@tempdimb}%
              \captionsetup{margin=\z@}%
              \tp@usefloat{\the\@tempcnta}%
            \end{minipage}%
          }%
        \fi
      \egroup
      \expandafter\ifx\csname tp@subflt@cpt@0\endcsname\relax\else
        \tp@make@caption{0}%
      \fi
    }%
}

%    \end{macrocode}
% prints a sub figure
%    \begin{macrocode}
\def\tp@usefloat#1{{%
    \parindent\z@
    \csname tp@subflt@fig@#1\endcsname\nobreak\par
    \vskip\tp@use@setting{tpCaptionSep}%
    \expandafter\ifx\csname tp@subflt@cpt@#1\endcsname\relax
    \else
      \tp@make@caption{#1}%
    \fi
}}
%    \end{macrocode}
% prints the caption
%    \begin{macrocode}
\def\tp@make@caption#1{%
  {\edef\tp@cap{%
     \expandafter\ifx\csname tp@subflt@lbl@#1\endcsname\relax\else
       \bgroup
         \expandonce{\tp@use@setting{tpCaptionLabelFont}}%
         \csname tp@subflt@lbl@#1\endcsname%
       \egroup
       \tp@use@setting{tpCaptionLabelSep}%
     \fi
     \bgroup
       \expandonce{\tp@use@setting{tpCaptionFont}}%
       \expandonce{\csname tp@subflt@cpt@#1\endcsname}%
     \egroup
     \expandafter\ifx\csname tp@subflt@lgd@#1\endcsname\relax\else
       \tp@use@setting{tpCaptionLegendSep}%
       \csname tp@subflt@lgd@#1\endcsname%
     \fi
     \tp@use@setting{tpCaptionSourceSep}%
     \csname tp@subflt@src@#1\endcsname
   }%
   \expandafter\captionof\expandafter{\expandafter\tp@captype\expandafter}\expandafter*\expandafter{\tp@cap}%
   \addcontentsline{\tp@caplisttype}{\tp@captype}{\string\numberline{\csname tp@subflt@lbl@#1\endcsname}\csname tp@subflt@cpt@#1\endcsname\csname tp@subflt@src@#1\endcsname}%
   \tp@reset@caption{#1}%
 }}
%    \end{macrocode}
% Resetting caption parameters
%    \begin{macrocode}
\def\tp@reset@caption#1{
  \ltx@LocalExpandAfter\global\expandafter\let\csname tp@subflt@fig@#1\endcsname\relax%
  \ltx@LocalExpandAfter\global\expandafter\let\csname tp@subflt@cpt@#1\endcsname\relax%
  \ltx@LocalExpandAfter\global\expandafter\let\csname tp@subflt@src@#1\endcsname\relax%
  \ltx@LocalExpandAfter\global\expandafter\let\csname tp@subflt@lbl@#1\endcsname\relax%
  }
%    \end{macrocode}
% shell for sub floats
%    \begin{macrocode}
\def\tpSubFloat{%
  \global\advance\tp@subflts\@ne
}
\def\endtpSubFloat{%
  \setbox\tp@subfltbox\hbox{\csname tp@subflt@fig@\the\tp@subflts\endcsname}%
  \expandafter\xdef\csname tp@subflt@width@\the\tp@subflts\endcsname{\the\wd\tp@subfltbox}%
  \expandafter\xdef\csname tp@subflt@height@\the\tp@subflts\endcsname{\the\ht\tp@subfltbox}%
  \expandafter\ifdim\csname tp@subflt@height@\the\tp@subflts\endcsname>\tp@subflt@maxheight\relax
    \expandafter\global\expandafter\tp@subflt@maxheight=\csname tp@subflt@height@\the\tp@subflts\endcsname\relax
  \fi}


%    \end{macrocode}
% Hooks to change the text font of a standard \LaTeX\ tabular
%    \begin{macrocode}
\def\tablefont{\small}
\def\arraystretch{1.3}
\def\@tabular{%
  \leavevmode
  \hbox \bgroup\tablefont $\col@sep\tabcolsep \let\d@llarbegin\begingroup%$
                                    \let\d@llarend\endgroup
  \ifx\ST@tableformat\@undefined\gdef\@tablefont{\tablefont}\fi
  \@tabarray}
\let\@classzold\@classz
\def\@classz{%
   \expandafter\ifx\d@llarbegin\begingroup
     \toks \count@ =
     \expandafter{\expandafter\@tablefont\the\toks\count@}%
   \fi
   \@classzold}
\def\endtabular{%
  \endarray
  $\egroup}%$
\expandafter\let\csname endtabular*\endcsname=\endtabular



%%%%%%%%%%
% LEGACY %
%%%%%%%%%%



% \renewcommand \thefigure
%      {\@arabic\c@figure}
% %% TODO: Projektspezifisch!
% \@removefromreset{figure}{chapter}

% \renewcommand \thetable
%      {\@arabic\c@table}
% %% TODO: Projektspezifisch!
% \@removefromreset{table}{chapter}

% \captionsetup{%
%    format=plain
%   ,labelformat=empty
%   ,font+=it
%   ,singlelinecheck=false
%   ,justification=RaggedRight
%   ,listformat=empty
% }

% %    \end{macrocode}
% % \section{Sources}
% % Many Transpect projects use different markup for the caption of a floating object and its source.
% %    \begin{macrocode}
% \newlength\aboveSourceSkip \aboveSourceSkip0mm

% \newcommand\transpectBild[3][]{\tr@nspectFloat{#1}{#2}{#3}}
% \newcommand\transpectTab[3][]{\aboveSourceSkip1mm\let\use@depth\relax\tr@nspectFloat{#1}{#2}{#3}}
% \let\transcriptBild\transpectBild
% \let\transcriptTab\transpectTab

% \newbox\tr@nspectFlt
% \newdimen\tr@nspectFltWd
% \newdimen\tr@nspectWd
% \newdimen\tr@nspectHt
% \newdimen\tr@nspectSep


% %    \end{macrocode}
% % \section{Scaling graphics to a common height}
% % The \lstinline{\sameheight} macro is used to scale all figures
% % incorporated via \lstinline{\includesubgraphics} to a common height
% % such that the line of graphics fills \lstinline{\shhsize} (\lstinline{\hsize} by default).
% %    \begin{macrocode}

% \newbox\@includesubgraphicsbox
% \newcount\c@includesubgraphics \c@includesubgraphics\z@
% \newdimen\includesubgraphics@maxheight
% \newdimen\subgraphicssep \subgraphicssep\fboxsep
% \newdimen\sh@margins
% \RequirePackage{ltxcmds}

% \newcommand*\includesubgraphics[2][]{%
%   \global\advance\c@includesubgraphics\@ne
%   \ltx@LocalExpandAfter\gdef\csname subgraphics@caption@\the\c@includesubgraphics\endcsname{#1}%
%   \ltx@LocalExpandAfter\gdef\csname subgraphics@name@\the\c@includesubgraphics\endcsname{#2}%
%   \setbox\@includesubgraphicsbox\hbox{\includegraphics{#2}}%
%   \ltx@LocalExpandAfter\xdef\csname subgraphics@width@\the\c@includesubgraphics\endcsname{\the\wd\@includesubgraphicsbox}%
%   \ltx@LocalExpandAfter\xdef\csname subgraphics@height@\the\c@includesubgraphics\endcsname{\the\ht\@includesubgraphicsbox}%
%   \expandafter\ifdim\csname subgraphics@height@\the\c@includesubgraphics\endcsname>\includesubgraphics@maxheight\relax
%     \ltx@LocalExpandAfter\global\expandafter\includesubgraphics@maxheight=\csname subgraphics@height@\the\c@includesubgraphics\endcsname\relax
%   \fi}
% %    \end{macrocode}
% %
% %    \begin{macrocode}
% % \def\sameheight#1{%
% %   \bgroup
% %     \tp@reset@subflts
% % %    \end{macrocode}
% % %% measurement
% % %    \begin{macrocode}
% %     \setbox\@tempboxa\hbox{#1}%
% % %    \end{macrocode}
% % %% 1st iteration: Calculate widths of all subfigures when scaled up to the highest subfigure's height:
% % %% \lstinline{\@tempdima}: total sum of those widths
% % %    \begin{macrocode}
% %     \@tempdima=\z@\relax
% %     \sh@iterate{\@tempcnta}{\@ne}{\tp@subflts}{%
% %       \edef\@tempa{\CalcRatio{\tp@subflt@maxheight}{\csname tp@subflt@height@\the\@tempcnta\endcsname}}%
% %       \ifnum\@tempcnta>\@ne\global\advance\@shhsize-\tp@subflt@sep\relax\fi
% %       \expandafter\@tempdimc\csname tp@subflt@width@\the\@tempcnta\endcsname\relax
% %       \@tempdimb=\@tempa\@tempdimc\relax
% %       \expandafter\edef\csname  tp@subflt@adj@width@\the\@tempcnta\endcsname{\the\@tempdimb}%
% %       \global\advance\@tempdima\@tempdimb
% %   }%
% % %    \end{macrocode}
% % %% 2nd iteration: Calculate width ratio of all adjusted subfigures against total width of all figures plus separators and output:
% % %    \begin{macrocode}
% %   \@tempcnta\z@
% %   \sh@iterate{\@tempcnta}{\@ne}{\tp@subflts}{%
% %     \edef\@tempa{\CalcRatio{\csname tp@subflt@adj@width@\the\@tempcnta\endcsname}{\@tempdima}}%
% %     \@tempdimb\@tempa\@shhsize\relax
% %     %\edef\@tempb{\csname subgraphics@name@\the\@tempcnta\endcsname}%
% %     \ifnum\@tempcnta>\@ne\hskip\subgraphicssep\fi
% %     \def\@igopts{width=\linewidth}%
% %     \begin{minipage}[b]{\@tempdimb}%
% %       \captionsetup{margin=\z@}%
% %       %\csname subgraphics@caption@\the\@tempcnta\endcsname%
% %       \csname tp@subflt@fig@\the\@tempcnta\endcsname
% %       %\expandafter\expandafter\expandafter\includegraphics\expandafter\expandafter\expandafter[\expandafter\@igopts\expandafter]\expandafter{\@tempb}%
% %       % \ifx\c@psource\@undefined\else
% %       %   \vskip\dimexpr-\topskip+\aboveSourceSkip+.5mm\relax
% %       %   \vtop{\captionsetup{font=footnotesize}\caption*{\c@psource}}%
% %       %   \global\let\c@psource\@undefined
% %       % \fi
% %     \end{minipage}%
% %   }%
% %   \egroup
% % }
% %    \end{macrocode}
% % END: sameheight
% %
% % \#1: layout variant, \#2 subufigures
% %    \begin{macrocode}
% \newcommand\transriptFixedFigure[3][]{%
%   \bgroup
%   \shhsize\hsize
%   \subgraphicssep2mm
%   \ifx#2A\relax
%     \@tempdima\z@
%   \else
%     \ifx#2B\relax
%       \@tempdima20mm
%     \else
%       \ifx#2C\relax
%         \@tempdima30mm
%       \else
%         \ifx#2D\relax
%           \@tempdima50mm
%         \fi
%       \fi
%     \fi
%   \fi
%   \advance\shhsize-\@tempdima
%   \sh@margins.5\@tempdima
%   \vskip1\baselineskip
%   \captionsetup{margin=\sh@margins\relax}%
%   \if!#1!\else#1\fi
%   \hskip\sh@margins\sameheight{#3}%
%   \ifx\c@psource\@undefined\else
%     \vskip\aboveSourceSkip
%     \captionsetup{font=footnotesize,skip=-1mm}%
%     \caption*{\c@psource}%
%     \global\let\c@psource\@undefined
%   \fi
%   \egroup}

% \def\capsource#1{\def\c@psource{#1}}%

% \let\oldfigure\figure
% \let\oldendfigure\endfigure

% \renewenvironment{figure}{\savenotes\oldfigure}{\oldendfigure\spewnotes}


% \long\def\tr@nspectFloat#1#2#3{%
%   \def\@rgi{#1}%
%   \setbox\tr@nspectFlt\hbox{#3}%
%   \tr@nspectFltWd=\wd\tr@nspectFlt\relax
%   \tr@nspectHt=\dimexpr\ht\tr@nspectFlt\ifx\use@depth\relax+\dp\tr@nspectFlt\fi\relax
%   \tr@nspectSep\dimexpr(\textwidth-\tr@nspectFltWd)/2\relax
%   \vskip\baselineskip
%   \ifdim\tr@nspectFltWd<.5\textwidth\relax
%     \bgroup
%       %\tr@nspectSep4mm
%       \tr@nspectWd\dimexpr\textwidth-\tr@nspectFltWd\relax
%       \noindent\begin{minipage}[\ifx\use@depth\relax t\else b\fi][\tr@nspectHt][t]{\tr@nspectWd}%
%         \captionsetup{justification=RaggedRight}%
%         \captionof{figure}{#2}%
%         \ifx\@rgi\@empty\else
%           \vfill
%         \captionsetup{font=footnotesize,skip=\z@}%
%         \caption*{#1}%
%         \vspace*{\dimexpr-\dp\tr@nspectFlt-1mm}%
%        \fi
%       \end{minipage}\hfill
%       \begin{minipage}[t][\tr@nspectHt]{\tr@nspectFltWd}\centering
%         \unhbox\tr@nspectFlt%
%       \end{minipage}%
%     \egroup
%   \else
%     \noindent\hskip\tr@nspectSep
%     \begin{minipage}{\tr@nspectFltWd}\centering%
%       \captionof{figure}{#2}%
%       \unhbox\tr@nspectFlt\par
%       \ifx\@rgi\@empty\else
%         \captionsetup{font=footnotesize,skip=-1mm}\caption*{#1}%
%         \vspace*{\dimexpr-\dp\tr@nspectFlt-1mm}%
%       \fi
%     \end{minipage}%
%   \fi
%   \vskip\baselineskip
%   \global\let\use@depth\@undefined
%   \tr@nspectWd\z@% \tr@nspectSep\z@
% }
