%    \end{macrocode}
% \chapter{transpect-floats.dtx}
% This module provides handlers for floating objects like tables and
% figires common to all Transpect projects
%
%    \begin{macrocode}
%%
%% module for le-tex transpect.cls that extends floating objects.
%%
%% Maintainer: p.schulz@le-tex.de
%%
%% lualatex  -  texlive > 2019
%%
\NeedsTeXFormat{LaTeX2e}[2019/01/01]
\ProvidesPackage{transpect-floats}
    [2020/22/04 0.90 le-tex transpect floats module]
%    \end{macrocode}
% Hard dependencies
%    \begin{macrocode}
\usepackage{tabularx}
\usepackage{multirow}
\usepackage{caption}
%    \end{macrocode}
% Additional column types:
% \begin{description}
% \item[Q] Like tabularx's X but left aligned
% \item[W] Like tabularx's X but right aligned
% \item[L] Like p but left aligned
% \item[P] Like p but left aligned
% \item[C] Like tabularx's X but centered
% \item[Z] Like p but centered
% \item[e] Used to expand whitespaces evenly between columns. Used
%   between the descriptor for the first and the second physical
%   column and does itself not provide a column!
% \end{description}
%
%    \begin{macrocode}
\newcolumntype{Q}{>{\raggedright\arraybackslash\unskip}X}
\newcolumntype{W}{>{\raggedleft\arraybackslash\unskip}X}
\newcolumntype{L}{>{\raggedright\arraybackslash\unskip}p}
\newcolumntype{P}{>{\RaggedRight\arraybackslash\unskip}p}
\newcolumntype{R}{>{\raggedleft\arraybackslash\unskip}p}
\newcolumntype{C}{>{\centering\arraybackslash\unskip}X}
\newcolumntype{Z}{>{\centering\arraybackslash\unskip}p}
\newcolumntype{e}{!{\extracolsep{\fill}}}

\renewcommand \thefigure
     {\@arabic\c@figure}
%% TODO: Projektspezifisch!
\@removefromreset{figure}{chapter}

\renewcommand \thetable
     {\@arabic\c@table}
%% TODO: Projektspezifisch!
\@removefromreset{table}{chapter}

\captionsetup{%
   format=plain
  ,labelformat=empty
  ,font+=it
  ,singlelinecheck=false
  ,justification=RaggedRight
  ,listformat=empty
}

%    \end{macrocode}
% \section{Sources}
% Many Transpect projects use different markup for the caption of a floating object and its source.
%    \begin{macrocode}
\newlength\aboveSourceSkip \aboveSourceSkip0mm

\newcommand\transcriptBild[3][]{\tr@nscriptFloat{#1}{#2}{#3}}
\newcommand\transcriptTab[3][]{\aboveSourceSkip1mm\let\use@depth\relax\tr@nscriptFloat{#1}{#2}{#3}}

\newbox\tr@nscriptFlt
\newdimen\tr@nscriptFltWd
\newdimen\tr@nscriptWd
\newdimen\tr@nscriptHt
\newdimen\tr@nscriptSep


%    \end{macrocode}
% \section{Scaling graphics to a common height}
% The \lstinline{\sameheight} macro is used to scale all figures
% incorporated via \lstinline{\includesubgraphics} to a common height
% such that the line of graphics fills \lstinline{\shhsize} (\lstinline{\hsize} by default).
%    \begin{macrocode}

\newbox\@includesubgraphicsbox
\newcount\c@includesubgraphics \c@includesubgraphics\z@
\newdimen\includesubgraphics@maxheight
\newdimen\subgraphicssep \subgraphicssep\fboxsep
\newdimen\shhsize \shhsize=\hsize\relax
\newdimen\sh@margins
\RequirePackage{ltxcmds}

\newcommand*\includesubgraphics[2][]{%
  \global\advance\c@includesubgraphics\@ne
  \ltx@LocalExpandAfter\gdef\csname subgraphics@caption@\the\c@includesubgraphics\endcsname{#1}%
  \ltx@LocalExpandAfter\gdef\csname subgraphics@name@\the\c@includesubgraphics\endcsname{#2}%
  \setbox\@includesubgraphicsbox\hbox{\includegraphics{#2}}%
  \ltx@LocalExpandAfter\xdef\csname subgraphics@width@\the\c@includesubgraphics\endcsname{\the\wd\@includesubgraphicsbox}%
  \ltx@LocalExpandAfter\xdef\csname subgraphics@height@\the\c@includesubgraphics\endcsname{\the\ht\@includesubgraphicsbox}%
  \expandafter\ifdim\csname subgraphics@height@\the\c@includesubgraphics\endcsname>\includesubgraphics@maxheight\relax
    \ltx@LocalExpandAfter\global\expandafter\includesubgraphics@maxheight=\csname subgraphics@height@\the\c@includesubgraphics\endcsname\relax
  \fi}
%    \end{macrocode}
%
%    \begin{macrocode}
\def\sameheight#1{%
  \bgroup
  \let\c@psource\@undefined
  \c@includesubgraphics=\z@\relax
  \let\@shhsize\shhsize
  \includesubgraphics@maxheight=\z@\relax
%    \end{macrocode}
%% measurement
%    \begin{macrocode}
  \setbox\@tempboxa\hbox{#1}%
%    \end{macrocode}
%% 1st iteration: Calculate widths of all subfigures when scaled up to the highest subfigure's height:
%% \lstinline{\@tempdima}: total sum of those widths
%    \begin{macrocode}
  \@tempdima=\z@\relax
  \sh@iterate{\@tempcnta}{\@ne}{\c@includesubgraphics}{%
    \edef\@tempa{\CalcRatio{\includesubgraphics@maxheight}{\csname subgraphics@height@\the\@tempcnta\endcsname}}%
    \ifnum\@tempcnta>\@ne\global\advance\@shhsize-\subgraphicssep\relax\fi
    \expandafter\@tempdimc\csname subgraphics@width@\the\@tempcnta\endcsname\relax
    \@tempdimb=\@tempa\@tempdimc\relax
    \expandafter\xdef\csname  subgraphics@adjusted@width@\the\@tempcnta\endcsname{\the\@tempdimb}%
    \global\advance\@tempdima\@tempdimb
  }%
%    \end{macrocode}
%% 2nd iteration: Calculate width ratio of all adjusted subfigures against total width of all figures plus separators and output:
%    \begin{macrocode}
  \@tempcnta\z@
  \sh@iterate{\@tempcnta}{\@ne}{\c@includesubgraphics}{%
    \edef\@tempa{\CalcRatio{\csname subgraphics@adjusted@width@\the\@tempcnta\endcsname}{\@tempdima}}%
    \@tempdimb\@tempa\@shhsize\relax
    \edef\@tempb{\csname subgraphics@name@\the\@tempcnta\endcsname}%
    \ifnum\@tempcnta>\@ne\hskip\subgraphicssep\fi
    \def\@igopts{width=\linewidth}%
    \begin{minipage}[b]{\@tempdimb}%
      \captionsetup{margin=\z@}%
      \csname subgraphics@caption@\the\@tempcnta\endcsname%
      \expandafter\expandafter\expandafter\includegraphics\expandafter\expandafter\expandafter[\expandafter\@igopts\expandafter]\expandafter{\@tempb}%
      \ifx\c@psource\@undefined\else
        \vskip\dimexpr-\topskip+\aboveSourceSkip+.5mm\relax
        \vtop{\captionsetup{font=footnotesize}\caption*{\c@psource}}%
        \global\let\c@psource\@undefined
      \fi
    \end{minipage}%
  }%
  \egroup
}
%    \end{macrocode}
% END: sameheight
%
% \#1: layout variant, \#2 subufigures
%    \begin{macrocode}
\newcommand\transriptFixedFigure[3][]{%
  \bgroup
  \shhsize\hsize
  \subgraphicssep2mm
  \ifx#2A\relax
    \@tempdima\z@
  \else
    \ifx#2B\relax
      \@tempdima20mm
    \else
      \ifx#2C\relax
        \@tempdima30mm
      \else
        \ifx#2D\relax
          \@tempdima50mm
        \fi
      \fi
    \fi
  \fi
  \advance\shhsize-\@tempdima
  \sh@margins.5\@tempdima
  \vskip1\baselineskip
  \captionsetup{margin=\sh@margins\relax}%
  \if!#1!\else#1\fi
  \hskip\sh@margins\sameheight{#3}%
  \ifx\c@psource\@undefined\else
    \vskip\aboveSourceSkip
    \captionsetup{font=footnotesize,skip=-1mm}%
    \caption*{\c@psource}%
    \global\let\c@psource\@undefined
  \fi
  \egroup}

\def\capsource#1{\def\c@psource{#1}}%

\let\oldfigure\figure
\let\oldendfigure\endfigure

\renewenvironment{figure}{\savenotes\oldfigure}{\oldendfigure\spewnotes}


\long\def\tr@nscriptFloat#1#2#3{%
  \def\@rgi{#1}%
  \setbox\tr@nscriptFlt\hbox{#3}%
  \tr@nscriptFltWd=\wd\tr@nscriptFlt\relax
  \tr@nscriptHt=\dimexpr\ht\tr@nscriptFlt\ifx\use@depth\relax+\dp\tr@nscriptFlt\fi\relax
  \tr@nscriptSep\dimexpr(\textwidth-\tr@nscriptFltWd)/2\relax
  \vskip\baselineskip
  \ifdim\tr@nscriptFltWd<.5\textwidth\relax
    \bgroup
      %\tr@nscriptSep4mm
      \tr@nscriptWd\dimexpr\textwidth-\tr@nscriptFltWd\relax
      \noindent\begin{minipage}[\ifx\use@depth\relax t\else b\fi][\tr@nscriptHt][t]{\tr@nscriptWd}%
        \captionsetup{justification=RaggedRight}%
        \captionof{figure}{#2}%
        \ifx\@rgi\@empty\else
          \vfill
        \captionsetup{font=footnotesize,skip=\z@}%
        \caption*{#1}%
        \vspace*{\dimexpr-\dp\tr@nscriptFlt-1mm}%
       \fi
      \end{minipage}\hfill
      \begin{minipage}[t][\tr@nscriptHt]{\tr@nscriptFltWd}\centering
        \unhbox\tr@nscriptFlt%
      \end{minipage}%
    \egroup
  \else
    \noindent\hskip\tr@nscriptSep
    \begin{minipage}{\tr@nscriptFltWd}\centering%
      \captionof{figure}{#2}%
      \unhbox\tr@nscriptFlt\par
      \ifx\@rgi\@empty\else
        \captionsetup{font=footnotesize,skip=-1mm}\caption*{#1}%
        \vspace*{\dimexpr-\dp\tr@nscriptFlt-1mm}%
      \fi
    \end{minipage}%
  \fi
  \vskip\baselineskip
  \global\let\use@depth\@undefined
  \tr@nscriptWd\z@% \tr@nscriptSep\z@
}

\def\tablefont{\nsffamily\small}
\def\arraystretch{1.3}
\def\@tabular{%
  \leavevmode
  \hbox \bgroup\tablefont $\col@sep\tabcolsep \let\d@llarbegin\begingroup%$
                                    \let\d@llarend\endgroup
  \ifx\ST@tableformat\@undefined\gdef\@tablefont{\tablefont}\fi
  \@tabarray}
\let\@classzold\@classz
\def\@classz{%
   \expandafter\ifx\d@llarbegin\begingroup
     \toks \count@ =
     \expandafter{\expandafter\@tablefont\the\toks\count@}%
   \fi
   \@classzold}
\def\endtabular{%
  \endarray
  $\egroup}%$
\expandafter\let\csname endtabular*\endcsname=\endtabular
