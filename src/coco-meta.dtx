% \chapter{coco-meta.dtx}
%    \begin{macrocode}[numbers=none,gobble=1]
%<*meta>
%    \end{macrocode}
%
% This file provides some macros that are used to process meta data,
% both for the whole document, as well as parts of a document.
%
% File preamble
%    \begin{macrocode}
%%
%% module for CoCoTeX that provides handling of a document's meta data.
%%
%% Maintainer: p.schulz@le-tex.de
%%
%% lualatex  -  texlive > 2019
%%
\NeedsTeXFormat{LaTeX2e}[2018/12/01]
\ProvidesPackage{coco-meta}
    [\filedate \fileversion CoCoTeX meta module]
\RequirePackage{coco-common}
%    \end{macrocode}
% \DescribeContainer{CommonMeta} is an abstract Container for
%   commonly used meta data, both for whole documents as well as parts
%   of documents.
%    \begin{macrocode}
\ccDeclareContainer{CommonMeta}{%
  \ccDeclareType{Components}{%
    \ccDeclareRole[author]{Author}%
    \ccm@declare@comp
    \ccm@extended@common@macros
    \ccm@declare@affils
  }%
  \ccDeclareType{Properties}{}%
}
%    \end{macrocode}
%
% \section{Counted Container Handlers}
%
% \subsection{Generic Blocks}
%
% \begin{macro}{\ccm@generic@comp} is used to define a generic meta
%   data block. It provides two Components for each instance, one for
%   the block's Heading or Label, and one for its Content.
%    \begin{macrocode}
\def\ccm@generic@comp{%
%    \end{macrocode}
% \begin{Component}{CC} is the formatted list of all GenericMeta
%   components.
%    \begin{macrocode}
  \ccDeclareComponent{GenericMetaBlock}{\expandafter\global}{}%
%    \end{macrocode}
% \end{Component}
%    \begin{macrocode}
  \ccDeclareComponentGroup{GenericMeta}{%
%    \end{macrocode}
% \begin{Component}{Heading}is the label of a generic meta datum
%    \begin{macrocode}
    \ccDeclareCountedComponent{Heading}%
%    \end{macrocode}
% \end{Component}
% \begin{Component}{Heading}is the content of a generic meta datum
%    \begin{macrocode}
    \ccDeclareCountedComponent{Content}%
%    \end{macrocode}
% \end{Component}
%    \begin{macrocode}
  }}
%    \end{macrocode}
% \end{macro}
% \begin{macro}{\ccm@generic@eval} evaluates the Components and
%   tells the Framework how the generic counted Sub-Containers should
%   be rendered.
%    \begin{macrocode}
\def\ccm@generic@eval{{%
  \def\cc@cur@cont{titlepage}%
  \ccComposeCollection{GenericMeta}{generic-meta-format}{GenericMetaBlock}
}}
%    \end{macrocode}
% \end{macro}
%
%
% \subsection{Contributor Roles}
%
% Contributors are Counted Containers that represent the meta data of
% people that share a role in contributing content to a document.
% Examples for such roles are an article{\slash}chapter{\slash}book's
% authors, or a collection{\slash}series' editors.
%
%
% \subsection{Declaring Contributor Role Blocks}
%
% \begin{macro}{\ccDeclareRoleBlock} is used to create a new
%   Collection Container (named \lstinline{\ccPrefix#2#3}) for a Role
%   with the name \InlineArg{2}. A Role Block\index{Role Block} is a
%   Component of the Parent Container, which contains a formatted list
%   of certain Components of all members of the Role. Format and
%   selection of the utilised Components are specified via the
%   Property given in \InlineArg{4}. Role Blocks can also be directly
%   used inside the Parent Container as Overrides.
%
%   The optional argument \InlineArg*{1} tells the evaluator in the
%   Parent Container's \lstinline{end} macro how the collector is to
%   be composed. Valid values are \lstinline{compose}
%   (\textit{default}), which uses \UsageMacro{\ccComposeCollection}
%   to compose the Collection Component, or \lstinline{apply}, which
%   uses \UsageMacro{\ccApplyCollection}, instead.
%    \begin{macrocode}
\def\ccDeclareRoleBlock{\@ifnextchar[\cc@declare@role@block{\cc@declare@role@block[compose]}}%]
\def\cc@declare@role@block[#1]#2#3#4{%
  \ifcsdef{ccm@role@#1}
    {\ccDeclareComponent{#2#3}{\expandafter\global}{}%
     \csgdef{ccm@role@\cc@cur@cont @#2@#3}{#4}%
     \csappto{@ccm@role@eval@\cc@cur@cont @#2}
        {\csname ccm@role@#1\endcsname{#2}{#3}}}
    {\ccPackageError{Meta}{Argument}
      {Invalid optional argument in \string\ccDeclareRoleBlock!}
      {Only `apply' or `compose' are allowed as values^^Jin the optional argument of \string\ccDeclareRoleBlock!}}}%
%    \end{macrocode}
% \end{macro}
% \begin{macro}{\ccm@role@eval} creates the name lists for the
%   role. \InlineArg{1} is the name of the role.
%    \begin{macrocode}
\def\ccm@role@eval#1{\csname @ccm@role@eval@\cc@cur@cont @#1\endcsname}
%    \end{macrocode}
% \end{macro}
% \begin{macro}{\ccm@eval@role}
%   \begin{ArgList}
%   \item[1] is the name of the macro used to compose the Collection (either \lstinline{\ccComposeCollection}, or \lstinline{\ccApplyCollection})
%   \item[2] is the name of the role
%   \item[3] is the name of the list.
% \end{ArgList}
% The access Component is \lstinline{\ccPrefix #2#3}, i.e., the prefix
% and both argumets together.
%    \begin{macrocode}
\def\ccm@eval@role#1#2#3{%
%    \end{macrocode}
% First, we check if the Collection Component has already been set in
% the input. If so, we set an internal flag to indicate that the
% Collection Component has been filled manually.
%    \begin{macrocode}
  \ccIfComp{#2#3}{\cslet{cc@used@#2#3@override}\@empty}{%
%    \end{macrocode}
% Second, we check if the counter for the Role is defined and greater
% than 0. If neither is the case, this means that the Group does not
% occur in the input, at all, so we don't need to do anything.
%    \begin{macrocode}
    \ifcsdef{cc#2Cnt}
      {\expandafter\ifnum\csname cc#2Cnt\endcsname>\z@
%    \end{macrocode}
% otherwise, we call the Property that is stored in
% \lstinline{\ccm@role@\cc@cur@cont @#2@#3} and store the result in the
% Component \lstinline{#2#3}.
%    \begin{macrocode}
          #1{#2}{\csname ccm@role@\cc@cur@cont @#2@#3\endcsname}{#2#3}%
        \fi
      }{}}}
%    \end{macrocode}
% \end{macro}
% \begin{macro}{\ccm@role@apply} \#1 is the name of the role and
%   \#2 is the name of the composition. This macro applies
%   (i.e. \textit{fully expands}) the \lstinline{\ccm@role@\cc@cur@cont @#1@#2}
%   Property and stores the result in the \lstinline{#1#2} Component.
%    \begin{macrocode}
\def\ccm@role@apply#1#2{\ccm@eval@role\ccApplyCollection{#1}{#2}}
%    \end{macrocode}
% \end{macro}
% \begin{macro}{\ccm@role@compose} \#1 is the name of the role and
%   \#2 is the name of the composition. This stores the
%   \textit{unexpaded} contents of the \lstinline{\ccm@role@\cc@cur@cont @#1@#2}
%   Property in the \lstinline{#1#2} Component.
%    \begin{macrocode}
\def\ccm@role@compose#1#2{\ccm@eval@role\ccComposeCollection{#1}{#2}}
%    \end{macrocode}
% \end{macro}
%
% \subsubsection{Declaring Contributor Roles}
%
% \begin{macro}{\ccDeclareRole} is used to declare the Components that
%   belong to each member of a contributor role.
%   \begin{ArgList}
%   \item*[1] is the internal name of the Role's formatting
%     Property. If omitted, it is the same as \InlineArg{2}
%   \item[2] is the name of the role
%   \end{ArgList}
%   The output of all members of a role is controlled by a Component
%   called “\lstinline{<role>NameList}” that is formatted according to
%   the \lstinline{<role>-format} Property. For reasons of naming
%   conventions, the role names for a Component and its respective
%   Property do not necessarily need to be identical.
%    \begin{macrocode}
\def\ccDeclareRole{\cc@opt@second\cc@declare@role}%
\def\cc@declare@role[#1]#2{%
  \ccDeclareComponentGroup[%
    {\ccDeclareAttributeHandler*{corresp}{\ccSetProperty{is-corresp}{true}}}%
  ]{#2}{%
%    \end{macrocode}
%
%
% \paragraph{Contributor Role Components}
%
% \begin{Component}[OR]{FullName} is the full name of the Role
%   member. If omitted, it is calculated by the
%   \UsageProp{role-full-name-format} Property.
%    \begin{macrocode}
    \ccDeclareCountedComponent{FullName}%
%    \end{macrocode}
% \end{Component}
% \begin{Component}[OR]{CiteName} is the Full Role member name that is
%   used for citation advices.  If omitted, it is calculated
%   by the \UsageProp{role-cite-name-format} Property.
%    \begin{macrocode}
    \ccDeclareCountedComponent{CiteName}%
%    \end{macrocode}
% \end{Component}
% \begin{Component}[OR]{ShortCiteName} is a shortened version of the
%   \UsageComponent{CiteName} Component. If omitted, it is calculated
%   by the \UsageProp{role-short-cite-name-format} Property.
%    \begin{macrocode}
    \ccDeclareCountedComponent{ShortCiteName}%
%    \end{macrocode}
% \end{Component}
% \begin{Component}[OR]{PDFInfoName} is the version of the Role member
%   name that is used in the PDF Info dictionary.  If omitted, it is calculated
%   by the \UsageProp{role-pdfinfo-name-format} Property.
%    \begin{macrocode}
    \ccDeclareCountedComponent{PDFInfoName}%
%    \end{macrocode}
% \end{Component}
% \begin{Component}[OR]{Initial} holds the initials of the Role
%   member. If omitted, the initials are attempted to be calculated
%   from the \UsageComponent{FirstName} and \UsageComponent{MidName}
%   Components via the \UsageProp{initials-format} Property.
%    \begin{macrocode}
    \ccDeclareCountedComponent{Initial}%
%    \end{macrocode}
% \end{Component}
% \begin{Component}{LastName} is the surname of the Role member.
%    \begin{macrocode}
    \ccDeclareCountedComponent{LastName}%
%    \end{macrocode}
% \end{Component}
% \begin{Component}{FirstName} is the first name of the Role member.
%    \begin{macrocode}
    \ccDeclareCountedComponent{FirstName}%
%    \end{macrocode}
% \end{Component}
% \begin{Component}{MidName} is/are the middle name(s) of the Role
%   member.
%    \begin{macrocode}
    \ccDeclareCountedComponent{MidName}%
%    \end{macrocode}
% \end{Component}
% \begin{Component}{Honorific} is a other honorific title for the Role
%   member.
%    \begin{macrocode}
    \ccDeclareCountedComponent{Honorific}%
%    \end{macrocode}
% \end{Component}
% \begin{Component}{Lineage} is the name suffix, typically something
%   like “jr.” or “the 3rd”.
%    \begin{macrocode}
    \ccDeclareCountedComponent{Lineage}%
%    \end{macrocode}
% \end{Component}
% \begin{Component}{ORCID} is the ORCID (Open Researcher and
%   Contributor ID) of the Role member. Depending on the publisher
%   style, this can be a full URL or just the identifier.
%    \begin{macrocode}
    \ccDeclareCountedComponent{ORCID}%
%    \end{macrocode}
% \end{Component}
% \begin{Component}{AffilRef} is the ID of an entry in the Affil Component Group.
%    \begin{macrocode}
    \ccDeclareCountedComponent{AffilRef}% for references to the Affil Group
%    \end{macrocode}
% \end{Component}
% \begin{Component}{Affiliation} is the Affiliation of the Role
%   member.
%
%   \textbf{Note} that only one \UsageComponent{AffilRef} or
%   \UsageComponent{Affiliation} should be used for any Role member,
%   not both at the same time.
%    \begin{macrocode}
    \ccDeclareCountedComponent{Affiliation}% for affiliations  as direct Author meta data
%    \end{macrocode}
% \end{Component}
% \begin{Component}{Email} is the email/contact address of the Role
%   member.
%    \begin{macrocode}
    \ccDeclareCountedComponent{Email}%
%    \end{macrocode}
% \end{Component}
% \begin{Component}[OR]{CorrespondenceAs} is how the Role member is to
%   be addressed when she is the corresponding Role member. If
%   omitted, it is calculated by the
%   \UsageProp{role-correspondence-as-format} Property.
%    \begin{macrocode}
    \ccDeclareCountedComponent{CorrespondenceAs}%
%    \end{macrocode}
% \end{Component}
%
%
% \subsubsection{Contributor Role Group Handlers}
%
% The Group Handlers fill the previously defined Override Components
% when they are not explicitly given inside the Component Group.
%    \begin{macrocode}
  }%
  \ccDeclareGroupHandler{#2}{%
    \ccUnlessComp{FullName}{\ccComponent{FullName}{\ccUseProperty{#1-full-name-format}}}%
    \ccUnlessComp{Initial}{\ccComponent{Initial}{\ccUseProperty{initials-format}}}%
    \ccUnlessComp{CiteName}{\ccComponent{CiteName}{\ccUseProperty{#1-cite-name-format}}}%
    \ccUnlessComp{ShortCiteName}{\ccComponent{ShortCiteName}{\ccUseProperty{#1-short-cite-name-format}}}%
    \ccUnlessComp{PDFInfoName}{\ccComponent{PDFInfoName}{\ccUseProperty{#1-pdfinfo-name-format}}}%
    \ccUnlessComp{CorrespondenceAs}{\ccComponent{CorrespondenceAs}{\ccUseProperty{#1-correspondence-as-format}}}%
    \ccWhenComp{AffilRef}{\ccWhenComp{Affiliation}{%
        \ccPackageError{Meta}{Ambiguity}
          {You cannot use both Containers AffilRef and Affiliation in the same `\ccPrefix#2' Sub-Container}
          {At least one `\ccPrefix#2' Sub-Container contains both AffilRef and Affiliation. This is not allowed. Please decide for one affiliation strategy: Either two lists with cross-references, or affiliations directly as an author's meta-data.}}}%
  }%
%    \end{macrocode}
%
%
% \paragraph{Declaring the Contributor Role's Collection Components}
%
% Recall that the Collection Component's name are all prefixed by the
% Role's name, e.g., the actual Collection Component
% \UsageComponent{NameList} of a Role named “Author” is accessed by
% calling the \UsageComponent{AuthorNameList} Component.
%
% \begin{Component}[CL]{NameList} is the formatted list of all Role
%   member's \UsageComponent{FullName} Components according to the
%   \UsageProp{role-block-print-format} Property.\DescribeComp{AuthorNameList}
%    \begin{macrocode}
  \ccDeclareRoleBlock{#2}{NameList}{#1-list-print-format}%
%    \end{macrocode}
% \end{Component}
% \begin{Component}[CL]{CitationList} is the formatted list of all
%   Role member's \UsageComponent{CiteName} according to the
%   \UsageProp{role-block-cite-format} Property.\DescribeComp{AuthorCitationList}
%    \begin{macrocode}
  \ccDeclareRoleBlock{#2}{CitationList}{#1-list-cite-format}%
%    \end{macrocode}
% \end{Component}
% \begin{Component}[CL]{ShortCitationList} is the formatted list of
%   all Role member's \UsageComponent{ShortCiteName} Component
%   \UsageProp{role-block-short-cite-format} Property.\DescribeComp{AuthorShortCitationList}
%    \begin{macrocode}
  \ccDeclareRoleBlock{#2}{ShortCitationList}{#1-list-short-cite-format}%
%    \end{macrocode}
% \end{Component}
% \begin{Component}[CL]{PDFInfo} is the formatted string that is sent
%   to the PDF's Info dictionary. Its format is determined by the
%   \UsageProp{role-block-pdfinfo-format} Property.\DescribeComp{AuthorPDFInfo}
%    \begin{macrocode}
  \ccDeclareRoleBlock[apply]{#2}{PDFInfo}{#1-list-pdfinfo-format}%
%    \end{macrocode}
% \end{Component}
% \begin{Component}[CL]{Correspondence} is the list of all Role
%   member's \UsageComponent{CorrespondanceAs} Components according to
%   the \UsageProp{role-block-correspondence-format} Property.\DescribeComp{AuthorCorrespondence}
%    \begin{macrocode}
  \ccDeclareRoleBlock{#2}{Correspondence}{#1-list-correspondence-format}%
%    \end{macrocode}
% \end{Component}
%    \begin{macrocode}
}
%    \end{macrocode}
% \end{macro}
% \begin{macro}{\ccAddToRole} appends another Component declaration
%   block \InlineArg{2} to a pre-defined Role
%   \InlineArg{1}. \TODO{make into LaTeX kernel hook}
%    \begin{macrocode}
\def\ccAddToRole#1#2{\csgappto{cc@group@#1@hook}{#2}}
%    \end{macrocode}
% \end{macro}
%
%
% \section{Labeled Components}
%
% Labeled Components are two Components, one for the Content and one
% for the Label.
%
% \begin{macro}{\ccDeclareLabeledComp} declares two Components: one
%   named \lstinline{\ccPrefix #2} for the value, and another one
%   named \lstinline{\ccPrefix #2Label} for its corresponding
%   label. \#3 is used for Property overrides. The optional Argument
%   \#1 allows to set a default value for the Label.
%    \begin{macrocode}
\def\ccDeclareLabeledComp{\cc@opt@empty\cc@declare@labeled@comp}
\def\cc@declare@labeled@comp[#1]#2#3{%
  \ccDeclareComponent{#2}{\expandafter\global}{}%
  \ccDeclareComponent{#2Label}{\expandafter\global}{}%
  \csxdef{labeled-meta-property-infix-\cc@cur@cont-#2}{#3}%
  \if!#1!\else
    \long\csgdef{cc@\cc@cur@cont @#2Label}{#1}%
  \fi\ignorespaces}
%    \end{macrocode}
% \end{macro}
% \begin{macro}{\ccUseLabeledComp} returns the Labeled Component with
%   its label. The starred version omits automatic Tagging if the
%   \lstinline{coco-accessibility} module is active.
%    \begin{macrocode}
\def\ccUseLabeledComp{\@ifstar{\global\let\ccm@no@tag\relax\cc@use@labeled@comp}{\cc@use@labeled@comp}}
\def\cc@use@labeled@comp#1{%
  \ccWhenComp{#1}{%
%    \end{macrocode}
%
% \begin{macro}{\ccCurInfix} stores the currently active Property
%   infix for the Labeled Component. Is is used to call the right
%   format Property for the Labeled Component, which defaults to
%   \UsageProp{labeled-meta-[ccCurInfix]-format}. If this Property
%   doesn't exists, formatting falls back to
%   \UsageProp{labeled-meta-format}.
%    \begin{macrocode}
    \letcs\ccCurInfix{labeled-meta-property-infix-\cc@cur@cont-#1}%
%    \end{macrocode}
% \end{macro}
% \begin{macro}{\ccCurComp} stores the currently active name of the
%   Labeled Component, which is used in the generic
%   \UsageProp{labeled-meta-format} Property.
%    \begin{macrocode}
    \def\ccCurComp{#1}%
%    \end{macrocode}
% \end{macro}
%    \begin{macrocode}
    \ifx\ccm@no@tag\relax\else\ccaStructStart{MetaDatum}\fi
    \ccIfProp{labeled-meta-\ccCurInfix-format}
      {\ccUseProperty{labeled-meta-\ccCurInfix-format}}
      {\ccUseProperty{labeled-meta-format}}%
    \ifx\ccm@no@tag\relax\else\ccaStructEnd{MetaDatum}\fi
  }\global\let\ccm@no@tag\@undefined}
%    \end{macrocode}
%
%
% \section{Meta Data Rolemaps for Tagged PDFs}
%
% Role mapping for accessibility tagging:
%    \begin{macrocode}
\ccaAddRolemap{Authors}{P}
\ccaAddRolemap{Editors}{P}
\ccaAddRolemap{SeriesEditors}{P}
\ccaAddRolemap{Affiliations}{P}
\ccaAddRolemap{MetaDatum}{Div}
\ccaAddRolemap{MetaDatumLabel}{P}
\ccaAddRolemap{MetaDatumValue}{P}
\ccaAddRolemap{MetaDatumBlock}{Div}
\ccaAddRolemap{Abstract}{Div}
\ccaAddRolemap{AbstractLabel}{P}
\ccaAddRolemap{AbstractText}{Div}
\ccaAddRolemap{Keywords}{Div}
\ccaAddRolemap{KeywordsLabel}{P}
\ccaAddRolemap{KeywordsText}{Div}
%    \end{macrocode}
% \end{macro}
%
% \section{Common Meta Data}
%
% \begin{macro}{\ccm@declare@comp} defines some commonly used meta Components
%    \begin{macrocode}
\def\ccm@declare@comp{%
%    \end{macrocode}
% \begin{Component}{Copyright} holds the Copyright notice.
%    \begin{macrocode}
    \ccDeclareComponent{Copyright}{\expandafter\global}{}% Copyright text
%    \end{macrocode}
% \end{Component}
% \begin{Component}{LicenceLogo} is a component for a license
%   logo. This usually contains an \lstinline{\includegraphics}.
%    \begin{macrocode}
    \ccDeclareComponent{LicenceLogo}{}{}%
%    \end{macrocode}
% \end{Component}
% \begin{Component}{LicenceName} is the name of the license.
%    \begin{macrocode}
    \ccDeclareComponent{LicenceName}{}{}%
%    \end{macrocode}
% \end{Component}
%    \begin{macrocode}
}%
%    \end{macrocode}
% \end{macro}
% \DescribeContainer{article-meta} is an abstract Container that holds
% meta data specific to a journal's Article.
%    \begin{macrocode}
  %% for single articles
\ccDeclareContainer{article-meta}{%
  \ccDeclareType{Components}{%
%    \end{macrocode}
% \begin{Component}{StartPage} is the number of the starting page of
%   an article
%    \begin{macrocode}
    \ccDeclareGlobalComponent{StartPage}
%    \end{macrocode}
% \end{Component}
% \begin{Component}{EndPage} is the number of the ending page of an
%   article
%    \begin{macrocode}
    \ccDeclareGlobalComponent{EndPage}
%    \end{macrocode}
% \end{Component}
% \begin{Component}{CiteAs} holds a string as to how the article
%   should be cited in other publications.
%    \begin{macrocode}
    \ccDeclareLabeledComp[Cite as]{CiteAs}{cite-as}
%    \end{macrocode}
% \end{Component}
% \begin{Component}{Submitted} holds the date when the article was
%   submitted to the journal.
%    \begin{macrocode}
    \ccDeclareLabeledComp[Submitted]{Submitted}{sumbitted}
%    \end{macrocode}
% \end{Component}
% \begin{Component}{Received} holds the date when the article was
%   received by the journal
%    \begin{macrocode}
    \ccDeclareLabeledComp[Received]{Received}{received}
%    \end{macrocode}
% \end{Component}
% \begin{Component}{Revised} holds the date when the article was
%   revised by its author(s)
%    \begin{macrocode}
    \ccDeclareLabeledComp[Revised]{Revised}{revised}
%    \end{macrocode}
% \end{Component}
% \begin{Component}{Reviewed} holds the date when the article was
%   reviewed by the editors or reviewers.
%    \begin{macrocode}
    \ccDeclareLabeledComp[Reviewed]{Reviewed}{reviewed}
%    \end{macrocode}
% \end{Component}
% \begin{Component}{Accepted} holds the date when the article was
%   accepted for publication by the journal.
%    \begin{macrocode}
    \ccDeclareLabeledComp[Accepted]{Accepted}{accepted}
%    \end{macrocode}
% \end{Component}
% \begin{Component}{Published} holds the date when the article is due
%   to be published.
%    \begin{macrocode}
    \ccDeclareLabeledComp[Published]{Published}{published}
%    \end{macrocode}
% \end{Component}
% \begin{Component}{COIStatement} holds the author's Conflict of
%   Interest statement
%    \begin{macrocode}
    \ccDeclareLabeledComp[Conflict of Interest]{COIStatement}{coi-statement}
%    \end{macrocode}
% \end{Component}
%    \begin{macrocode}
  }}
%    \end{macrocode}
% 
% \begin{macro}{\ccm@extended@common@macros}
%   provides some extended markup. Some headings use these Components
%   for compilations of contributions by different authors. They are
%   also loaded by article title pages.
%    \begin{macrocode}
\def\ccm@extended@common@macros{%
%    \end{macrocode}
% \begin{Component}{Abstract} holds the contribution's abstract or
%   content summary.
%    \begin{macrocode}
  \ccDeclareLabeledComp[Abstract]{Abstract}{abstract}%
%    \end{macrocode}
% \end{Component}
% \begin{Component}{Keywords} holds a list of keywords related to the
%   contribution.
%    \begin{macrocode}
  \ccDeclareLabeledComp[Keywords]{Keywords}{keyword}%
%    \end{macrocode}
% \end{Component}
% \begin{Component}{DOI} holds the Digital Object
%   Identifier. Depending on the Publisher style, this may be the full
%   URI, or just the identifier.
%    \begin{macrocode}
  \ccDeclareLabeledComp{DOI}{doi}%
%    \end{macrocode}
% \end{Component}
% \begin{Component}{TitleEn} holds the English title of the
%   publication when the contribution's main language is \textit{not}
%   english.
%    \begin{macrocode}
  \ccDeclareLabeledComp{TitleEn}{title-en}%
%    \end{macrocode}
% \end{Component}
%    \begin{macrocode}
  \ccm@generic@comp
}
%    \end{macrocode}
% \end{macro}
%
%
% \subsection{Affiliations}
%
% \begin{macro}{\ccm@declare@affils} is a wrapper that creates the
%   user-level macros for the affiliations.
%    \begin{macrocode}
\def\ccm@declare@affils{%
%    \end{macrocode}
% \begin{Component}[CC]{AffilBlock} is the Collection Component for
%   the contribition's Affiliations list. \textbf{Note} that the
%   \UsageComponent{AffilBlock} itself is not generated in this
%   module. The two modules \lstinline{coco-headings} and
%   \lstinline{coco-title} that both depend on the
%   \lstinline{coco-meta} module have their own mechanisms to build
%   their respective \UsageComponent{AffilBlock} Collection
%   Components.
%    \begin{macrocode}
  \ccDeclareComponent{AffilBlock}{\expandafter\global}{}%
%    \end{macrocode}
% \end{Component}
%    \begin{macrocode}
  \ccDeclareComponentGroup{Affil}{%
%    \end{macrocode}
% \begin{Component}[OR]{Affiliation} is the fully formatted
%   Affiliation string. If omitted, the Component is built using the
%   \UsageProp{affiliation-format} Property.
%    \begin{macrocode}
    \ccDeclareCountedComponent{Affiliation}%
%    \end{macrocode}
% \end{Component}
% \begin{Component}{Address} is the address where the Role member is
%   working.
%    \begin{macrocode}
    \ccDeclareCountedComponent{Address}%
%    \end{macrocode}
% \end{Component}
% \begin{Component}{Institute} is the name of the university,
%   department or institution where the Role member is working
%    \begin{macrocode}
    \ccDeclareCountedComponent{Institute}%
%    \end{macrocode}
% \end{Component}
% \begin{Component}{Country} is the country where the institution is
%   locaed in.
%    \begin{macrocode}
    \ccDeclareCountedComponent{Country}%
%    \end{macrocode}
% \end{Component}
% \begin{Component}{Department} is the department where the Role
%   member is working.
%    \begin{macrocode}
    \ccDeclareCountedComponent{Department}%
%    \end{macrocode}
% \end{Component}
% \begin{Component}[OR]{AffilID} is the internal identifier that is
%   referenced by the Role member's \UsageComponent{AffilRef}
%   Component. If omitted, the ID is the value of an automatic counter
%   that is incremented by one at the beginning of each Affil Group
%   Container counter in the same Parent Container.
%    \begin{macrocode}
    \ccDeclareCountedComponent{AffilID}%
%    \end{macrocode}
% \end{Component}
%    \begin{macrocode}
  }%
  \ccDeclareGroupHandler{Affil}{%
    \ccUnlessComp{AffilID}{\ccComponentEA{AffilID}{\ccAffilCnt}}%
    \ccUnlessComp{Affiliation}{\ccComponent{Affiliation}{\ccUseProperty{affiliation-format}}}%
  }%
}
%    \end{macrocode}
% \end{macro}
%
%
% \section{Meta Data Properties}
%
%    \begin{macrocode}
\ccAddToType{Properties}{CommonMeta}{%
%    \end{macrocode}
%
%
% \subsection{Initials}
%
% \begin{Property}{initials-format}{<any>} generates an Role member's
%   initials from the \UsageComponent{FirstName} and
%   \UsageComponent{MidName} Components.
%    \begin{macrocode}
  \ccSetProperty{initials-format}{%
    \expandafter\ifx\csname cc@\cc@cur@cont @\cc@cnt@grp-FirstName-\the\ccCurCount\endcsname\cc@long@empty\else
      \expandafter\ifx\csname cc@\cc@cur@cont @\cc@cnt@grp-FirstName-\the\ccCurCount\endcsname\relax\else
        \expandafter\expandafter\expandafter\@car\csname cc@\cc@cur@cont @\cc@cnt@grp-FirstName-\the\ccCurCount\endcsname\relax\@nil\ccUseProperty{initials-period}%
      \expandafter\ifx\csname cc@\cc@cur@cont @\cc@cnt@grp-MidName-\the\ccCurCount\endcsname\cc@long@empty\else
        \expandafter\ifx\csname cc@\cc@cur@cont @\cc@cnt@grp-MidName-\the\ccCurCount\endcsname\relax\else
          \ccUseProperty{initials-sep}%
          \expandafter\expandafter\expandafter\@car\csname cc@\cc@cur@cont @\cc@cnt@grp-MidName-\the\ccCurCount\endcsname\relax\@nil\ccUseProperty{initials-period}%
        \fi\fi
    \fi\fi
  }
%    \end{macrocode}
% \end{Property}
% \begin{Property}{initials-sep}{<any>} is the separator between two
%   \UsageComponent{Initial}s.
%    \begin{macrocode}
  \ccSetProperty{initials-sep}{~}
%    \end{macrocode}
% \end{Property}
% \begin{Property}{initials-period}{<any>} is the symbol that is
%   inserted at the end of each \UsageComponent{Initial}.
%    \begin{macrocode}
  \ccSetProperty{initials-period}{.}
%    \end{macrocode}
% \end{Property}
%
%
% \subsection{Member Role Composition Properties}
%
% \subsection{Overrides Within a Role Counted Component}
%
% The next Properties control how the Compoent Overrides within a
% single Role Counted Component are composed.
% \begin{Property}{role-full-name-format}{<any>} how the
%   \UsageComponent{FullName} Component for each Role member is
%   generated.
%    \begin{macrocode}
  \ccSetProperty{role-full-name-format}{%
    \if\ccUseComp{Honorific}\relax
    \else
      \ccUseComp{Honorific}\space
    \fi
    \ccUseComp{FirstName}\space
    \if\ccUseComp{MidName}\relax
    \else
      \ccUseComp{MidName}\space
    \fi
    \ccUseComp{LastName}%
    \if\ccUseComp{Lineage}\relax
    \else
      \space\ccUseComp{Lineage}%
    \fi%
  }%
%    \end{macrocode}
% \end{Property}
% \begin{Property}{role-cite-name-format}{<any>} how the
%   \UsageComponent{CiteName} for each Role member is formatted.
%    \begin{macrocode}
  \ccSetProperty{role-cite-name-format}{\ccIfComp{LastName}{\ccUseComp{LastName},~\ccUseComp{Initial}}{\ccUseComp{FullName}}}% How CiteName for each name is built
%    \end{macrocode}
% \end{Property}
% \begin{Property}{role-short-cite-name-format}{<any>} how the
%   \UsageComponent{ShortCiteName} Component of a Role member is
%   formatted
%    \begin{macrocode}
  \ccSetProperty{role-short-cite-name-format}{\ccUseComp{LastName}}%
%    \end{macrocode}
% \end{Property}
% \begin{Property}{role-pdfinfo-name-format}{<any>} how the
%   \UsageComponent{PDFInfoName} of a Role member is formatted
%    \begin{macrocode}
  \ccPropertyLet{role-pdfinfo-name-format}{role-cite-name-format}%
%    \end{macrocode}
% \end{Property}
% \begin{Property}{role-correspondence-as-format}{<any>} How the
%   \UsageComponent{CorrespondenceAs} string of a Role Member is
%   formatted
%    \begin{macrocode}
  \ccSetProperty{role-correspondence-as-format}{\ccUseComp{Email}}%
%    \end{macrocode}
% \end{Property}
%
%
% \subsection{Format of Single Role Collection Component Items}
%
% the next properties control how single items in the Parent
% container's Collection Components are to be formatted.
%
% \begin{Property}{role-block-print-format}{<any>} How a single entry in
%   the \UsageComponent{NameList} is formatted.
%    \begin{macrocode}
  \ccSetProperty{role-block-print-format}{\ccUseComp{FullName}\ifnum\ccCurCount<\ccTotalCount\ccUseProperty{counted-name-sep}\fi}%
%    \end{macrocode}
% \end{Property}
% \begin{Property}{role-block-cite-format}{<any>} how a single entry in
%   the \UsageComponent{CitationList} is formatted
%    \begin{macrocode}
  \ccSetProperty{role-block-cite-format}{\ccUseComp{CiteName}\ifnum\ccCurCount<\ccTotalCount\ccUseProperty{counted-name-sep}\fi}%
%    \end{macrocode}
% \end{Property}
% \begin{Property}{role-block-short-cite-format}{<any>} how a single
%   entry in the \UsageComponent{ShortCitationList} is formatted
%    \begin{macrocode}
  \ccSetProperty{role-block-short-cite-format}{\ccUseComp{ShortCiteName}\ifnum\ccCurCount<\ccTotalCount\ccUseProperty{counted-name-sep}\fi}%
%    \end{macrocode}
% \end{Property}
% \begin{Property}{role-block-pdfinfo-format}{<any>} how a single entry
%   in the \UsageComponent{PDFInfo} Component is formatted
%    \begin{macrocode}
  \ccSetProperty{role-block-pdfinfo-format}{\ccUseComp{PDFInfoName}\ifnum\ccCurCount<\ccTotalCount\and\fi}% How each item in the Component <Role>PDFInfo is formatted
%    \end{macrocode}
% \end{Property}
% \begin{Property}{role-block-correspondence-format}{<any>} how a single entry
%   in the \UsageComponent{Correspondence} Component is formatted
%    \begin{macrocode}
  \ccSetProperty{role-block-correspondence-format}{%
    \ccIfPropVal{is-corresp}{true}
      {\ifx\is@first@corresp\relax
         \ccUseProperty{corresp-sep}%
       \else
         \global\let\is@first@corresp\relax
       \fi
       \ccUseComp{CorrespondenceAs}%
     }{}}%
%    \end{macrocode}
% \end{Property}
% \begin{Property}{counted-name-sep}{<any>} how single entries in
%   \UsageComponent{NameList} are separated
%    \begin{macrocode}
  \ccSetProperty{counted-name-sep}{,\space}%
%    \end{macrocode}
% \end{Property}
% \begin{Property}{name-and}{<any>} is a Property that can be used when
%   composing Role specific Collection Components. Is is usually used
%   between the penultimate and the last entry in the Collection
%   Component.
%    \begin{macrocode}
  \ccSetProperty{name-and}{\space and\space}%
%    \end{macrocode}
% \end{Property}
% \begin{Property}{name-etal}{<any>} is a Property that can be used when
%   composing Role specific Collection Components. Is is usually used
%   after the first entry in the list, when the total number of
%   entries is too large.
%    \begin{macrocode}
  \ccSetProperty{name-etal}{\space et~al.}%
%    \end{macrocode}
% \end{Property}
% \begin{Property}{name-sep}{<any>} is the default separator between
%   entries in a Role specific Collection Component.
%    \begin{macrocode}
  \ccSetProperty{name-sep}{,\space}%
%    \end{macrocode}
% \end{Property}
% \begin{Property}{keywords-sep}{<any>} is the default separator between
%   Entries in the Keywords list.
%    \begin{macrocode}
  \ccSetProperty{keywords-sep}{,\space}%
%    \end{macrocode}
% \end{Property}
% \begin{Property}{corresp-mark}{<any>} is the default marker for the
%   “Correspondence” Role member, i.e., the Role member who is
%   designated the primary contact person of a contribution.
%    \begin{macrocode}
  \ccSetProperty{corresp-mark}{*}%
%    \end{macrocode}
% \end{Property}
% \begin{Property}{corresp-sep}{<any>} is the default seperator between
%   entries in the \UsageComponent{Correspondence} Collection
%   Component.
%    \begin{macrocode}
  \ccSetProperty{corresp-sep}{,\space}%
%    \end{macrocode}
% \end{Property}
%
%
% \subsection{Collection Component Properties Specific to Author Role}
%
% The Properties defined here are mostly aliases of the more generic Properties defined above.
%
%
% \subsubsection{Author Instance Override Properties}
%
% \begin{Property}{author-cite-name-format}{<any>} how an Author's
%   \UsageComponent{CiteName} is formatted.
%    \begin{macrocode}
  \ccPropertyLet{author-cite-name-format}            {role-cite-name-format}%
%    \end{macrocode}
% \end{Property}
% \begin{Property}{author-short-cite-name-format}{<any>} how an author's
%   \UsageComponent{ShortCiteName} is formatted.
%    \begin{macrocode}
  \ccPropertyLet{author-short-cite-name-format}      {role-short-cite-name-format}%
%    \end{macrocode}
% \end{Property}
% \begin{Property}{author-full-name-format}{<any>} how an author's
%   \UsageComponent{FullName} Component is composed
%    \begin{macrocode}
  \ccPropertyLet{author-full-name-format}            {role-full-name-format}%
%    \end{macrocode}
% \end{Property}
% \begin{Property}{author-pdfinfo-name-format}{<any>} how an author's
%   \UsageComponent{PDFInfoName} Component is composed
%    \begin{macrocode}
  \ccPropertyLet{author-pdfinfo-name-format}         {role-pdfinfo-name-format}%
%    \end{macrocode}
% \end{Property}
% \begin{Property}{author-correspondence-as-format}{<any>} how an
%   author's \UsageComponent{CorrespondenceAs} entry is to be
%   formatted
%    \begin{macrocode}
  \ccPropertyLet{author-correspondence-as-format}    {role-correspondence-as-format}%
%    \end{macrocode}
% \end{Property}
%
%
% \subsubsection{Author-Specific Collcection Component Override Properties}
%
% \begin{Property}{author-list-print-format}{<any>} is the format of
%   each entry in the \UsageComponent{AuthorNameList} Component.
%    \begin{macrocode}
  \ccPropertyLet{author-list-print-format}           {role-block-print-format}%
%    \end{macrocode}
% \end{Property}
% \begin{Property}{author-list-cite-format}{<any>} is the format of each
%   entry in the \UsageComponent{AuthorCitationList} Component.
%    \begin{macrocode}
  \ccPropertyLet{author-list-cite-format}            {role-block-cite-format}%
%    \end{macrocode}
% \end{Property}
% \begin{Property}{author-list-short-cite-format}{<any>} is the format
%   of each entry in the \UsageComponent{AuthorShortCitationList}
%   Component.
%    \begin{macrocode}
  \ccPropertyLet{author-list-short-cite-format}      {role-block-short-cite-format}%
%    \end{macrocode}
% \end{Property}
% \begin{Property}{author-list-pdfinfo-format}{<any>} is the format of
%   each entry in the \UsageComponent{AuthorPDFInfo} Component.
%    \begin{macrocode}
  \ccPropertyLet{author-list-pdfinfo-format}         {role-block-pdfinfo-format}%
%    \end{macrocode}
% \end{Property}
% \begin{Property}{author-list-correspondence-format}{<any>} is the
%   format of the \UsageComponent{AuthorContribution} Collection
%   Component.
%    \begin{macrocode}
  \ccPropertyLet{author-list-correspondence-format}  {role-block-correspondence-format}%
%    \end{macrocode}
% \end{Property}
%
%
% \subsection{Format of Affiliation Lists}
%
% \begin{Property}{affiliation-format}{<any>} is the format of the
%   \UsageComponent{Affiliation} Component for each
%   \UsageComponent{Affil} Instance.
%    \begin{macrocode}
  \ccSetProperty{affiliation-format}{%
    \ccWhenComp{Institute}{\ccUseComp{Institute}}%
    \ccWhenComp{Department}{, \ccUseComp{Department}}%
    \ccWhenComp{Address}{, \ccUseComp{Address}}%
  }%
%    \end{macrocode}
% \end{Property}
% \begin{Property}{affil-sep}{<any>} is the separator between the
%   entries of the \UsageComponent{AffilBlock} Collection Component.
%    \begin{macrocode}
  \ccSetProperty{affil-sep}{\par}
%    \end{macrocode}
% \end{Property}
% \begin{Property}{affil-block-item-face}{<any>} are the font
%   parameters used to print each entry in the
%   \UsageComponent{AffilBlock} Collection Component.
%    \begin{macrocode}
  \ccSetProperty{affil-block-item-face}{}%
%    \end{macrocode}
% \end{Property}
% \begin{Property}{affil-block-item-format} is the format of each
%   entry in the \UsageComponent{AffilBlock} list
%    \begin{macrocode}
  \ccSetProperty{affil-block-item-format}{%
    \textsuperscript{\ccUseComp{AffilID}}%
    \bgroup
      \ccUseProperty{affil-block-item-face}%
      \ccUseComp{Affiliation}
    \egroup%
    \ifnum\ccCurCount<\ccTotalCount\relax\ccUseProperty{affil-sep}\fi%
  }
%    \end{macrocode}
% \end{Property}
% \begin{Property}{affil-block-face}{<any>} is the font used to print
%   the \UsageComponent{AffilBlock} Collection Component.
%    \begin{macrocode}
  \ccSetProperty{affil-block-face}{\small\normalfont}%
%    \end{macrocode}
% \end{Property}
% \begin{Property}{affil-block-format}{<any>} prints the
%   \UsageComponent{AffilBlock} Collection Component.
%    \begin{macrocode}
  \ccSetProperty{affil-block-format}{%
    \ccWhenComp{AffilBlock}
      {\bgroup
         \ccUseProperty{affil-block-face}%
         \ccUseComp{AffilBlock}%
       \egroup
       \par
     }}
%    \end{macrocode}
% \end{Property}
%
%
% \subsection{Properties for Labeled Componetns}
%
% \begin{Property}{labeled-meta-format}{<any>}\DescribeProperty{labeled-meta-[infix]-format}
%   is the generic Property that determins how Labeled Components are
%   composed. It checks for implicit formatting properties speific to
%   each labeled Component and falls back to generic defaults if those
%   are not defined by the user or publisher style.
%    \begin{macrocode}
  \ccSetProperty{labeled-meta-format}{%
    \ccIfProp{labeled-meta-before-\ccCurInfix}
      {\ccUseProperty{labeled-meta-before-\ccCurInfix}}
      {\ccUseProperty{labeled-meta-before}}%
    \bgroup
      \ifx\ccm@no@tag\relax\else\ccaStructStart{MetaDatumLabel}\fi
      \ccIfProp{labeled-meta-\ccCurInfix-face}
        {\ccUseProperty{labeled-meta-\ccCurInfix-face}}
        {\ccUseProperty{labeled-meta-face}}%
      \ccIfProp{labeled-meta-\ccCurInfix-label-format}
        {\ccUseProperty{labeled-meta-\ccCurInfix-label-format}}
        {\ccUseProperty{labeled-meta-label-format}}%
      \ifx\ccm@no@tag\relax\else\ccaStructEnd{MetaDatumLabel}\fi
      \ifx\ccm@no@tag\relax\else\ccaStructStart{MetaDatumValue}\fi
      \ccUseComp{\ccCurComp}%
      \ifx\ccm@no@tag\relax\else\ccaStructEnd{MetaDatumValue}\fi
    \egroup
    \ccIfProp{labeled-meta-after-\ccCurInfix}
      {\ccUseProperty{labeled-meta-after-\ccCurInfix}}
      {\ccUseProperty{labeled-meta-after}}%
  }
%    \end{macrocode}
% \end{Property}
% \begin{Property}{labeled-meta-label-format}{<any>}\DescribeProperty{labeled-meta-[infix]-label-format}\DescribeProperty{labeled-meta-before-[infix]-label}
%   is the generic format of the label of a Labeled Component.\UsageComp{Label}
%    \begin{macrocode}
  \ccSetProperty{labeled-meta-label-format}{%
    \ccWhenComp{\ccCurComp Label}{%
      \bgroup
        \ccUseProperty{labeled-meta-before-\ccCurInfix-label}%
        \ccIfProp{labeled-meta-\ccCurInfix-label-face}
          {\ccUseProperty{labeled-meta-\ccCurInfix-label-face}}
          {\ccUseProperty{labeled-meta-label-face}}%
        \ccUseComp{\ccCurComp Label}%
        \ccIfProp{labeled-meta-\ccCurInfix-label-sep}
          {\ccUseProperty{labeled-meta-\ccCurInfix-label-sep}}
          {\ccUseProperty{labeled-meta-label-sep}}%
      \egroup
    }}
%    \end{macrocode}
% \end{Property}
% \begin{Property}{labeled-meta-label-face}{<any>}\DescribeProperty{labeled-meta-[infix]-label-face} is the font setting
%   for the Label of a Labeled Component.
%    \begin{macrocode}
  \ccSetProperty{labeled-meta-label-face}{\bfseries}
%    \end{macrocode}
% \end{Property}
% \begin{Property}{labeled-meta-label-sep}{<any>}\DescribeProperty{labeled-meta-[infix]-label-sep}
%   is the default and fallback separator between the Labeled
%   Component's \UsageComponent{Label} and its value.
%    \begin{macrocode}
  \ccSetProperty{labeled-meta-label-sep}{:\enskip}
%    \end{macrocode}
% \end{Property}
% \begin{Property}{labeled-meta-face}{<any>}\DescribeProperty{labeled-meta-[infix]-face} is the face of a Labeled
%   Component. It applies to bothe the Label and the Value, but can be
%   locally overridden by the \UsageProp{labeled-meta-label-face}
%   Property.
%    \begin{macrocode}
  \ccSetProperty{labeled-meta-face}{}
%    \end{macrocode}
% \end{Property}
% \begin{Property}{labeled-meta-before}{<any>}\DescribeProperty{labeled-meta-[infix]-before}
%   is the code expanded before the Labeled Component is printed. 
%
%   \textbf{Note} that the Property is expanded \textit{outside} the
%   local group of the Labeled Compoent.
%    \begin{macrocode}
  \ccSetProperty{labeled-meta-before}{}
%    \end{macrocode}
% \end{Property}
% \begin{Property}{labeled-meta-after}{<any>}\DescribeProperty{labeled-meta-[infix]-after}
%   is the code expanded after the Labeled Component is printed.
%
%   \textbf{Note} that the Property is expanded \textit{outside} the
%   local group of the Labeled Compoent.
%    \begin{macrocode}
  \ccSetProperty{labeled-meta-after}{\par}
%    \end{macrocode}
% \end{Property}
%    \begin{macrocode}
}
%    \end{macrocode}
%    \begin{macrocode}[numbers=none,gobble=1]
%</meta>
%    \end{macrocode}
