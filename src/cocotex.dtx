% \chapter{cocotex.dtx}
%
%    \begin{macrocode}[numbers=none,gobble=1]
%<*class>
%    \end{macrocode}
% This is the main class file for the {\CoCoTeX} Framework.
%
% File Preamble
%    \begin{macrocode}
%%
%% Common document class for \textit{xerif} projects.
%%
%% Maintainer: p.schulz@le-tex.de
%%
%% lualatex  -  texlive > 2019
%%
\NeedsTeXFormat{LaTeX2e}[2018/12/01]
\ProvidesClass{cocotex}
    [\filedate \fileversion cocotex]
%    \end{macrocode}
%
%
% \section{Hard-coded requirements}
%
%    \begin{macrocode}
\RequirePackage{kvoptions-patch}
\RequirePackage{xkeyval}
%    \end{macrocode}
%
%
% \section{Class Options}
%
% Passing options down to the \LaTeX\space standard packages
%    \begin{macrocode}
\DeclareOptionX{main}{\PassOptionsToPackage{\CurrentOption}{babel}}
\DeclareOption{es-noindentfirst}{\PassOptionsToPackage{es-noindentfirst}{babel}}
\DeclareOption{es-noshorthands}{\PassOptionsToPackage{es-noshorthands}{babel}}
\PassOptionsToPackage{shorthands=off}{babel}
%    \end{macrocode}
% The option \lstinline{pubtype} (short for ``publication type'') has
% possible four values: \lstinline{mono}, \lstinline{collection},
% \lstinline{journal}, and \lstinline{article}. \lstinline{mono} (also
% the default when no \lstinline{pubtype} is given) and
% \lstinline{collection} are used to switch between single and
% multiple contributor documents; \lstinline{collection} and
% \lstinline{journal} to switch between one-time text collections and
% periodicals, respectively. All three types implicitly load the
% {\LaTeX} standard class \lstinline{book}.
%
% \lstinline{collection} is used when the document's components
% (i.\,e., chapters) are contributed by different authors like
% collections or proceedings. \lstinline{journal} is used for
% collections where each contribution is accompanied by a myriad of
% meta data. \lstinline{mono} stands for monographs, i.e., whole books
% that are written by the same author(s).
%
% The publicaten type \lstinline{article} is intended for single
% articles of a journal. It loads the {\LaTeX} standard class
% \lstinline{article}.
%    \begin{macrocode}
\newif\ifcollection \collectionfalse
\newif\ifarticle    \articlefalse
\newif\ifmonograph  \monographfalse
\newif\ifjournal    \journalfalse
\define@choicekey{cocotex.cls}{pubtype}[\cc@pub@type\nr]{collection,article,journal,mono}{%
  \ifcase\nr\relax% collection
    \global\collectiontrue
  \or% article
    \global\articletrue
  \or% journal
    \global\journaltrue
  \else% monograph
    \global\monographtrue
  \fi
}
\DeclareOptionX*{\PassOptionsToClass{\CurrentOption}{article}}
\DeclareOptionX*{\PassOptionsToClass{\CurrentOption}{book}}
%    \end{macrocode}
% Passing options down to various {\CoCoTeX} modules:
%    \begin{macrocode}
\DeclareOptionX{debug}{\PassOptionsToPackage{\CurrentOption}{coco-kernel}}
\DeclareOptionX{debug-domain}{\PassOptionsToPackage{\CurrentOption}{coco-kernel}}
\DeclareOptionX{a11y}{\PassOptionsToPackage{init}{coco-accessibility}}
\DeclareOptionX{lang-id}{\PassOptionsToPackage{\CurrentOption}{coco-accessibility}}
\DeclareOptionX{nodetree}{\PassOptionsToPackage{\CurrentOption}{coco-accessibility}}
\DeclareOptionX{showspaces}{\PassOptionsToPackage{\CurrentOption}{coco-accessibility}}
\DeclareOptionX{no-spaces}{\PassOptionsToPackage{\CurrentOption}{coco-accessibility}}
\DeclareOptionX{no-paras}{\PassOptionsToPackage{\CurrentOption}{coco-accessibility}}
\DeclareOptionX{no-compress}{\let\cc@no@pdf@compression\relax}
\DeclareOptionX{color-enc}{\PassOptionsToPackage{\CurrentOption}{coco-common}}
\DeclareOptionX{usescript}{\PassOptionsToPackage{\CurrentOption}{coco-script}}
\DeclareOptionX{nofigs}{\PassOptionsToPackage{\CurrentOption}{coco-floats}}
\DeclareOptionX{ennotoc}{\PassOptionsToPackage{\CurrentOption}{coco-notes}}
\DeclareOptionX{endnotes}{\PassOptionsToPackage{\CurrentOption}{coco-notes}}
\DeclareOptionX{resetnotesperchapter}{\PassOptionsToPackage{\CurrentOption}{coco-notes}}
\DeclareOptionX{endnotesperchapter}{\PassOptionsToPackage{\CurrentOption}{coco-notes}}
\ProcessOptionsX
%    \end{macrocode}
%
%
% \section{Class Hook}
%
% \begin{macro}{\ccAfterClassHook}
%   Almost all user level macros have been renamed when {\CoCoTeX}
%   became independent from \lstinline{xerif}. In order to ensure
%   backwards-compatibility, we define a hook that holds aliases from
%   the old names to the new ones. Those are defined in the
%   \lstinline{coco-xerif} module (which is \textit{not} part of
%   {\CoCoTeX} itself, but included in \lstinline{xerif}'s common
%   files). The hook is expanded at the very end of the
%   \lstinline{cocotex.cls} file. The \lstinline{coco-xerif} module
%   itself is loaded early in \lstinline{coco-common.sty}.
%
%   Note that this hook is temporary. As soon as all legacy styles are
%   adjusted to the new macro names, this hook will be removed!
%    \begin{macrocode}
\def\ccAfterClassHook{}
%    \end{macrocode}
% \end{macro}
% \begin{macro}{\ccToggleCountedConditionalsHook} is a hook to ensure
%   backwards-compatibility within the processing of Counted Components
%
%   Note that this hook is temporary. As soon as all legacy styles are
%   adjusted to the new macro names, this hook will be removed!
%    \begin{macrocode}
\def\ccToggleCountedConditionalsHook{}%
%    \end{macrocode}
% \end{macro}
%
%
% \section{Internal Requirement}
%
%    \begin{macrocode}
\RequirePackage{coco-common}
%    \end{macrocode}
%
%
% \section{Loading and Adjusting Underlying DocumentClass}
%
% All publication types supported by {\CoCoTeX} are based on one of
% \LaTeX's default classes \lstinline{book} or \lstinline{article}:
%    \begin{macrocode}
\ifarticle
  \LoadClass[10pt,a4paper]{article}
\else
  \LoadClass[10pt,a4paper]{book}
\fi
%    \end{macrocode}
%
%
% \subsection{General Typography}
%
% Offsets are the removed to make all values relative to the upper left corner of the page to ease maintainance.
%    \begin{macrocode}
\voffset-1in\relax
\hoffset-1in\relax
%    \end{macrocode}
% Automatted typesetting needs some room to play
%    \begin{macrocode}
\emergencystretch=2em
%    \end{macrocode}
% and strong restrictions:
%    \begin{macrocode}
\frenchspacing
\clubpenalty10000
\widowpenalty10000
%    \end{macrocode}
%
%
% \subsubsection{Empty Pagestyle}
%
% Page style without any headers or footers
%    \begin{macrocode}
\def\ps@empty{%
  \let\@oddhead\@empty
  \let\@evenhead\@empty
  \let\@oddfoot\@empty
  \let\@evenfoot\@empty
}
%    \end{macrocode}
%
%
% \subsubsection{Vacancy Pages}
%
% Vacancy pages in general need to have page style \lstinline{empty}:
%    \begin{macrocode}
\def\cleardoublepage{\clearpage\if@twoside \ifodd\c@page\else
    \hbox{}\thispagestyle{empty}\newpage\if@twocolumn\hbox{}\newpage\fi\fi\fi}
%    \end{macrocode}
%
%
% \subsubsection{Book Parts}
%
% re-defined to make front- and backmatter components distinguish-able
%    \begin{macrocode}
\ifarticle\else
  \newif\if@frontmatter  \@frontmatterfalse
  \renewcommand\frontmatter{%
    \cleardoublepage
    \@mainmatterfalse
    \@frontmattertrue
    \pagenumbering{arabic}}

  \renewcommand\mainmatter{%
    \cleardoublepage
    \@frontmatterfalse
    \@mainmattertrue}

  \renewcommand\backmatter{%
    \cleardoublepage
    \@mainmatterfalse
    \@frontmatterfalse}
\fi
%    \end{macrocode}
% \Deprecated
%    \begin{macrocode}
\usepackage{soul}
%    \end{macrocode}
%
%
% \section{Loading other {\CoCoTeX} Modules}
%
% \subsection{coco-accessibility}
%
% We load the accessibility module always, even if we don't end up
% actually using it.
%    \begin{macrocode}
\RequirePackage{coco-accessibility}
%    \end{macrocode}
%
% \subsection{coco-script}
%
% Inclusion of the script module which also loads the babel package
%    \begin{macrocode}
\ifLuaTeX
\RequirePackage{coco-script}
\else
\RequirePackage{babel}
\fi
%    \end{macrocode}
%
%
% \subsection{coco-headings}
%
%    \begin{macrocode}
\RequirePackage{coco-headings}
%    \end{macrocode}
%
%
% \subsection{coco-floats}
%
% Inclusion of the float module
%    \begin{macrocode}
\RequirePackage{coco-floats}
%    \end{macrocode}
%
%
% \subsection{coco-title}
%
% Inclusion of the title page module
%    \begin{macrocode}
\RequirePackage{coco-title}
%    \end{macrocode}
%
%
% \subsection{coco-notes}
%
% Inclusion of the end-/footnotes module
%    \begin{macrocode}
\RequirePackage{coco-notes}
%    \end{macrocode}
% Fallback, in case, \lstinline{coco-headings.sty} is not loaded for
% some reason.
%
%
% \section{Further Hard Dependencies}
%
% \subsection{Index}
%
% Some more hard dependencies:
%    \begin{macrocode}
\RequirePackage{index}
\makeindex
%    \end{macrocode}
%
%
% \subsection{Hyperref}
%
%    \begin{macrocode}
\RequirePackage{hyperref}
%    \end{macrocode}
% Finally, some \lstinline{hyperref} settings
%
% \TODO{check, which of those are better placed inside the local
% publisher's styles}
%    \begin{macrocode}
\hypersetup{%
%    \end{macrocode}
% first, we want links to be breakable
%    \begin{macrocode}
    breaklinks%
%    \end{macrocode}
% and the table of contents not to be automatically linked, as this
% causes problems with the \lstinline{ltpdfa} package and we add the
% links via the \lstinline{coco-common} module, anyways.
%    \begin{macrocode}
   ,linktoc=none%
%    \end{macrocode}
% pdf broders are controlled via the coco-frame module, if necessary
%    \begin{macrocode}
   ,pdfborder={0 0 0}%
%    \end{macrocode}
% The next option causes hyperref to calculate the encoding of
% DocumentInfo and other direct-to-PDF data (bookmarks, etc.)
% automatically
%    \begin{macrocode}
   ,pdfencoding=unicode
   ,unicode=true
%    \end{macrocode}
% Bookmarks are numbered by default.
%    \begin{macrocode}
   ,bookmarksnumbered=true%
   ,bookmarksopen=false%
}
%    \end{macrocode}
% Disables PDF compression when the \lstinline{no-compress} document
% option is set.
%    \begin{macrocode}
\ifx\cc@no@pdf@compression\relax
  \ifx\pdfobjcompresslevel\@undefined
    \edef\pdfobjcompresslevel{\pdfvariable objcompresslevel}%
  \fi
  \pdfcompresslevel=0
  \pdfobjcompresslevel=0
\fi
%    \end{macrocode}
%
%
% \section{End of Dcument Class Hook}
%
% Expanding backwards-compatibility aliases from the coco-xerif module:
%    \begin{macrocode}
\ccAfterClassHook
%    \end{macrocode}
%    \begin{macrocode}[numbers=none,gobble=1]
%</class>
%    \end{macrocode}
