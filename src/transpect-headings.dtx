% \chapter{transpect-headings.dtx}
% This module provides handlers for headings like parts, chapters,
% sections, or inline headings common to all Transpect projects.
%
%    \begin{macrocode}[gobble=1]
%<*headings>
%    \end{macrocode}
%
%    \begin{macrocode}
%%
%% module for le-tex transpect.cls that extends heading objects.
%%
%% Maintainer: p.schulz@le-tex.de
%%
%% lualatex  -  texlive >= 2019
%%
\NeedsTeXFormat{LaTeX2e}[2018/12/01]
\ProvidesPackage{transpect-headings}
    [\filedate \fileversion le-tex transpect headings module]
\RequirePackage{transpect-meta}
%    \end{macrocode}
%
% Headings are handled differently with \lstinline{transpect.cls}
% compared to standard \LaTeX, since transpect manuscripts tend to
% have a whole collection of additional information that are pressed
% into the headings, like subtitles or section authors down to
% subsection level, etc. Therefore, the \lstinline{\@startsection} and
% \lstinline{\@make[s]chapterhead} facilities from {\LaTeX} are no longer
% sufficient. At the same time, the package does not redefine those
% macros and keeps them available for backwards compatibility.
%
% First, we load the \lstinline{bookmark} package:
%    \begin{macrocode}
\RequirePackage{bookmark}
%    \end{macrocode}
% Since we use our own heading levels, we disable all automatically generated bookmarks.
%    \begin{macrocode}
\hypersetup{bookmarksdepth=-999}
%    \end{macrocode}
%
% \section{Facility for declaring heading levels and their layouts}%
% \begin{tpContainer}{heading}
%    \begin{macrocode}
\tpDeclareContainer{tpHeading}{%
  \tpInherit {Components,Properties} from Meta;
  \tpDeclareType{Components}{%
%    \end{macrocode}
% We already have the Author Component inherited from the
% \lstinline{Meta} Container. We therefore just need to declare the
% overrides.
%    \begin{macrocode}
    \tp@heading@authors%
%    \end{macrocode}
% The remaining Components are built as usual.
%    \begin{macrocode}
    \tp@provide@hd@macros{Title}%
    \tp@provide@hd@macros{Subtitle}%
    \tp@provide@hd@macros{Number}%
%    \end{macrocode}
% \Deprecated
%    \begin{macrocode}
    \tpIfProp{extended}{%
      \tp@extended@ht@macros
    }{}%
%    \end{macrocode}
% \lstinline{Quote} and \lstinline{QuoteSource} (so far) don't need
% overrides; they are declared the standard way.
%    \begin{macrocode}
    \tpDeclareComp{Quote}{}{}%
    \tpDeclareComp{QuoteSource}{}{}%
  }%
  \tpDeclareType{Properties}{\tp@heading@default}%
  \tpDeclareEnv{\@heading}{\end@heading}%
}
%    \end{macrocode}
% \end{tpContainer}
% \begin{macro}{\tpDeclareHeading} is the user-level macro to declare
%   new headings.
% \begin{description}[2em]
% \item[\#1] (optional) inherit-from: load all properties from that heading level, first.
% \item[\#2] level: used for toc entries. -1 for part, 0 for chapter, 1 for section, etc.
% \item[\#3] name: part, chapter, section, etc, to be used in toc, head lines, bookmarks, etc.
% \item[\#4] Property definitions and switches
% \end{description}
%    \begin{macrocode}
\long\def\tpDeclareHeading{\tp@opt@empty\@tpDeclareHeading}
\long\def\@tpDeclareHeading[#1]#2#3#4{%
%    \end{macrocode}
% Each new heading constitutes its own Sub-Container of the heading
% Container. The name of this Sub-Container is the headings name.
%    \begin{macrocode}
  \tpDeclareContainer{#3}{%
    \edef\@argi{#1}%
    \tpDeclareType{Parent}{\tp@heading@create@parent{#1}{#3}}
%    \end{macrocode}
% We inherit everything from the heading levels parent, or from the
% default heading if no parent is present.
%    \begin{macrocode}
    \ifx\@argi\@empty
      \tpInherit {Components,Properties} from tpHeading;
    \else
      \tpInherit {Components,Properties,Parent} from #1;
    \fi
%    \end{macrocode}
% The main body of the heading Declaration is a list of Property
% definitions which we append to the Sub-Container's ``Property''
% Type.
%    \begin{macrocode}
    \tpDeclareType{Properties}{%
      #4%
    }%
%    \end{macrocode}
% For each heading we declare some common macros like the ToC entry
% handlers, the heading's counters and its hooks.
%    \begin{macrocode}
    \tpDeclareType{Init}{%
      \tp@init@hooks{#3}%
      \tp@init@l@{toc}{#2}{#3}%
      \tp@init@cnt{#3}%
      \tp@restore@init{heading}{number-#2-maxwd}%
    }%
%    \end{macrocode}
% Unlike other Sub-Containers, headings form no own {\LaTeX}
% environment.  Instead, headings are specifications of one common
% \lstinline{heading} environment. Is is outsourced into the internal
% \lstinline{\tp@declare@heading} macro, which is defined below.
%
% The reason for that is that we don't want to define versions of the
% same property macros for each and every single heading
% level. Instead, we locally re-define the general low-level macros
% that represent the heading's properties for each instance of the
% generalised \lstinline{heading} container.
%    \begin{macrocode}
    \tp@declare@heading{#2}{#3}%
  }%
}
%    \end{macrocode}
% \end{macro}
% \begin{macro}{\tp@heading@create@parent} stores the heading level's name and its
%   parent, if it exists.
%    \begin{macrocode}
\def\tp@heading@create@parent#1#2{%
  \def\tp@heading@name{#2}%
  \if!#1!\else
    \tpCheckParent{#1}{#2}%
  \fi%
}
%    \end{macrocode}
% \end{macro}
% \begin{macro}{\tp@declare@heading} consists of two parts: In the
%   first part, the inheritance mechanism and the initializers for
%   each new heading level are triggered.
%    \begin{macrocode}
\def\tp@declare@heading#1#2{%
  \tpEvalType{Parent}%
  \tpEvalType{Init}%
%    \end{macrocode}
%
% \begin{macro}{\tpUseHeading} is defined as second step. It is called
%   at the end of each \lstinline{heading} environment to process the
%   Components within the Container instance. Each heading level has
%   its own ``version'' of this macro.
%    \begin{macrocode}
  \csgdef{tpUseHeading#2}{%
%    \end{macrocode}
% Since heading levels don't define their own environments, we make
% sure that \lstinline{heading} is the namespace we are working in.
%    \begin{macrocode}
    \tpNamespace{heading}%
    \@setpar{\@@par}%
%    \end{macrocode}
% Properties are stored in macros specific to the current heading
% Sub-Container, therefore we evaluate the level's Properties, not
% those of the \lstinline{heading} Container. However, since we made
% use of the inheritance mechanism earlier, each Sub-Container's
% Property list also contains the general \lstinline{heading} Property
% list.
%    \begin{macrocode}
    \tpEvalType[#2]{Properties}%
%    \end{macrocode}
% Immediately after setting the cascaded Properties we execute those
% lists to be able to access the Properties.
%    \begin{macrocode}
    \csname tp@heading@#2@properties\endcsname
%    \end{macrocode}
% Processing the author name list (from transpect-meta.sty).
%    \begin{macrocode}
    \tp@meta@role@eval{Author}%
    %\tpComposeCollection{tpAuthor}{author-contact-block-format}{AuthorContactBlock}%
%    \end{macrocode}
% Hyperref related stuff.
%    \begin{macrocode}
    \def\Hy@toclevel{#1}%
%    \end{macrocode}
% Call the mechanism to calculate the heading's counter.
%    \begin{macrocode}
    \tp@auto@number{#1}{#2}%
%    \end{macrocode}
% Here, the actual construction of the heading begins.
%    \begin{macrocode}
    \tpUseProperty{heading-par}%
    \expandafter\ifx\csname tp@parent@#2\endcsname\relax\else
      \edef\@@parent{before-hook-\csname\tp@parent@#2\endcsname}%
      \expandafter\tpUseHook\expandafter{\@@parent}%
    \fi
    \tpUseHook{before-hook-\tp@heading@name}%
    \tpUseProperty{before-heading}%
%    \end{macrocode}
% Add vertical space before the heading
%    \begin{macrocode}
    \tp@do@before@skip
%    \end{macrocode}
% The counters we calculated earlier and the space needed to render
% them are evaluated
%    \begin{macrocode}
    \tp@format@number{}{}{#2}%
%    \end{macrocode}
% The value of after-skip is essential to determine whether the
% heading is to be displayed as block or inline element. In case, some
% heading definition omits setting a proper value, we build a
% fallback.
%    \begin{macrocode}
    \tpIfProp{after-skip}{\expandafter\global\expandafter\@tempskipa\expandafter=\tpUseProperty{after-skip}\relax}{\global\@tempskipa=1sp\relax}%
%    \end{macrocode}
% Labels to be used with LaTeX's cross reference mechanism are defined
%    \begin{macrocode}
      \tp@heading@create@labels{#2}% label facility
    \def\@svsec{%
%    \end{macrocode}
% The \lstinline{heading block} is the composition of all of the
% heading's Components that are to be printed where the
% \lstinline{heading} environment is in the source.
%    \begin{macrocode}
      \tpUseProperty{before-heading-block}%
      \leftskip\tpUseProperty{margin-left}%
      \rightskip\tpUseProperty{margin-right}%
      \bgroup
        \tpUseProperty{heading-block}%
        \tpIfProp{extended}{\tpUseProperty{extended-heading}}{}%
      \egroup%
%    \end{macrocode}
% Generate entries for ToC, bookmarks and page headers.
%    \begin{macrocode}
      \tpIfPropVal{no-toc}{true}{}{\tp@make@toc}% ToC entries
      \tpIfPropVal{no-BM}{true}{}{\tp@make@bookmarks}% Bookmarks
      \tp@make@run% Running headers
      \tpUseProperty{after-heading-block}%
    }%
%    \end{macrocode}
% Finally, we decide whether the printable material we stored in
% \lstinline{\@svsec} is to be rendered as a block or inline. This is
% adopted from \LaTeX's \lstinline{\@startsection}. The distinction is
% made by the sign of \lstinline{after-skip}: a positive value yields
% a block heading, a negative value yields an inline heading.
%    \begin{macrocode}
    \ifdim\@tempskipa <\z@\relax
      \tp@inline@heading
    \else
      \tp@block@heading
    \fi
%    \end{macrocode}
% This macro is called at the end of the heading environment. In order
% to deal with possible vertical spaces after the heading, we wait
% until the group of the heading environemnt is closed before we
% actually print the fully composed heading. The definition of
% \lstinline{\next} happens in either \lstinline{\tp@inline@heading}
% or \lstinline{\tp@block@heading}.
%    \begin{macrocode}
    \aftergroup\next%
  }%
}
%    \end{macrocode}
% \end{macro}
% \end{macro}
% 
% \begin{macro}{\tp@do@before@skip} is a routine that determins the skip that is inserted before a heading.
%    \begin{macrocode}
\def\tp@do@before@skip{%
  \setlength\@tempskipa{\tpUseProperty{before-skip}}%
  \ifdim\@tempskipa<\z@\relax
    \def\do@skip{\minusvspace{-\@tempskipa}}%
  \else
    \def\do@skip{\addvspace{\@tempskipa}}%
  \fi%
  \if@nobreak
    \everypar{}%
    \do@skip
  \else
    \addpenalty\@secpenalty
    \do@skip
  \fi}
%    \end{macrocode}
% \end{macro}
%
% \subsection{Initializers for New Heading Levels}%
%
% \begin{macro}{\tp@init@hooks} initializes the Hooks for heading
%   level \#1.
%    \begin{macrocode}
\def\tp@init@hooks#1{%
  \tpDeclareHook{toc-before-hook-#1}%
  \tpDeclareHook{toc-after-hook-#1}%
  \tpDeclareHook{before-hook-#1}%
}
%    \end{macrocode}
% \end{macro}
% \begin{macro}{\tp@init@cnt} initialises a counter with the name \#1
%   for automatic numbering if it doesn't exist, yet.
%    \begin{macrocode}
\def\tp@init@cnt#1{\tpIfPropVal{numbering}{auto}{\@definecounter{#1}}{}}
%    \end{macrocode}
% \end{macro}
%
% transpect-tex provides a feature that allows headings to be indented
% to the maximum width that is needed to print the counters of
% \textit{all (other) heading levels}. This overall maximum widths are
% stored in the .aux-files for both the headings itself and their
% toc-entries.
%    \begin{macrocode}
\tp@restore@init{heading}{toc-number-maxwd}
\tp@restore@init{heading}{number-maxwd}
%    \end{macrocode}
% \subsection{Initializers for Instances of Heading Levels}%
%
% \begin{macro}{\tp@auto@number}
%   advances the heading counter if the \lstinline{numbering} Property
%   is set to \texttt{auto} and the current heading is not overridden
%   by the \lstinline{Number} Component.  \#1 is the numeric level of
%   the heading, \#2 is the name of the heading's counter.
%    \begin{macrocode}
\def\tp@auto@number#1#2{%
  \tpIfPropVal{numbering}{auto}
    {\expandafter\ifx\csname tp@heading@attr@nonumber\endcsname\@empty
     \else
       \tpIfComp{Number}{}
         {\ifnum #1>\c@secnumdepth\relax\else
            \refstepcounter{#2}%
            \edef\@tempa{\csname the#2\endcsname}%
            \expandafter\tpNumber\expandafter{\@tempa}%
          \fi}%
     \fi}{}%
}
%    \end{macrocode}
% \end{macro}
% \begin{macro}{\tp@heading@create@labels} generates the labels to be used
%   with \LaTeX's cross reference mechanisms. This macro locally
%   redefines LaTeX's \lstinline{\label} macro.
%    \begin{macrocode}
\def\tp@heading@create@labels#1{%
  \ifx\Hy@MakeCurrentHrefAuto\@undefined\else
    \Hy@MakeCurrentHrefAuto{tp.#1}%
    \Hy@raisedlink{\hyper@anchorstart{\@currentHref}\hyper@anchorend}%
  \fi
  \expandafter\ifx\csname tp@heading@attr@label\endcsname\relax\else
    \tpIfComp{Number}{%
      \edef\@tempa{\expandonce{\tp@heading@Number}}%
      \let\@currentlabel\@tempa\relax
      \let\@currentlabelname\tp@heading@Title
      \expandafter\tpltx@label\expandafter{\tp@heading@attr@label}%
    }{}%
  \fi
  \global\let\label\tpltx@label}
%    \end{macrocode}
% \end{macro}
%
%
% \section{Externalisation of Heading Compoents}
%
% Components of headings may be used far away from the heading
% itself. Since, by design, Components are defined strictly local
% within their containers, those externale usages demand special
% treatment.
%
% \subsection{Table of Contents Entry}
%
% \begin{macro}{\tp@make@toc} initializes the creation of a
%   \lstinline{heading} instance's entry in the table of contents.
%
%   Each entry is in itself treated as a Container. As such, it
%   consists of Components that are written into the .toc file.
%    \begin{macrocode}
\def\tp@make@toc{%
  \tp@check@empty{heading}{Title}{Toc}%
  \tp@check@empty{heading}{Number}{Toc}%
  \tp@check@empty{heading}{Subtitle}{Toc}%
  \tp@check@author{Toc}%
  \expandafter\ifx\csname tp@heading@attr@notoc\endcsname\@empty\else
    \protected@edef\tp@heading@toc@entry{%
      \tpIfComp{TocTitle}{\string\tpTocTitle{\string\ignorespaces\space\expandonce{\tp@heading@TocTitle}}}{}%
      \tpIfComp{TocNumber}{\string\tpTocNumber{\string\ignorespaces\space\expandonce{\tp@heading@TocNumber}}}{}%
      \tpIfComp{TocAuthorNameList}{\string\tpTocAuthorNameList{\string\ignorespaces\space\expandonce{\tp@heading@TocAuthorNameList}}}{}%
      \tpIfComp{TocSubtitle}{\string\tpTocSubtitle{\string\ignorespaces\space\expandonce{\tp@heading@TocSubtitle}}}{}%
    }%
    \tpIfProp{toc-level}{\edef\tp@heading@name{\tpUseProperty{toc-level}}}{}%
    \protected@write\@auxout
      {\tpGobble}%
      {\string\@writefile{toc}{\protect\tpContentsline{\tp@heading@name}{\tp@heading@toc@entry}{\thepage}\protected@file@percent}}\relax
  \fi
}
\def\tp@check@author#1{%
  \tpIfComp{#1AuthorNameList}{}{%
    \tpIfComp{AuthorNameList}{%
      \expandafter\csname tp#1AuthorNameList\expandafter\endcsname\expandafter{\tp@heading@AuthorNameList}%
    }{\tpCompDef\tp@tempa{tpAuthor}{author-list-format}%
      \ifx\tp@tempa\relax
      \else
        \expandafter\csname tp#1AuthorNameList\expandafter\endcsname\expandafter{\tp@tempa}%
      \fi
    }%
  }%
}%
%    \end{macrocode}
% \end{macro}
% \begin{macro}{\tpContentsline} is a fork of \LaTeX's \lstinline{\contentsline}, but mine is \lstinline{\long}er.
%    \begin{macrocode}
\long\def\tpContentsline#1{\csname l@#1\endcsname}
%    \end{macrocode}
% \end{macro}
% \begin{macro}{\tp@toc@extract@data} is called within
%   the \lstinline{\l@<level>} macro to extract the Components for
%   each entry in the .toc file. \#1 is the numerical heading level,
%   \#2 is the name of the heading level, \#3 is the content of the
%   toc entry (which holds the Components), \#4 is the page number.
%    \begin{macrocode}
\def\tp@toc@extract@data#1#2#3#4{%
  \tpNamespace{heading}%
  \tpCascadeProps{#2}{heading}%
  \tpDeclareComp{TocPage}{}{}%
  \tpTocPage{\tpUseProperty{toc-page-face}#4}%
  \tpDeclareComp{TocTitle}{}{}%
  \tpDeclareComp{TocSubtitle}{}{}%
  \tpDeclareComp{TocNumber}{}{}%
  \tpDeclareComp{TocAuthorNameList}{}{}%
  \tp@expand@l@contents{#3}{heading}{Toc}{Title}%%
  \tp@format@number{toc-}{Toc}{#1}%
}
%    \end{macrocode}
% \end{macro}
% \begin{macro}{\tp@toc@print@entry} is also called within the
%   \lstinline{\l@<level>} macro and eventually prints the entry by
%   expanding a \lstinline{heading}'s toc-specific Properties.
%    \begin{macrocode}
\def\tp@toc@print@entry#1{%
  \bgroup
    \tpUseHook{toc-before-hook-#1}%
    \tpUseProperty{toc-before-entry}%
    \tpUseProperty{toc-heading}%
    \tpUseHook{toc-after-hook-#1}%
    \tpUseProperty{toc-after-entry}%
  \egroup}
%    \end{macrocode}
% \end{macro}
%
%\subsection{Facility to create the running title macros}
%
% \begin{macro}{\tp@make@run}
%   prepares the Components used to compose the running titles. It
%   checks if the user provides page header specific overrides in the
%   \lstinline{heading} instance. If not, it uses the non-specific
%   Components instead, as long as they are not empty.
%
%   After all the header-specific Components are set, the heading
%   level specific property \lstinline{running-heading} is evaluated
%   and passed to the corresponding \lstinline{\<level>mark} macros
%   iff they exist.
%    \begin{macrocode}
\def\tp@make@run{%
  \tp@check@empty{heading}{Title}{Run}%
  \tp@check@empty{heading}{Number}{Run}%
  %\tp@check@empty{heading}{AuthorNameList}{Run}%
  \tp@check@author{Run}%
  \tp@check@empty{heading}{Subtitle}{Run}%
  \tpUseProperty{running-extra}%
   \tpIfProp{running-level}
    {\expandafter\let\expandafter\tp@mark@name\csname\tpUseProperty{running-level}mark\endcsname}
    {\expandafter\let\expandafter\tp@mark@name\csname\tp@heading@name mark\endcsname}%
    \ifx\tp@mark@name\relax\ifx\tp@heading@parent\@undefined\else
      \expandafter\let\expandafter\tp@mark@name\csname\tp@heading@parent mark\endcsname%
    \fi\fi
  \ifx\tp@mark@name\relax\else
    \begingroup
      \tpGobble
      \expandafter\tp@mark@name\expandafter{\csname tp@heading@running-heading\endcsname}%
    \endgroup
  \fi
}
%    \end{macrocode}
% \end{macro}
%
% \subsection{Facility to create PDF bookmarks}
%
% \begin{macro}{\tp@make@bookmarks} generates an entry that is
%   directly written as Bookmark into the PDF file. This is done using
%   the \lstinline{bookmark} package.
%    \begin{macrocode}
\def\tp@make@bookmarks{%
  \tp@check@empty[Toc]{heading}{Title}{BM}%
  \tp@check@empty[Toc]{heading}{Number}{BM}%
  \tp@check@empty[Toc]{heading}{AuthorNameList}{BM}%
  \tp@check@empty[Toc]{heading}{Subtitle}{BM}%
  \expandafter\ifx\csname tp@heading@attr@noBM\endcsname\@empty\else
    \tpIfProp{bookmark-level}{\edef\Hy@toclevel{\tpUseProperty{bookmark-level}}}{}%
    \begingroup
      \tpGobble
      \protected@edef\@tempa{\csname tp@heading@bookmark\endcsname}%
      \bookmark[level=\Hy@toclevel,dest=\@currentHref]{\expandonce{\@tempa}}%
    \endgroup
  \fi
}
%    \end{macrocode}
% \end{macro}
%
%\section{Rendering the Headings}
%
%\subsection{Inline Headings}
%
% \begin{macro}{\tp@inline@heading}
%   Inline headings are stored in a temporary box and expanded after
%   the next (non-heading) paragraph is opened.
%    \begin{macrocode}
\newbox\tp@inlinesecbox
\def\tp@inline@heading{%
  \tpIfProp{after-indent}{\global\@afterindenttrue}{\global\@afterindentfalse}%
  \tpIfProp{interline-para}
    {\global\setbox\tp@inlinesecbox\hbox{\ifvoid\tp@inlinesecbox\else\unhbox\tp@inlinesecbox\tpUseProperty{interline-para-sep}\fi\@svsec}}%
    {\global\setbox\tp@inlinesecbox\hbox{\@svsec}}
  \@nobreakfalse
  \global\@noskipsectrue
  \gdef\next{%
    \global\everypar{%
      \if@noskipsec
        \global\@noskipsecfalse
        {\setbox\z@\lastbox}%
        \clubpenalty\@M
        \begingroup \unhbox\tp@inlinesecbox \endgroup
        \unskip
        \hskip -\@tempskipa
      \else
        \clubpenalty \@clubpenalty
        \global\setbox\tp@inlinesecbox\box\voidb@x
        \everypar{}%
      \fi}%
    \ignorespaces}}
%    \end{macrocode}
% \end{macro}
%
%\subsection{Block Headings}
%
% \begin{macro}{\tp@block@heading} is used to print block headings.
%    \begin{macrocode}
\def\tp@block@heading{%
  \@svsec
  \par \nobreak
  \tpIfProp{after-indent}{\global\@afterindenttrue}{\global\@afterindentfalse}%
  \gdef\next{%
    \vskip \@tempskipa
    \@afterheading
    \ignorespaces}}
%    \end{macrocode}
% \end{macro}
%
% \section{The \protect\texttt{heading} environment}
%
% \subsection{Environment Macros}
%
% \begin{macro}{\@heading} is the macro called at the begin of the
%   \lstinline{heading} environment. Optional \#1 stores the headings
%   local parameters, \#2 is the level of the heading.
%    \begin{macrocode}
\def\@heading{\@ifnextchar [{\@@heading}{\@@heading[]}}%]
\def\@@heading[#1]#2{%
%    \end{macrocode}
% Some {\LaTeX} kernel macros are saved, the namespace is
% set and counted groups from previous headings are reset.
%    \begin{macrocode}
  \tp@heading@reserve
%    \end{macrocode}
% Handling of the optional argument
%    \begin{macrocode}
  \tpParseAttributes{heading}{#1}%
%    \end{macrocode}
% The mandatory argument contains the heading level. This corresponds
% to \LaTeX's way of counting heading levels, where, by default,
% \lstinline{part} is \lstinline{-1}, \lstinline{chapter} is
% \lstinline{0}, \lstinline{section} is \lstinline{1}, etc.
%    \begin{macrocode}
  \edef\tp@heading@name{#2}%
%    \end{macrocode}
% The cascaded Properties of the heading level are expanded. This is
% excluded into its own macro to simplify re-definition if necessary.
%    \begin{macrocode}
  \tp@heading@load@props%
  \tpEvalType[tpHeading]{Components}%
}
%    \end{macrocode}
% \end{macro}
% \begin{macro}{\endheading}
%   is stuff that happens at the end of the \lstinline{heading}
%   environment.
%    \begin{macrocode}
\def\end@heading{%
  \expandafter\ifx\csname tpUseHeading\tp@heading@name\endcsname\relax
    \PackageError{transpect.cls}{Heading level \tp@heading@name\space unknown!}{A Heading with level \tp@heading@name\space is unknown. Use the \string\tpDeclareHeading\space macro to declare heading levels.}%
  \else
    \csname tpUseHeading\tp@heading@name\endcsname%
  \fi
  \tp@heading@reset
}
\let\heading\tpHeading
\let\endheading\endtpHeading
%    \end{macrocode}
% \end{macro}
%
% \subsection{Content Handlers}
%
% \begin{macro}{\tp@heading@reserve}
%   re-directs some of \LaTeX's kernel macros and makes sure that some
%   other macros have their default values:
%    \begin{macrocode}
\def\tp@heading@reserve{%
  \tpNamespace{heading}%
  \let\tpltx@dbl@backslash\\
  \let\\\tpBreak
  \let\tpltx@label\label
  \let\tp@heading@label\relax
  \csname tpAuthor-Cnt\endcsname\z@
  \tp@reset@components{\tp@cur@cont}%
  }
%    \end{macrocode}
% \end{macro}
% \begin{macro} restores \LaTeX's default definitions (however, this
%   should be unnecessary since \lstinline{heading} is an environment
%   and therefore constitutes a closed group).
%    \begin{macrocode}
\def\tp@heading@reset{%
  \let\tp@cur@cont\relax
  \let\\\tpltx@dbl@backslash
  \let\label\tpltx@label
  \let\tp@heading@name\relax
  \let\tp@heading@label\relax
  }
%    \end{macrocode}
% \end{macro}
%
% \begin{macro}{\tp@heading@load@props} expands the level specific
%   Property list.
%    \begin{macrocode}
\def\tp@heading@load@props{\csname tp@heading@\tp@heading@name @properties\endcsname}
%    \end{macrocode}
% \end{macro}
% \Deprecated The next two macros should go to the Meta Container.
% \begin{macro}{\tp@extended@ht@macros}
%   provides some extended markup for certain headings levels, mostly
%   used for compilations of contributions by different authors like
%   collections, proceedings, or journals.
%    \begin{macrocode}
\def\tp@extended@ht@macros{%
  \tp@provide@ext@hd@macros{Abstract}%
  \tp@provide@ext@hd@macros{Keywords}%
  \tp@provide@ext@hd@macros{DOI}%
}
%    \end{macrocode}
% \end{macro}
% \begin{macro}{\tp@provide@ext@hd@macros}
%   Provides component macros to encode meta data. Each macro consists
%   of two parts: one that carries the content, and one that allows to
%   override the Title.
%    \begin{macrocode}
\def\tp@provide@ext@hd@macros#1{%
  \tpDeclareComp{#1}{}{}%
  \tpDeclareComp{#1Title}{}{}%
}
%    \end{macrocode}
% \end{macro}
% \begin{macro}{\tp@heading@role@handlers} sets up the additional Components for the Author Role specific to headings.
%    \begin{macrocode}
\def\tp@heading@authors{%
  % \tpAddToRole{Author}{%
  %   \tpDeclareCountedComp{AuthorContact}%
  % }%
  %\tpDeclareRoleBlock{Author}{ContactBlock}{author-contact-block-format}%
  % \tpGroupHandler{tpAuthor}{%
  %   \tpIfComp{AuthorContact}{}{\csname tpAuthorContact\endcsname{\tpUseProperty{author-contact-format}}}{}}%
  \tp@provide@hd@overrides{AuthorNameList}%
}
%    \end{macrocode}
% 
% \end{macro}
% \begin{macro}{\tp@provide@hd@macros} is a wrapper that creates the
%   user-level macros for the Component itself and its overrides. \#1
%   is the Component name.
%    \begin{macrocode}
\def\tp@provide@hd@macros#1{%
  \tpDeclareComp{#1}{}{}%
  \tp@provide@hd@overrides{#1}%
}
%    \end{macrocode}
% \end{macro}
% \begin{macro}{\tp@provide@hd@overrides} declares the Component
%   macros for a heading Component's overrides. \#1 is the Component
%   name. The overrides allow a four-way distinction between
%   \textit{i} the data printed in-situ (\lstinline{\tp#1}), \textit{ii} data sent
%   to toc (\lstinline{\tpToc#1}), (iii) data sent to the page styles (\lstinline{\tpRun#1}), and (iv) the data sent
%   to the PDF bookmarks (\lstinline{\tpBM#1}).
%    \begin{macrocode}
\def\tp@provide@hd@overrides#1{%
  \tpDeclareComp{Toc#1}{}{}% toc overrides
  \tpDeclareComp{Run#1}{}{}% running overrides
  \tpDeclareComp{BM#1}{}{}% bookmark overrides
}
%    \end{macrocode}
% \end{macro}
%
% \section{Defaults}
%
% \begin{macro}{\tp@heading@default} are the global default Property
%   definitions for all headings.
%    \begin{macrocode}
\tpAddToDefault{heading}{%
  \tpSetProperty{interline-para}{}%
  \tpSetProperty{interline-para-sep}{\space}
  \tpSetProperty{heading-par}{%
    \tpIfProp{interline-para}{\if@noskipsec \leavevmode \fi}{}%
    \par
    \global\@afterindenttrue
  }%
  \tpSetProperty{before-heading}{}%
  \tpSetProperty{title-face}{\bfseries}%
  \tpSetProperty{subtitle-face}{\normalfont}%
  \tpSetProperty{author-face}{\normalfont}%
  \tpSetProperty{quote-face}{\raggedleft}%
  \tpSetProperty{quote-source-face}{}%
  \tpSetProperty{heading-block}
    {\tpUseProperty{title-face}%
     \tpIfComp{Number}
       {\tpUseProperty{hang-number}}
       {\leftskip0pt}%
     {\tpUseComp{Title}}\par%
     \tpIfComp{Subtitle}{{\tpUseProperty{subtitle-face}\tpUseComp{Subtitle}}\par}{}%
     \tpIfComp{AuthorNameList}{{\tpUseProperty{author-face}\tpUseComp{AuthorNameList}}\par}{}%
     \tpIfComp{Quote}{%
       \bgroup
         \tpUseProperty{quote-face}%
         \tpUseComp{Quote}\par
         \tpIfComp{QuoteSource}{{\tpUseProperty{quote-source-face}--\space\tpUseComp{Quote}}\par}{}%
       \egroup}{}%
    }%
  \tpSetProperty{extended-heading}{%
    \tpIfComp{Abstract}
      {\par\vskip\baselineskip
       {\bfseries\tpIfComp{AbstractTitle}{\tpUseComp{AbstractTitle}}{Abstract}}\par
       {\itshape\small\tpUseComp{Abstract}}\par}
      {}%
    \tpIfComp{Keywords}
      {\par\vskip\baselineskip
       {\bfseries\tpIfComp{KeywordsTtitle}{\tpUseComp{KeywordsTitle}}{Keywords}}\par
       {\itshape\small\tpUseComp{Keywords}\par}}
     {}%
   }%
  \tpSetProperty{before-skip}{\z@skip}% TODOC: values < 0pt use \minusvspace, else \addvspace. LaTeX's default behaviour of @afterindent is relocated to the after-indent property.
  \tpSetProperty{after-heading-block}{}%
  \tpSetProperty{before-heading-block}{\parindent\z@ \parskip\z@}%
  \tpSetProperty{after-indent}{}%
  \tpSetProperty{margin-left}{}%
  \tpSetProperty{margin-right}{\@flushglue}%
  \tpSetProperty{after-skip}{1sp}%
  \tpSetProperty{indent}{auto}%
  \tpSetProperty{number-width}{}%
  \tpSetProperty{number-sep}{\space}%
  \tpSetProperty{number-align}{left}%
  \tpSetProperty{numbering}{auto}%
  %% running header
  \tpSetProperty{running-level}{}% override level for running title, name
  \tpSetProperty{running-heading}{%
    \tpIfComp{RunAuthorNameList}{\tpUseComp{RunAuthorNameList}:\space}{}%
    \tpUseComp{RunTitle}%
  }%
  %% ToC
  \tpSetProperty{no-toc}{false}% toc entries are generally disabled iff true
  \tpSetProperty{no-BM}{false}% bookmark entries are generally disabled, iff true
  \tpSetProperty{toc-page-sep}{\dotfill}% between toc-title and page
  \tpSetProperty{toc-page-face}{}%
  \tpSetProperty{toc-number-width}{}% current width of number
  \tpSetProperty{toc-number-align}{left}% alignment of number within hbox when hanging
  \tpSetProperty{toc-title-face}{}% face of title
  \tpPropertyLet{toc-number-face}{toc-title-face}% format of number
  \tpSetProperty{toc-number-sep}{\enskip}% thing between numebr and toc-title
  \tpSetProperty{toc-margin-top}{\z@}% left indent of the whole entry
  \tpSetProperty{toc-margin-bottom}{\z@}% right margin of the whole entry
  \tpSetProperty{toc-margin-left}{auto}% left indent of the whole entry
  \tpSetProperty{toc-margin-right}{\@pnumwidth}% right margin of the whole entry
  \tpSetProperty{toc-link}{none}% should toc entries be linked? values: none,title,page,all
  \tpSetProperty{toc-level}{}% override heading level for ToC, name!
  \tpSetProperty{toc-indent}{auto}% offset of the first line of the entry. auto: hang indent by max-number-width for the level
  \tpSetProperty{toc-before-entry}{% stuff before anything is output; used to setup margins, alignment, line-breaking rules, etc.
    \addvspace{\tpUseProperty{toc-margin-top}}%
    \parindent \z@
    \let\\\@centercr
    \hyphenpenalty=\@M
    \rightskip \tpUseProperty{toc-margin-right} \@plus 1fil\relax
    \parfillskip -\rightskip
    \leftskip\tpUseProperty{toc-margin-left}%
  }%
  \tpSetProperty{toc-after-entry}{\par\addvspace{\tpUseProperty{toc-margin-bottom}}}% Thing at the end of the entry, after the page number
  \tpSetProperty{toc-heading}{% Order and formatting of the entry itself
    \tpUseProperty{toc-title-face}%
    \tpIfComp{TocNumber}
      {\tpUseProperty{toc-hang-number}}
      {\leftskip0pt\leavevmode}%
    \tpIfComp{TocAuthorNameList}{\tpUseComp{TocAuthorNameList}:\space}{}%
    \tpUseComp{TocTitle}%
    \tpUseProperty{toc-page-sep}\tpUseComp{TocPage}%
  }%
  %% PDF-Bookmarks
  \tpSetProperty{bookmark-level}{}% override heading level for PDF bookmarks, numeric!
  \tpSetProperty{bookmark}{%
    \tpIfComp{BMNumber}{\tpUseComp{BMNumber}\space}{}%
    \tpUseComp{BMTitle}%
  }%
  \tpSetProperty{orcid-link}{% how the ORC-ID is rendered
    \tpIfComp{ORCID}{\tpCompLink{ORCID}{\includegraphics[height=1ex]{logos/ORCID_iD.pdf}}}{}%
  }%
  %% a single Author's contact infomration block
  \tpSetProperty{author-contact-format}{%Format of a single author's contact information
    \tpUseComp{FullName}\tpIfComp{Affil}{\textsuperscript{\tpUseComp{Affil}}}{}%
    \tpUseProperty{orcid-link}%
    %
  }%
  \tpSetProperty{author-list-format}{% Format of the whole contact information block
    \tpUseComp{FullName}\ifnum\tpCurCount<\tpTotalCount\tpUseProperty{counted-name-sep}\fi
  }%
}
%    \end{macrocode}
% \end{macro}
%
% \section{Miscellaneous}
%
% \subsection{Alternative paragraph separation}
%
% \begin{macro}{\tpNewPar} is a user-level macro to have a vertical
%   skip between two local paragraphs and no indent in the second
%   one. The amount of vertical space between the paragraphs can be
%   adjusted with the optional argument. If \#1 is omitted,
%   \lstinline{\tpnewparskip} is inserted, which defaults to
%   \lstinline{1\baselineskip} if the dimension isn't set to something
%   other than 0pt in the preamble. This macro is intended to be used
%   at the end of the first of the paragraphs.
%    \begin{macrocode}
\newdimen\tpnewparskip \AtBeginDocument{\ifdim\tpnewparskip=\z@\relax \tpnewparskip=1\baselineskip\relax\fi}
\def\tpNewPar{\@ifnextchar[{\@tpnewpar}{\@tpnewpar[\the\tpnewparskip]}}%]
\def\@tpnewpar[#1]{%
  \ifhmode\par\fi
  \vskip#1\relax
  \@afterheading
}
%    \end{macrocode}
% \end{macro}
% \Deprecated
% \begin{macro}{\TitleBreak}
%    \begin{macrocode}
\let\TitleBreak\tpBreak
%    \end{macrocode}
% \end{macro}
%    \begin{macrocode}[gobble=1]
%</headings>
%    \end{macrocode}
