%    \end{macrocode}
% \chapter{transpect-headings.dtx}
% This module provides handlers for headings like parts, chapters,
% sections, or inline headings common to all Transpect projects
%
%    \begin{macrocode}
%%
%% module for le-tex transpect.cls that extends heading objects.
%%
%% Maintainer: p.schulz@le-tex.de
%%
%% lualatex  -  texlive >= 2019
%%
\NeedsTeXFormat{LaTeX2e}[2019/01/01]
\ProvidesPackage{transpect-headings}
    [2020/18/05 0.90 le-tex transpect headings module]
%    \end{macrocode}
%
% Headings are handled differently with \lstinline{transpect.cls}
% compared to standard \LaTeX, since transpect manuscripts tend to
% have a whole collection of additional information that are pressed
% into the headings, like subtitles or section authors down to
% subsection level, etc. Therefore, the \lstinline{\@startsection} and
% \lstinline{\@make[s]chapterhead} facilities from {\LaTeX} are no longer
% sufficient. At the same time, the package does not redefine those
% macros and keeps them available for backwards compatibility.
%
% \section{Blocks}
% Blocks are bundled units of heading elements. They provide three
% hooks: \lstinline{Before<Name><Level>},
% \lstinline{After<Name><Level>}, and
% \lstinline{<Name>Format<Level>}. The arguments are \#1:
% \lstinline{<Name>}, \#2 \lstinline{<Level>}.
%    \begin{macrocode}
\def\tpDeclareBlock#1#2{%
  \tpDeclareHook{Before#1#2}%
  \tpDeclareHook{#1Format#2}%
  \tpDeclareHook{After#1#2}%
}
%    \end{macrocode}
% The use-call function for Blocks
%    \begin{macrocode}
\def\tpUseHeadingBlock#1#2{%
  \expandafter\ifx\csname heading@#1\endcsname\relax\else
    \tpUseHook{Before#1#2}%
    {\tpUseHook{#1Format#2}\csname heading@#1\endcsname}%
    \tpUseHook{After#1#2}%
  \fi}
%    \end{macrocode}
%
% \section{Facility for declaring heading levels and their layouts}%
% \begin{description}
% \item[1] level: used for toc entries. -1 for part, 0 for chapter, 1 for section, etc.
% \item[2] name: part, chapter, section, etc, to be used in toc, head lines, bookmarks, etc.
% \item[3] Property definitions and switches
% \end{description}
%    \begin{macrocode}
\def\tpDeclareHeading#1#2#3{%
  \expandafter\def\csname heading@#2@name\endcsname{#2}%
  \def\tpHeadingProperty##1##2{\tpAddToHook{##1#2}{##2}}%
%    \end{macrocode}
% \subsection{Block hooks}
% Formatting and upper/lower boundaries of the author's names
%    \begin{macrocode}
  \tpDeclareBlock{Author}{#2}%
%    \end{macrocode}
% Formatting and upper/lower boundaries of the subtitle
%    \begin{macrocode}
  \tpDeclareBlock{Subtitle}{#2}%
%    \end{macrocode}
% Formatting and upper/lower boundaries of mottos/quotes/quotations
%    \begin{macrocode}
  \tpDeclareBlock{Quote}{#2}%
%    \end{macrocode}
% Formatting and upper/lower boundaries of heading counters. Thise are formatted as the actual title, bzt can be overridden with \lstinline{\NumberFormat}.
%    \begin{macrocode}
  \tpDeclareBlock{Number}{#2}%
%    \end{macrocode}
%
% \subsection{Heading-wide hooks}
% Block format for the whole title, used for properties that afffect the whole heading, like horizontal alignment
%    \begin{macrocode}
  \tpDeclareHook{BlockFormat#2}%
%    \end{macrocode}
% Code that is applied at the very beginning of the heading, usually contains a \lstinline{\clearpage} for the lower levels, but also preceeding skips
%    \begin{macrocode}
  \tpDeclareHook{BeforeHeading#2}%
%    \end{macrocode}
% Code that is applied at the end of the heading (but before \lstinline{\@afterheading}). This does NOT contain after-heading skips.
%    \begin{macrocode}
  \tpDeclareHook{AfterHeading#2}%
%    \end{macrocode}
% After heading skip. This is used to separate inline headings (value
% < 0pt) from free-floating headings (else), compare
% \lstinline{\@startsection}'s fifth argument.
%    \begin{macrocode}
  \tpDeclareHook{AfterSkip#2}%
%    \end{macrocode}
% format of the actual heading title
%    \begin{macrocode}
  \tpDeclareHook{TitleFormat#2}%
%    \end{macrocode}
% This switch provides a way to put the author's name(s) after the
% heading. Default is that the author's are put on top of the heading.
%    \begin{macrocode}
  \def\tpAuthorAfter{\expandafter\global\expandafter\let\csname tp@author@fter@#2\endcsname\@empty}%
%    \end{macrocode}
% Facility for hanging indents in headings.
%
% \lstinline{\tpHangNumber} triggers hanging indent by the total width
% of the number block.
%    \begin{macrocode}
  \def\tpHangNumber{\expandafter\global\expandafter\let\csname tp@hang@number@#2\endcsname\@empty}%
%    \end{macrocode}
% \lstinline{\tpHangNumber} can be used when the hanging indent should
% always be of a fixed width. If ommitted, the hanging indent is
% determined by the number's overall width for each heading separately.
%    \begin{macrocode}
  \def\tpHangFixed##1{\expandafter\gdef\csname tp@hang@fixed@#2\endcsname{##1}}%
  \tpHangFixed{\z@}%
%    \end{macrocode}
% Evaluation of user's overrides
%    \begin{macrocode}
  \tp@init@l@{#2}%
  #3
%    \end{macrocode}
% Once a heading level is declared, we can now provide the facilities
% to output the headings. This code largely resembles \LaTeX's
% \lstinline{\@startsection} mechanism, albeit with some tweaks.
% \subsection{Providing the actual heading macro}
%    \begin{macrocode}
  \expandafter\def\csname tpUseHeading#2\endcsname{%
    \xdef\heading@level{#1}%
    \edef\Hy@toclevel{\csname toclevel@#1\endcsname}%
    \if@noskipsec \leavevmode \fi
    \par\tpUseHook{BeforeHeading#2}%
    \@afterindenttrue
    \everypar{}%
    \def\@svsec{%
      \bgroup
        \parindent\z@ \parskip\z@
        \tpUseHook{BlockFormat#2}%
        \expandafter\ifx\csname tp@author@fter@#2\endcsname\relax
          \tpUseHeadingBlock{Author}{#2}%
        \fi
        \expandafter\ifx\csname tp@hang@number@#2\endcsname\@empty
          \expandafter\@tempdimb\dimexpr\csname tp@hang@fixed@#2\endcsname\relax
          \ifdim\@tempdimb>\z@\relax\else
            \setbox\z@\hbox{\tpUseHook{TitleFormat#2}\tpUseHeadingBlock{Number}{#2}\strut}%
            \@tempdimb\wd\z@
          \fi
          \expandafter\@tempdima\dimexpr\hsize-\@tempdimb\relax
          \hskip\@tempdimb\vbox\bgroup
            \hsize\@tempdima\relax
            \hskip-\@tempdimb
        \fi
        {\tpUseHook{TitleFormat#2}%
         \tpUseHeadingBlock{Number}{#2}%
         \heading@Title}%
        \tpUseHeadingBlock{Subtitle}{#2}%
        \expandafter\ifx\csname tp@author@fter@#2\endcsname\relax\else
          \tpUseHeadingBlock{Author}{#2}%
        \fi
        \expandafter\ifx\csname tp@hang@number@#2\endcsname\@empty
          \vss\egroup
        \fi
        \tpUseHeadingBlock{Quote}{#2}%
%    \end{macrocode}
% Since all counters are taken from Word input, we can safely override \lstinline{\the<name>}:
%    \begin{macrocode}
        \ifx\heading@Number\relax
          \expandafter\global\expandafter\let\csname the#2\endcsname\@empty%
        \else
          \expandafter\xdef\csname the#2\endcsname{\heading@Number}%
        \fi
%    \end{macrocode}
% catch \LaTeX's referencing facility with some magick for \lstinline{hyperref}:
%    \begin{macrocode}
      \ifx\Hy@MakeCurrentHrefAuto\@undefined\else
        \Hy@MakeCurrentHrefAuto{#2}%
        \Hy@raisedlink{\hyper@anchorstart{\@currentHref}\hyper@anchorend}%
      \fi
      \ifx\tp@label\relax\else
        \ifx\heading@Number\relax
        \else
          \let\@currentlabel\heading@Number
          \let\@currentlabelname\heading@Title
          \expandafter\ltx@label\expandafter{\tp@label}%
        \fi
      \fi
      \global\let\label\ltx@label
%    \end{macrocode}
% Entry in table of contents
%    \begin{macrocode}
        \expandafter\ifnum\heading@level<\c@tocdepth\relax
          \ifx\heading@Toctitle\relax
            \ifx\heading@Number\relax
              \addcontentsline{toc}{#2}{\heading@Title}\relax
            \else
              \addcontentsline{toc}{#2}{\protect\numberline{\heading@Number}\heading@Title}\relax
            \fi
          \else
            \addcontentsline{toc}{#2}{\heading@Toctitle}\relax
          \fi
        \fi
%    \end{macrocode}
% Running headers
%    \begin{macrocode}
        \ifx\heading@Runtitle\relax
          \ifx\heading@Number\relax
            \csname #2mark\endcsname{\heading@Title}%
          \else
%    \end{macrocode}
% note that \lstinline{\chaptermark} and \lstinline{\sectionmark} need
% to be dealt with in your page style definitions or you might get
% faulty \lstinline{\thechapter} and \lstinline{\thesection} readings!
%    \begin{macrocode}
            \csname #2mark\endcsname{\numberline{\heading@Number}\heading@Title}%
          \fi
        \else
          \csname #2mark\endcsname{\heading@Runtitle}%
        \fi
      \egroup
      \tpUseHook{AfterHeading#2}%
    }%
%    \end{macrocode}
% Skip after the heading, taken partially from the definition of
% \LaTeX's \lstinline{\@xsect} macro:
%    \begin{macrocode}
    \expandafter\ifx\csname tp@hook@AfterSkip#2\endcsname\@empty
      \global\@tempskipa=1sp\relax
    \else
      \expandafter\expandafter\expandafter\global\expandafter\expandafter\@tempskipa\expandafter=\csname tp@hook@AfterSkip#2\endcsname\relax%
    \fi
    \ifdim\@tempskipa <\z@\relax
      \global\@afterindentfalse
      \expandafter\gdef\expandafter\@svsechd\expandafter{\@svsec}%
      \@nobreakfalse
      \global\@noskipsectrue
      \global\everypar{%
        \if@noskipsec
          \global\@noskipsecfalse
          {\setbox\z@\lastbox}%
          \clubpenalty\@M
          \begingroup \@svsechd \endgroup
          \unskip
          \hskip -\@tempskipa
        \else
          \clubpenalty \@clubpenalty
          \everypar{}%
        \fi}%
      \ignorespaces
    \else
      \@svsec
      \par \nobreak
      \vskip \@tempskipa
      \global\@afterindentfalse
      \aftergroup\@afterheading%
    \fi}}

%    \end{macrocode}
%
% \section{The headings environment}
%
%    \begin{macrocode}
\def\heading{\@ifnextchar [{\@heading}{\@heading[]}}%]

\def\@heading[#1]#2{%
  \global\let\heading@Author\relax
  \global\let\heading@Title\relax
  \global\let\heading@Number\relax
  \global\let\heading@Subtitle\relax
  \global\let\heading@Toctitle\relax
  \global\let\heading@Runtitle\relax
  \global\let\heading@Quote\relax
  \global\let\@optarg\relax
  \global\let\ltx@label\label
  \global\let\tp@label\relax
  \bgroup
%    \end{macrocode}
% handling of the optional argument
%    \begin{macrocode}
    \xdef\@optarg{#1}%
%    \end{macrocode}
% The mandatory argument contains the section level, this corresponds
% to \LaTeX's way of counting where part is -1, chapter is 0, section
% is 1, etc.
%    \begin{macrocode}
    \xdef\heading@name{#2}%
    \def\label##1{\gdef\tp@label{##1}}%
    \def\author##1{\gdef\heading@Author{##1}}%
    \def\title##1{\gdef\heading@Title{##1}}%
    \def\subtitle##1{\gdef\heading@Subtitle{##1}}%
    \def\toctitle##1{\gdef\heading@Toctitle{##1}}%
    \def\runtitle##1{\gdef\heading@Runtitle{##1}}%
    \def\number##1{\gdef\heading@Number{##1}}%
    \long\def\quote##1{\long\gdef\heading@Quote{##1}}%
}
%    \end{macrocode}
% The ending part of the heading environment.
%    \begin{macrocode}
\def\endheading{%
  \egroup
  \expandafter\ifx\csname tpUseHeading\heading@name\endcsname\relax
    \PackageError{transpect.cls}{Heading level \heading@name\space unknown!}{A Heading with level \heading@name\space is unknown. Use the \string\tpDeclareHeading\space macro to declare heading levels.}%
  \fi
  \csname tpUseHeading\heading@name\endcsname}
%    \end{macrocode}
%
% \section{Facility for ToC and ListOf entries}
%    \begin{macrocode}
\def\tp@init@l@#1{%
  \def\tpToCProperty##1##2{\tpAddToHook{##1#1}{##2}}%
%    \end{macrocode}
% Code to be inserted before the toc entry for that level
%    \begin{macrocode}
  \def\tpToCBefore##1{\expandafter\gdef\csname tp@toc@before@#1\endcsname{##1}}%\vskip 1.0em \@plus\p@
%    \end{macrocode}
% Code to be inserted after the toc entry for that level
%    \begin{macrocode}
  \def\tpToCAfter##1{\expandafter\def\csname tp@toc@after@#1\endcsname{##1}}%
%    \end{macrocode}
% If used, create a hanging indent the width of the heading's number
%    \begin{macrocode}
  \def\tpToCHangNumber{\expandafter\global\expandafter\let\csname tp@toc@hangnum@#1\endcsname\@empty}%
%    \end{macrocode}
% If greater than 0pt, use hanging indent of that width, needs to be defined as a dimension.
%    \begin{macrocode}
  \def\tpToCHangFixed##1{\expandafter\xdef\csname tp@toc@hangfixed@#1\endcsname{\dimexpr##1\relax}}%
  \expandafter\let\csname tp@toc@hangfixed@#1\endcsname\z@
%    \end{macrocode}
% Separator between the heading number and the heading title
%    \begin{macrocode}
  \def\tpToCNumSep##1{\expandafter\gdef\csname tp@toc@numsep@#1\endcsname{##1}}%\enskip
%    \end{macrocode}
% Separator between the heading title and the page number
%    \begin{macrocode}
  \def\tpToCPageSep##1{\expandafter\gdef\csname tp@toc@pagesep@#1\endcsname{##1}}%\dotfill
%    \end{macrocode}
% Format of the page number; extends \lstinline{\toTocFormat<level>}.
%    \begin{macrocode}
  \def\tpToCPageFormat##1{\expandafter\gdef\csname tp@toc@pageformat@#1\endcsname{##1}}%
%    \end{macrocode}
% Format of the entire toc entry, i.e., number, title, and page.
%    \begin{macrocode}
  \def\tpToCFormat##1{\expandafter\def\csname tp@toc@format@#1\endcsname{##1}}%\bfseries
%    \end{macrocode}
% Generically provide \LaTeX's internal toc entry formatting macros:
%    \begin{macrocode}
  \expandafter\def\csname l@#1\endcsname##1##2{%
    \addpenalty{-\@highpenalty}%
    \csname tp@toc@before@#1\endcsname
    \expandafter\ifx\csname tp@toc@hangnum@#1\endcsname\@empty
      \expandafter\@tempdima\csname tp@toc@hangfixed@#1\endcsname
    \else
      \def\numberline####1{%
        \setbox\@tempboxa\hbox{####1\csname tp@toc@numsep@#1\endcsname}}%
      \sbox\z@{##1}%
      \expandafter\@tempdima\wd\@tempboxa%
    \fi
    \def\numberline####1{####1\csname tp@toc@numsep@#1\endcsname}%
    \begingroup
      \parindent \z@ \rightskip \@pnumwidth
      \parfillskip -\@pnumwidth
      \leavevmode
      \csname tp@toc@format@#1\endcsname
      \advance\leftskip\@tempdima
      \hskip -\@tempdima
      ##1\nobreak
      \csname tp@toc@pagesep@#1\endcsname
      \nobreak\hb@xt@\@pnumwidth{\hss\csname tp@toc@pageformat@#1\endcsname ##2%
        \kern-\p@\kern\p@}\par
      \penalty\@highpenalty
      \csname tp@toc@after@#1\endcsname
    \endgroup
  }%
}
%    \end{macrocode}
%
% \section{Defaults}
%    \begin{macrocode}
\def\partmark#1{}%
\tpDeclareHeading{-1}{part}{%
  \tpHeadingProperty{BlockFormat}{\centering
    \markboth{}{}%
    \thispagestyle{empty}%
  }%
  \tpHeadingProperty{BeforeHeading}{\cleardoublepage\null\vskip10mm}%
  \tpHeadingProperty{AfterHeading}{\vfill}%
  \tpHeadingProperty{AuthorFormat}{\itshape}%
  \tpHeadingProperty{AfterAuthor}{\@@par\vskip2\baselineskip}%
  \tpHeadingProperty{TitleFormat}{\Huge}%
  \tpHeadingProperty{BeforeSubtitle}{\@@par\vskip\baselineskip}%
  \tpHeadingProperty{SubtitleFormat}{\leavevmode\vskip\baselineskip\large\bfseries}%
  \tpHeadingProperty{AfterNumber}{:\@@par}%
  \tpToCBefore{\vskip2\baselineskip}%
  \tpToCFormat{\bfseries}%
  \tpToCPageSep{\hfill}%
}

\tpDeclareHeading{0}{chapter}{%
  \tpHeadingProperty{BlockFormat}{\centering
    \markboth{}{}%
    \thispagestyle{empty}%
  }%
  \tpHeadingProperty{BeforeHeading}{\cleardoublepage\null\vskip10mm}%
  \tpHeadingProperty{AfterSkip}{3\baselineskip}%
  \tpHeadingProperty{AuthorFormat}{\itshape}%
  \tpHeadingProperty{AfterAuthor}{\@@par\vskip\baselineskip}%
  \tpHeadingProperty{TitleFormat}{\LARGE}%
  \tpHeadingProperty{SubtitleFormat}{\leavevmode\vskip\baselineskip\Large\bfseries}%
  \tpHeadingProperty{BeforeSubtitle}{\@@par\vskip.5\baselineskip}%
  \tpHeadingProperty{BeforeQuote}{\strut\@@par\vskip\baselineskip\hfil\hbox\bgroup\centering\vbox\bgroup\hsize.5\textwidth\centering}%
  \tpHeadingProperty{AfterQuote}{\egroup\egroup}%
  \tpHeadingProperty{AfterNumber}{:\enskip}%
  \tpToCBefore{\vskip1\baselineskip}%
  \tpToCFormat{\bfseries}%
  \tpToCPageSep{\bfseries\dotfill}%
  %% 
  \tpHeadingProperty{AfterNumber}{:\enskip}%
}

\tpDeclareHeading{1}{section}{%
  \tpAuthorAfter
  \tpHangNumber
  %\tpHangFixed{12mm}%
  \tpHeadingProperty{BeforeHeading}{\vskip2\baselineskip}%
  \tpHeadingProperty{AfterSkip}{1\baselineskip}%
  \tpHeadingProperty{BeforeSubtitle}{ -- }%
  \tpHeadingProperty{BeforeAuthor}{\@@par}%
  \tpHeadingProperty{AuthorFormat}{\itshape}%
  \tpHeadingProperty{AfterAuthor}{\@@par}%
  \tpHeadingProperty{TitleFormat}{\Large}%
  \tpHeadingProperty{SubtitleFormat}{\Large}%
  \tpHeadingProperty{BeforeQuote}{\@@par\vskip.5\baselineskip}%
  \tpHeadingProperty{AfterNumber}{:\enskip}%
  % \tpHeadingProperty{AfterQuote}{\@@par\vskip.5\baselineskip}%
}

\tpDeclareHeading{2}{subsection}{%
  \tpHeadingProperty{BeforeHeading}{\vskip1.5\baselineskip}%
  \tpHeadingProperty{AfterSkip}{0.5\baselineskip}%
  \tpHeadingProperty{BeforeSubtitle}{ -- }%
  \tpHeadingProperty{AfterAuthor}{\@@par}%
  \tpHeadingProperty{TitleFormat}{\large}%
  \tpHeadingProperty{SubtitleFormat}{}%
  \tpHeadingProperty{BeforeQuote}{\@@par\vskip.5\baselineskip}%
  \tpHeadingProperty{AfterNumber}{:\enskip}%
}


\tpDeclareHeading{3}{subsubsection}{}

\tpDeclareHeading{4}{paragraph}{%
  \tpHeadingProperty{BeforeHeading}{\vskip1\baselineskip \@minus5bp}%
  \tpHeadingProperty{AfterSkip}{-.5em}%
  \tpHeadingProperty{BeforeSubtitle}{ -- }%
  \tpHeadingProperty{AfterAuthor}{\@@par}%
  \tpHeadingProperty{TitleFormat}{\normalsize\bfseries}%
  \tpHeadingProperty{SubtitleFormat}{}%
  \tpHeadingProperty{AfterNumber}{:\enskip}%
}

\let\ltx@ps@headings\ps@headings
\def\ps@headings{%
  \ltx@ps@headings
  \def\chaptermark##1{%
    \def\numberline####1{####1\enskip}%
    \markboth{%
      ##1
    }{}}%
  \def\sectionmark##1{%
    \def\numberline####1{####1\enskip}%
    \markright{##1}}%
}

%    \end{macrocode}
%
% \section{ToC entries}
%    \begin{macrocode}
% \def\l@part#1#2{%
%   \ifnum \c@tocdepth >-2\relax
%     \addpenalty{-\@highpenalty}%
%     \addvspace{2.25em \@plus\p@}%
%     \setlength\@tempdima{3em}%
%     \begingroup
%       \parindent \z@ \rightskip \@pnumwidth
%       \parfillskip -\@pnumwidth
%       {\leavevmode
%        \large \bfseries #1\hfil
%        \hb@xt@\@pnumwidth{\hss #2%
%                           \kern-\p@\kern\p@}}\par
%        \nobreak
%          \global\@nobreaktrue
%          \everypar{\global\@nobreakfalse\everypar{}}%
%     \endgroup
%   \fi}



% \renewcommand*\l@section{\@dottedtocline{1}{1.5em}{2.3em}}
% \renewcommand*\l@subsection{\@dottedtocline{2}{3.8em}{3.2em}}
% \renewcommand*\l@subsubsection{\@dottedtocline{3}{7.0em}{4.1em}}
% \renewcommand*\l@paragraph{\@dottedtocline{4}{10em}{5em}}
% \renewcommand*\l@subparagraph{\@dottedtocline{5}{12em}{6em}}
