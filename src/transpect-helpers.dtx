%    \end{macrocode}
% \chapter{transpect-helpers.dtx}
% This file provides some basic macros and facilities like macro
% hooks.
%
%    \begin{macrocode}
%%
%% module for le-tex transpect.cls that provides some commonly used base macros.
%%
%% Maintainer: p.schulz@le-tex.de
%%
%% lualatex  -  texlive > 2019
%%
\NeedsTeXFormat{LaTeX2e}[2019/01/01]
\ProvidesPackage{transpect-helpers}
    [2020/22/04 0.90 le-tex transpect helpers module]
%    \end{macrocode}
% prefix macro used to execute code after the next \lstinline{\fi}:
%    \begin{macrocode}
\def\afterfi#1\fi{\fi#1}
%    \end{macrocode}
% \lstinline{\afterbox} prevents indentation and additional spacing after
% environments. Intended to be used in combination with
% \lstinline{\aftergroup}.
%    \begin{macrocode}
\def\@afterbox{%
  \everypar{%
    \if@nobreak
      \@nobreakfalse
      \clubpenalty \@M
      \if@afterindent \else
        {\setbox\z@\lastbox}%
        \everypar{}%
      \fi
    \else
      \clubpenalty \@clubpenalty
      {\setbox\z@\lastbox}%
      \everypar{}%
    \fi}}
%    \end{macrocode}
% This macro is used to calculate the ratio between two integers.
%    \begin{macrocode}
\def\CalcRatio#1#2{\strip@pt\dimexpr\number\numexpr\number\dimexpr#1\relax*65536/\number\dimexpr#2\relax\relax sp}
%    \end{macrocode}
%% traverse step-wise through counter \#1, start at number \#2 until and including number \#3 and do at every loop \#4 (from \lstinline{forloop.sty}):
%    \begin{macrocode}
\long\def\sh@iterate#1#2#3#4{%
  \advance#1\@ne\relax
  #1=#2\relax%
  \expandafter\ifnum#1>#3\relax%
  \else
    #4%
    \sh@iterate{#1}{\the#1}{#3}{#4}%
  \fi}%
%    \end{macrocode}
%% \section{Hooks}
%% In this part we provide the facility to utilize Hooks to patch code into different parts of the package depending on options and loaded packages.
%    \begin{macrocode}
\RequirePackage{etoolbox}
\def\tpDeclareHook#1{\expandafter\global\expandafter\let\csname tp@hook@#1\endcsname\@empty}
\def\tpAddToHook#1#2{\expandafter\ifx\csname tp@hook@#1\endcsname\relax\else\csgappto{tp@hook@#1}{#2}\fi}
\def\tpUseHook#1{\expandafter\ifx\csname tp@hook@#1\endcsname\relax\else\csname tp@hook@#1\endcsname\fi}
