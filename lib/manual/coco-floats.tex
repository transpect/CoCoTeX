\begin{heading}[label=chap:floats]{chapter}
  \tpTitle{coco-floats.sty}
\end{heading}
\index{coco-floats|textbf}

\begin{heading}{section}
  \tpTitle{Using floats}
\end{heading}

\begin{heading}{subsection}
  \tpTitle{Generic and pre-defined float environments}
\end{heading}

Generally, floats are used with the
\lstinline{tpFloat}\index[tex]{float@\texttt{tpFloat}*} environment:
\begin{lstlisting}[style=tex]
\begin{tpFloat}[<options>]
  <content>
\end{tpFloat}
\end{lstlisting}


\begin{heading}[label=sec:flt:opts]{subsubsection}
  \tpTitle{Options}
\end{heading}

\lstinline{<options>} are options that count only for the given float
environment. The following options are currently implemented for all
floats:
\begin{description}
\item[label=<label>] Label for the float. This can be used to
  reference the heading with \LaTeX's crossref macros like
  \lstinline{\ref}, \lstinline{\pageref}, etc.
\item[class] This allows to further sub-categorize the properties of a
  float environment, see~\ref{sec:flt:classes}.
\item[float-pos] The preferred float position which LaTeX shoudl use
  to place the float on the page, see \autoref{sec:flt:floatpos}.
\item[orientation] If the value is \lstinline{landscape} the float
  will be rotated by 90 degrees via the \lstinline{graphics} package's
  rotating mechanism. The direction can be adjusted with package
  options passed to the \lstinline{graphics} package, see the graphics
  package's documentation for details.
\item[nonumber] if the \lstinline{numbering} property is set to
  \lstinline{auto} (see \autoref{sec:flt:numbering}), suppress the
  numbering of that float. This attribute takes no value.
\item[nolist] if present, this float will generate no entry in the
  float family's contents list (like List of Figures, List of Tables,
  etc).
\item[subfloat] if the \lstinline{numbering} property is set to
  \lstinline{auto} (see \autoref{sec:flt:numbering}), sub-floats of
  this float instance are numbered separately from the overall caption
  type counter and are reset to 1 with each new float environment.
\end{description}


\begin{heading}[label=sec:flt:comp]{subsection}
  \tpTitle{Pre-Defined Components}
\end{heading}

\lstinline{<content>} is a list of Components (see~\ref{sec:enm}). All
floats have the following components pre-defined:

The pre-defined components of a heading are:
\begin{description}
\item[\string\tpCaption]\index[tex]{tpCaption@\texttt{\textbackslash tpCaption}}
  the main caption of the float environment which explains what is to
  be seen in the Content component.
\item[\string\tpLegend]\index[tex]{tpLegend@\texttt{\textbackslash tpLegend}}
  a legend, used to explain certain objects in the Content component.
\item[\string\tpSource]\index[tex]{tpSource@\texttt{\textbackslash tpSource}}
  a source line that described where the Content component is taken from.
\item[\string\tpNumber]\index[tex]{tpNumber@\texttt{\textbackslash tpNumber}}
  the counter of the float. If numbering is set to auto the use of
  \lstinline{tpNumber} \textit{always} overries the automatically
  generated counter. This also counts for sub-floats.
\item[\string\label]\index[tex]{label@\texttt{\textbackslash label}}
  the crossref target to be used with \LaTeX's
  \lstinline{\label}/\lstinline{\ref} mechanism.
\end{description}
Note that floats can only be referred to with the \lstinline{label}
mechanism, when they have a non-empty \lstinline{tpNumber} component.

Note also that there are \textit{two ways} to indicate the target of a
cross reference: with the \lstinline{\label} component, and with the
label-Attribut in the optional Argument of the float
environment. Since internally, the \lstinline{label} attribute sets
the \lstinline{\label} component prior to the content of the
environment is being read, the \lstinline{\label} component
\textit{overrides} the \lstinline{label} attribute if both are used in
the same environment.


\begin{heading}[label=sec:flt:content]{subsection}
  \tpTitle{Content Components}
\end{heading}

Each float environment needs to have a Component that serves as
content container. Generic float environments provide a Component
macro\lstinline{\tpContent}%
\index[tex]{tpContent@\texttt{\textbackslash tpContent}} for that
purpose.

The pre-defined \lstinline{tpFigure} environment uses
\lstinline{\tpFig}\index[tex]{tpFig@\texttt{\textbackslash tpFig}},
which usually contains one call to the \lstinline{\includegraphics}
macro.

The pre-defined \lstinline{tpTable} environment works a bit
differently. The package provides some minor re-definitions of the
\lstinline{tabularx} and \lstinline{tabulary} environments, which make
the tabular contents accessible via the \lstinline{\tpContent} macro.

Since both \lstinline{tabularx} and \lstinline{tabulary} environemnts
rely on the integrity of \LaTeX's \lstinline{tabular} environment,
none such override exists for default \lstinline{tabular}. Therefore,
the \lstinline{tabular} environment \textbf{cannot be used inside}
\lstinline{tpTable}!

The \lstinline{htmltabs} package is also supported (and preferred).


\begin{heading}[label=sec:flt:subfloat]{subsection}
  \tpTitle{Subfloats}
\end{heading}

Float environments can itselves contain several sub-float
environments. Subfloats are coded with the \lstinline{tpSubFloat}
environment:
\begin{lstlisting}[style=tex]
\begin{tpSubFloat}[<options>]
<content>
\end{tpSubFloat}
\end{lstlisting}
The \lstinline{<content>} Components and the options are the same as
for non-subdivided floats.

It is possible to have separate caption blocks for the overall float
and the sub-floats simultaneously, e.g.:
\begin{lstlisting}[style=tex]
\begin{tpFigure}
  \tpCaption{Two figures.}
  \tpNumber{Fig. 2}
  \begin{tpSubFloat}
    \tpNumber{(a)}
    \tpCaption{left figure}
    \tpFig{\includegraphics{example1.eps}}
  \end{tpSubFloat}
  \begin{tpSubFloat}
    \tpNumber{(b)}
    \tpCaption{right figure}
    \tpFig{\includegraphics{example2.eps}}
  \end{tpSubFloat}
\end{tpFigure}
\end{lstlisting}
When the property \lstinline{subfloat-same-height} is set to
\lstinline{true}, the two sub figures are scaled to the same height
and printed next to each other together with their captions in a float
that itself has a caption which overspanns the whole width of both sub
floats. By default, those subfloats are scaled proportionally by the
amount it takes to fill the full width of the current
\lstinline{\linewidth} minus the values of both the
\lstinline{margin-left} and \lstinline{margin-right} properties.

\begin{heading}[label=sec:flt:classes]{subsection}
  \tpTitle{Classes}
\end{heading}

TODO

\begin{heading}[label=sec:flt:floatpos]{subsection}
  \tpTitle{Float position and fixed-position floats}
\end{heading}

Floats in the coco-floats package usually are \textit{floating}
objects, i.e., they are placed in {\LaTeX} insert boxes outside the
text typeface and may be placed in the printed output somewhere else
than they are placed in the tex source file.

The parameter \lstinline{float-pos} in an float enviromnent's option
determins where a float should preferrably be placed. The values of
this parameter correspond to the known float position of default LaTeX
\lstinline{float} environments, with one exception:
\begin{description}[5mm]
\item[t] placement on the top of the page
\item[b] placement on the bottom of the page
\item[h] placement at the position where the float is in the source code, if possible
\item[p] placement on a single page
\end{description}
Those values can be combined to leave {\LaTeX} more freedom to move
floats around, e.g., \lstinline{float-pos=ht} tells the interpreter to
place the float at the position where it is in the source, but top on
the next page if the space doesn't fit for the whole float. If the
float is displaced, the remaining text will continue to be printed
until the current page is full. The float is then placed on the next
page according to the \lstinline{float-pos} value.

One exception from \LaTeX's default behaviour is the
\lstinline{float-pos} value \lstinline{h!}. This forces LaTeX print
the float next, no matter how much space is left on the current
page. If the float object doesn't fit, a large portion of the page may
be left empty and only the following page starts with the float.



\begin{heading}{section}
  \tpTitle{Declaring new and altering existing Float environments}
\end{heading}

New float environments can be defined with the
\lstinline{\tpDeclareFloat} macro:
\begin{lstlisting}[style=tex]
\tpDeclareFloat[<parent>]{<name>}
  {<caption-type>}
  {<listof>}
  {<properties>}
\end{lstlisting}
This macro creates two environments: a \lstinline{<name>} environment,
and a starred \lstinline{<name>*} version to be used for page-wide
floats in \LaTeX's \lstinline{twocolumn} mode.

The \lstinline{\tpDeclareFloat} macro is also used to change the
properties of pre-existing \lstinline{coco-float}
environments. The package ships with two pre-defined float
environments: \lstinline{tpFigure} and \lstinline{tpTable} for figures
and tables, respectively.

The first, optional, argument \lstinline{<parent>} indicates another
float enviroment from which the properties should be inherited. All
floats inherit the \textit{defaut float} properties first, then the
\lstinline{<parent>}'s properties and finally the float environemnt's
own property list is applied at the beginning of each instance of that
float environment.

The first mandatory argument \lstinline{<name>} indicated the name of the float
environment.

The second mandatory argument \lstinline{<caption-type>} indicates the
internal counter that should be used for that float.

The third mandatory argument \lstinline{<listof>} indicates the ending
of the write stream for \LaTeX's list-of mechanism.

The last argument \lstinline{<properties>} is the list of properties
of the float, see~\autoref{sec:flt:props} on a description of all
properties used with floats.

As an example, the \lstinline{tpFigure} environment is defined as
follows:
\begin{lstlisting}[style=tex]
\tpDeclareFloat{tpFigure}{figure}{lof}{%
  \tpSetProperty{subfloat-same-height}{true}%
  \tpSetProperty{content-handler}{\tpFigureHandler}%
  \tpSetProperty{float-block}{\tpFigureFloat}%
  \tpSetProperty{subfloat-block}{\tpFigureSubFloat}%
}
\end{lstlisting}



Since float environments must be able to deal with subfloats,
Components are declared with the macro \lstinline{\tpMakeFltComp}
which takes three arguments:
\begin{lstlisting}[style=tex]
\tpMakeFltComp{<macro name>}{<internal name>}{<component name>}
\end{lstlisting}
\begin{description}[35mm]
\item[\texttt{<macro name>}] is the name of the Component macro.
\item[\texttt{<internal name>}] is the internally used name of the component. It must be different from the other two names.
\item[\texttt{<component name>}] is the name used in \lstinline{\tpUseComp} to refer to that component as well as in the conditionals (see \autoref{sec:common:components}).
\end{description}



\begin{heading}[label=sec:flt:props]{section}
  \tpTitle{Properties}
\end{heading}
\index{Floats!Properties}\index{Properties!Floats}


\begin{heading}[label=sec:flt:props:undef]{subsection}
  \tpTitle{Undefinable Properties}
\end{heading}

\describeUProp{float-number}

The \lstinline{float-number} property contains the float counter
accessible for usage inside the \lstinline{tpSubFloat} environment.

\describeUProp{sub-number}

This property contains the \lstinline{sub-number-before}, the
sub-float counter, and the
\lstinline{sub-number-after}. \lstinline{sub-number-format} is applied
to all of that.



\begin{heading}[label=sec:flt:spacing]{subsection}
  \tpTitle{Spacing}
\end{heading}
\index{Floats!Spacing}

\describeProp{intext-skip-top}{\lstinline{<dimen>}}{0pt}
\describeProp{intext-skip-bottom}{\lstinline{<dimen>}}{0pt}

Space above and below non-floating Floats.

\describeProp{float-skip-top}{\lstinline{<dimen>}}{0pt}
\describeProp{float-skip-bottom}{\lstinline{<dimen>}}{0pt}

Space above and below floating Floats.

\describeProp{sub-float-sep}{\lstinline{<dimen>}}{\texttt{\string\fboxsep}}

Space between Subfloats

\describeProp{margin-left}{\lstinline{<dimen>}}{0pt}
\describeProp{margin-right}{\lstinline{<dimen>}}{0pt}

Left and right margins within the floating environment.


\begin{heading}[label=sec:flt:comp]{subsection}
  \tpTitle{Composition}
\end{heading}
\index{Floats!composition}

\describeProp{before-float}{\lstinline{<any>}}{\texttt{\string\parindent\string\z@}}

Stuff to be executed before content is evaluated.

\describeProp{content-handler}{\lstinline{<any>}}{\textit{see below}}

The caption type specific content handler. The function behind the
\lstinline{content-handler} provides the float's accepted components
and tells the package how to handle the actual content (i.e.,
everything apart from the captions) of the floating environment.

The default value for any float is the \lstinline{\tpGenericHandler}
macro. The \lstinline{tpFigure} float uses
\texttt{\string\tpFigureHandler}, while \lstinline{tpTable} uses
\texttt{\string\tpTableHandler}.

\describeProp{float-block}{\lstinline{<any>}}{\textit{see below}}

The caption type specific content printer. The function behind this
property tells the package how to print the contents of the float
environment. This is where the Content component or the macro that
contains the contents is expanded and printed.

The default value for any float is \lstinline{\tpGenericFloat} which
simply expands to
\lstinline|\tpUseComp{Content}|. \lstinline{tpTable}'s default value
is \lstinline{\tpTableFloat}, \lstinline{tpFigure}'s default is
\lstinline{\tpFigureFloat}.

\describeProp{subfloat-block}{\lstinline{<any>}}{\textit{see below}}

The caption type specific content printer for subfloats. If a float
contains subfloats the Content Components of those subfloats may need
to be handled differently than a single Content Component.

By default, this property is only pre-defined for the \lstinline{tpFigure}
environment, which uses
\texttt{\string\tpFigureSubFloat}. \lstinline{tpTable} and the generic
float use the same default value for \lstinline{subfloat-block} as is
used for \lstinline{float-block}, i.e., \lstinline{\tpTableFloat} for
\lstinline{tpTable}, and \lstinline{\tpGenericFloat} for any other
float.

\describeProp{subfloat-same-height}{\lstinline{<any>}}{}

If non-empty, sub-floats are scaled to a common height. 

By default, this property is set to empty for all floats, except for
\lstinline{tpFigure}, whose default value is \lstinline{true}. 

\textbf{Warning:} Use this property with a non-empty value only for
float contents that are actually scalable. As of this version of
coco-floats, this is only implemented for figures inserted using
the \lstinline{\includegraphics} command.

\begin{heading}[label=sec:flt:capt]{subsection}
  \tpTitle{Captions}
\end{heading}
\index{Floats!Caption}

\describeProp{caption-format}{\lstinline{<any>}}{}

Format of the caption. This proeprty is applied to all parts of the
caption, both below and above the float content.

\describeProp{caption-format-top}{\lstinline{<any>}}{}
\describeProp{caption-format-bottom}{\lstinline{<any>}}{}

Format of the portions of the caption that are placed above or below
the float content, respectively.

\describeProp{caption-sep-top}{\lstinline{<dimen>}}{0pt}
\describeProp{caption-sep-bottom}{\lstinline{<dimen>}}{0pt}

Vertical space between the portion of the caption that is placed above
or below the float content and the content itself. These properties
are each only used when the respective top or bottom portions of the
caption are non-empty.


\describeProp{caption-top}{\lstinline{<any>}}{\textit{see below}}

The format of the portion of the caption that is placed \textit{above}
the float.

The default value for all floats is:
\begin{lstlisting}[style=tex]
\tpIfComp{Number}
  {\tpUseComp{Number}\tpUseProperty{number-sep}}
  {}%
\tpUseComp{Caption}%
\end{lstlisting}

\describeProp{caption-bottom}{\lstinline{<any>}}{\textit{see below}}

The format of the portion of the caption that is placed \textit{below}
the float.

The default value for all floats is:
\begin{lstlisting}[style=tex]
\tpUseComp{Legend}\\
\tpUseComp{Source}%
\end{lstlisting}

\describeProp{sub-caption-valign-top}{\texttt{t}, \texttt{b}, \texttt{m}}{\texttt{t}}
\describeProp{sub-caption-valign-bottom}{\texttt{t}, \texttt{b}, \texttt{m}}{\texttt{t}}

Vertical alignment of subfloat caption portions that are placed above
and below the content, respectively.

Captions are placed in \LaTeX's minipages and this property is directly
passed to the first optional argument of the \lstinline{minipage}
environment.

The default value for \lstinline{tpTable} is \lstinline{b}.


\begin{heading}[label=sec:flt:numbering]{subsection}
  \tpTitle{Numbering}
\end{heading}
\index{Floats!Numbering}

\describeProp{numbering}{\lstinline{auto}, \lstinline{<empty>}}{\lstinline{auto}}

If set to \lstinline{auto}, floats are numbered automatically.

\describeProp{number-sep}{\lstinline{<dimen>}}{\texttt{\string\enskip}}

Separator between Label and Caption.

\describeProp{sub-number-block}{\lstinline{<any>}}{\textit{see below}}

Composition of sub-float counters.

The defaut value is:
\begin{lstlisting}[style=tex]
\tpUseProperty{float-number}%
\tpUseProperty{sub-number-sep}%
\tpUseProperty{sub-number}%
\end{lstlisting}
Note that float-number and sub-number are undefinable properties that hold the flaot- and the subfloat counters, respectively; see \autoref{sec:flt:props:undef} for details.

\describeProp{sub-number-sep}{\lstinline{<any>}}{\texttt{\string\,}}

Separator between main float counter and sub-float counter in float
environments with the \lstinline{subfloat} option.

\describeProp{sub-number-style}{\lstinline{alph}, \lstinline{Alph}, \lstinline{arabic}, \lstinline{Roman}, \lstinline{roman}}{\lstinline{alph}}

Counting style of sub-float counters.

\describeProp{sub-number-before}{\lstinline{<any>}}{\texttt{(}}
\describeProp{sub-number-after}{\lstinline{<any>}}{\texttt{)}}

Material that is printed immediately \textit{before} and
\textit{after} the sub-number counter, respectively.
\lstinline{sub-number-format} is applied to the whole triade:
\begin{lstlisting}[style=tex]
\begingroup
  \tpUseProperty{sub-number-format}%
  \tpUseProperty{sub-number-before}%
  \csname @\tpUseProperty{sub-number-style}\endcsname
    {<sub-number counter>}%
 \tpUseProperty{sub-number-after}%
\endgroup
\end{lstlisting}
\describeProp{list-of-block}{\lstinline{<any>}}{\texttt{\string\tpUseComp{Caption}}}

The content that is passed to \LaTeX's list of <float> mechanism.


\describeProp{number-format}{\lstinline{<any>}}{\texttt{\string\bfseries}}

Format of the float counter, additional to \lstinline{caption-format}.

\describeProp{source-format}{\lstinline{<any>}}{}

Format of the \lstinline{Source} component.

\describeProp{legend-format}{\lstinline{<any>}}{}

Format of the \lstinline{Legend} component.

\describeProp{sub-number-format}{\lstinline{<any>}}{}

Format of the sub-float counter.




