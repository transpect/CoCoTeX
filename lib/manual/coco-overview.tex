\begin{Heading}{chapter}
  \Title{Overview}
\end{Heading}

\begin{Heading}{section}
  \Title{Installation and Usage}
\end{Heading}

\begin{Heading}{subsection}
  \Title{Installation}
\end{Heading}

\index{Installation}The current build of the package can be obtained from GitHub:
\begin{lstlisting}[style=tex,language=bash]
git clone https://github.com/transpect/CoCoTeX.git
\end{lstlisting}
The actual source files can be found in the \lstinline{src} sub-folder.

The most recent stable version can be found in the
\lstinline{releases} folder. It contains the \lstinline{cocotex.dtx}
file, its corresponding \lstinline{cocotex.ins} file, and both the source code documentation and this
End-User Manual as a pre-rendered PDFs.

The package is installed via
\begin{lstlisting}[style=bash]
latex cocotex.ins
\end{lstlisting}
This will create the \lstinline{cocotex.cls} file, as well as some
additional modules that follow the naming convention
\lstinline{coco-<module>.sty}. These modules will be explained in
greater detail below in \fullref{sec:modules}.

The documentation of the framework's source code can be created via
\begin{lstlisting}[style=bash]
lualatex cocotex.dtx
\end{lstlisting}
\textbf{Note:} You \textit{must} use \lstinline{lualatex} in order to
create the source code documentation!

\textbf{Note:} The source code documentation is a technical breakdown
of the framework's source code; it is not the same document as the
more user-oriented Manual you are currently reading.

\begin{Heading}{subsection}
  \Title{Usage}
\end{Heading}

{\CoCoTeX}\index{Usage} follows a modular design. It comes with several
\lstinline{.sty} files that can be used independently from
another. However, there is also a {\LaTeX} Document Class file
\lstinline{cocotex.cls} which can be used to load the whole framework
at once.

\begin{Heading}{subsubsection}
  \Title{Using cocotex.cls}
\end{Heading}

The \lstinline{cocotex.cls} serves as stand-in for the {\LaTeX}
default document classes \lstinline{article} and \lstinline{book}. It
is called with the usual {\LaTeX} commmand:
\begin{lstlisting}[style=tex]
\documentclass[<options>]{cocotex}
\end{lstlisting}

The actual document type can be set with the \lstinline{pubtype}\index{class option>pubtype}\index{pubtype|see{class option}} option:
\begin{lstlisting}[style=tex]
\documentclass[pubtype=<mono|article|collection|journal>]{cocotex}
\end{lstlisting}
The allowed values are:
\begin{description}
\item[\ttfamily mono] for monographs, i.e., books that are written by one or
  multiple authors as a whole,
\item[\ttfamily article] for single journal articles,
\item[\ttfamily collection] for books that are collections of contributions of
  multiple authors, and
\item[\ttfamily journal] for journals, i.e., collections of multiple journal
  articles.
\end{description}

% \textbf{Important:} While most modules are designed to be used with
% any {\TeX} render engine, {\CoCoTeX} documents that use the
% \lstinline{coco-script} module will have to be compiled with
% \textbf{lua\LaTeX}\footnote{{Xe\LaTeX} may work as well but is not
%   officially supported.}!

\begin{Heading}[label=sec:modules]{subsubsection}
  \Title{Using Single Modules}
\end{Heading}

{\CoCoTeX} is modular. That means you can use selected modules as
packages together with \LaTeX's default or other third-party document
classes. Modules are included like any other package, e.g.,
\begin{lstlisting}[style=tex]
\RequirePackage[<options>]{coco-floats}
\RequirePackage[<options>]{coco-headings}
\RequirePackage[<options>]{coco-title}
\end{lstlisting}

\begin{Heading}{section}
  \Title{Design Goals and Purpose}
\end{Heading}

{\CoCoTeX} is a programming framework for {\LaTeX} developers, who
need to build and maintain a number of (not too) different
publisher-specific style sheets and partly or fully automatted
typesetting processes. Its original purpose is to serve as a rendering
backend for the typesetting tool \textit{xerif}, but it is also usable
as a standalone extension to plain {\LaTeX}.

The following features are the main design goals of the {\CoCoTeX}
framework:
\begin{itemize}
\item Handling of different document types in the same stylesheet:
  \begin{itemize}
  \item journal articles
  \item whole journals
  \item chapters by different authors in proceedings and collections,
  \item text collections and proceedings, and
  \item monographs by (a) single author(s).
  \end{itemize}
\item Handling of recurring complex elements that are difficult to
  set-up using standard-\LaTeX, e.\,g.
  \begin{itemize}
  \item headings of all levels with authors, subtitles, quotes, etc.;
  \item a four-way distinction of material in a heading's title, its
    pendant in headers and footers, and their entry in the table(s) of
    contents, and in the PDF bookmarks; and
  \item the possibility to provide classes of text components like
    headings and floats, similar to classes in HTML/CSS; and
  \item the structured handling of meta-data, especially for
    titlepages.
  \end{itemize}
\end{itemize}

The framework introduces some new concepts into {\LaTeX} programming
that are extensivley influenced by object-oriented design principles.
The name {\CoCoTeX} is derived from two of those concepts, namely
\textbf{Co}ntainers and \textbf{Co}mponents. In the next sections,
those and other concepts are exlained in more detail.



\begin{Heading}[label=sec:concepts]{subsection}
  \Title{Basic Concepts}
\end{Heading}

One design goal of the {\CoCoTeX} Framework is to provide an easy and
unified way to configure the typesetting of blocks of inter-connected
data.

For instance, take \textit{headings}: They always consist of
a~\textit{Title}, but also may have some sort of \textit{Numbering},
some have a~\textit{Subtitle}, some might have a~dedicated
\textit{Author}, some are followed by a~\textit{Quote} or
a~\textit{Motto}. They may re-appear (partly) in the head-line of
a~page, as well as in the table of contents, in some cases with
slightly altered data.


\begin{Heading}[label=sec:overview.containers]{subsubsection}
  \Title{Containers}\index{Container|textbf}
\end{Heading}

Such a~bundle of structured information in the {\CoCoTeX} framework is
referred to as a~\textbf{Container}. In the
aforementioned example, the information pieces “Title”, “Subtitle”,
“Author” etc., together form a unit “heading”.

Containers can be derived from one another. For instance, the abstract
concept “heading” might be extended to various levels, like “section”,
“chapter”, “paragraph”, or “part”. Some of the abstract constituents,
like Title or Number, are shared among all those derivations of the
concept “heading”, while others might not. E.g. “Author” is usualy
used on “chapters”, but rarely on “sections” or even deeper
levels. The mechanism to pass certain properties or constituents from
one Container to another is called
\textbf{Inheritance}\index{Inheritance|textbf}.

\begin{Heading}[label=sec:overview.components]{subsubsection}
  \Title{Components}\index{Component|textbf}
\end{Heading}

The inter-connected pieces of information that constitute a Container
are called \textbf{Components} in the
{\CoCoTeX} framework. Most basic components are simple {\LaTeX} macros
that take one argument for the content that is to be stored inside
that Component for the respective Container.

Some Components can be collections of other Components. An “Author”
Component of a “heading” Container, for example, can contain a~first,
a~middle and a~last name, an~academic title, an~affiliation, or
an~email address, among other things. Those complex Components are
called \textbf{Component Groups}\index{Group Container|textbf}.

Some Components may occur multiple times in the same parent
Container. A~good example are mutliple Authors that contibuted to the
same chapter in a collection. Those Components are called
\textbf{Counted Components}\index{Counted Component|textbf}. Note that
despite the name, it is not necessarily the case that those Counted
Components are numbered or even ordered in any way. Rather,
``Counted'' refers to the way they are processed internally. Due to
the way both concepts are implemented in {\CoCoTeX}, Group Containers
are always also Counted Containers and vice-versa, so both terms might
be used interchangeably.

\begin{Heading}[label=sec:overview.properties]{subsubsection}
  \Title{Properties}\index{Properties|textbf}
\end{Heading}

While headings of the same level are usually rendered in the same way
for a given publisher style, the actual typeface may vary depending on
the Components that are actually filled with content for a given
heading: A~numbered heading might look slightly different than one
without a~number; a~heading with a~subtitle might have different
spacing than one without a~subtitle, and so on.

How Components are processed and ultimatly rendered is controlled by
so-called \textbf{Properties}. Properties are
mostly short pieces of {\LaTeX} code that are usually set by the
stylesheet developer according to a publisher's guidelines and
requirements. One of the design goals of {\CoCoTeX} is to keep the
code behind those Properties as comprehensible and pointed as
possible.

\begin{Heading}[label=sec:overview.types]{subsubsection}
  \Title{Types, Scope and Modular Inheritance}\index{Type|textbf}
\end{Heading}

Properties and Components can be seen as Container‐specific Data
\textbf{Type}s. They are only defined within the
scope of their parent Container and are usually not accessible from
the outside. When a new Container is declared, it can inherit the Data
Types from one or multiple other containers.

\begin{Heading}[label=sec:oop]{subsection}
  \Title{Concepts from Object-Oriented Programming}
\end{Heading}


\textbf{Containers} are comparable to the concept of \textit{classes}
in object-oriented programming. A concrete heading in a document is an
\textit{instance} of that class. \textbf{Components} serve as
\textit{class variables}, \textbf{Properties} can be seen as
\textit{instance methods}. \textbf{Types} can include macros and
control sequences that are somewhat comparable to \textit{class
  methods}.

The \textbf{Inheritance} and \textbf{Type} mechanisms are comparable
to \textit{Mixins} in some object-oriented programming languages like
Ruby.


\begin{Heading}{subsection}
  \Title{Implementation of {\CoCoTeX} Concepts in LaTeX Documents}
\end{Heading}

In the {\CoCoTeX} framework, \textit{Containers} are realised in the
document source as {\LaTeX} environments. \textit{Simple Components}
are {\LaTeX} commands that take one argument while Group Components
are {\LaTeX} environments that hold the Commands for its consitutent
Components:
\begin{lstlisting}[style=tex]
\begin{<Container>}[<options>]
  \<Component1>{<Content1>}
  \<Component2>{<Content2>}
  \begin{<GroupComponent>}
    \<Component3>{<Content3>}
    \<Component4>{<Content4>}
  \end{<GroupComponent>}
\end{<Container>}
\end{lstlisting}

The basic idea is that the Content in the Argument of the
Component commands within a Container are collected,
processed and the output is printed at the end of the corresponding
Container environment. Containers allow Components with the same name
to be used and processed independently in different Containers.

Components are only allowed within their corresponding Container
environments. Outside, Container sensitive Components may have
different meaning or even throw an \texttt{Undefined control sequence}
error.

Components provided by the modules of the {\CoCoTeX} framework usually
start with the \lstinline{tp}\footnote{That the prefix is „tp“ has
  historic reasons: The earliest version of {\CoCoTeX} was called
  „transpect-tex“ after the XML conversion tool \textit{transpect},
  which constitutes the first major component of \textit{xerif} (with
  {\CoCoTeX} being the second).} prefix and the component's name
begins with a capital letter. For instance,
\begin{lstlisting}[style=tex]
\begin{Figure}
  \Caption{A~nice image.}
\end{Figure}
\end{lstlisting}
utilizes a Component named \texttt{Caption} with the \texttt{Content}
\textit{“A~nice image.”} within the container \texttt{Figure}.

Properties consist of two parts, the property's name and its
value. Some Properties provided by the {\CoCoTeX} modules may have
a~fixed set of string values, while others are completely free to be
set and used.

In this manual, the properties provided by the various modules are
documented in the following way:

{\let\noindex\relax\describeProp{<name>}{\texttt{<allowed values>}}{\texttt{<default value>}}}

A property with the name \texttt{<name>} is set by default to
\texttt{<default value>}. The user may chose to set it to any of the
\texttt{<allowed values>}.
\begin{description}
\item[\texttt{<any>}]  the user is completely free to set this
  property to any value she wants.
\item[\texttt{<dimen>}] It is expected the property to be a
  dimension. This may be a length or dimension register, a fix
  value-unit pair that is understood by \TeX, or a macro that expands
  to a dimension/length.
\item[\texttt{<num>}] It is expected the property to be a numeric
  value. This may be a counter register, a fix value, or a macro that
  expands to a number.
\end{description}
\lstinline{<allowed values>} without angles mean that those are fixes
strings that have a special meaning. Those are explained in the
descriptions below the property header.

The “data type” \texttt{<empty>} is used to indicate that the property
is un-set or empty. This is the default for some of the properties
provided by the {\CoCoTeX} modules, but basicly all properties can
be set to \texttt{<empty>}.


\begin{Heading}{section}
  \Title{Overview: Modules}
\end{Heading}

The following modules are included in {\CoCoTeX}:

\begin{Heading}{subsection}
  \Title{User-Level Modules}
\end{Heading}

\begin{description}
\item[\ttfamily coco-headings.sty] The \lstinline{headings}
  module\index{coco-headings.sty} provides a new way to declare and
  use chapter, section and paragraph titles. It is described in
  greater detail in \fullref{chap:headings}.
\item[\ttfamily coco-floats.sty] The \lstinline{floats}
  module\index{coco-floats.sty} provides some extended handling for
  floating objects like tables or figures. It is described in greater
  detail in \fullref{chap:floats}.
\item[\ttfamily coco-title.sty] The
  \lstinline{title}\index{coco-title.sty} module provides meta data
  handlers for title pages. It is described in greater detail in
  \fullref{chap:title}.
\item[\ttfamily coco-frame.sty] \index{coco-frame.sty}
\item[\ttfamily coco-notes.sty] The \lstinline{notes}
  module\index{coco-notes.sty} handles the easy switching between
  footnotes and endnotes, as well as the position where and in what
  way endnotes are printed. It is described in greater detail in
  \fullref{chap:endnotes}.
% \item[\ttfamily coco-script.sty] This module provides support for non-latin scripts utilizing Google's Noto fonts. It is described in greater detail in \fullref{chap:scripts}.
% \item[\ttfamily coco-lists.sty]
\end{description}

\begin{Heading}{subsection}
  \Title{Backend Modules}
\end{Heading}

\begin{description}
\item[\ttfamily coco-kernel.sty] The \lstinline{kernel}
  module\index{coco-kernel.sty} is the heart of the CoCoTeX
  framwork. As such, it is a hard dependency for all other modules and
  loaded automatically.
\item[\ttfamily coco-common.sty] The \lstinline{common}
  module\index{coco-common.sty} is a collection helper macros and
  functions, that are not per-se part of the {\CoCoTeX} Framework, but
  utilised by multiple other modules. The common module is loaded
  automatically by some of the other modules, but not by all.
\item[\ttfamily coco-meta.sty] The \lstinline{meta}
  module\index{coco-common.sty} collects methods and concepts that are
  used by both the \lstinline{title} and \lstinline{headings}
  modules. It is therefore auto-loaded by both modules.
\end{description}


