\begin{Heading}{chapter}
  \Title{Overview}
  \RefLabel{chap:overview}
\end{Heading}

\begin{Heading}{section}
  \Title{Installation and Usage}
\end{Heading}

\begin{Heading}{subsection}
  \Title{Installation}
\end{Heading}

\index{Installation}The current build of the package can be obtained from GitHub:
\begin{lstlisting}[style=tex,language=bash]
git clone https://github.com/transpect/CoCoTeX.git
\end{lstlisting}
The actual source files can be found in the \lstinline{src} sub-folder.

% The latest stable version can be found inside the\lstinline{releases}
% folder. It contains the \lstinline{cocotex.dtx} file, its
% corresponding \lstinline{cocotex.ins} file, and both the source code
% documentation and this End-User Manual as a pre-rendered PDFs.

The most recent stable version that is in active use on the
\textit{xerif} servers can be found at
\url{https://github.com/transpect/xerif-latex}. Note, however, that is
version might be several minor versions behind the source code release
and contains additional, xerif-specific, files that are not part of
the official {\CoCoTeX} build.

The package is installed via
\begin{lstlisting}[style=bash]
latex cocotex.ins
\end{lstlisting}
This will create the \lstinline{cocotex.cls} file, as well as some
additional modules that follow the naming convention
\lstinline{coco-<Module>.sty}. These modules will be explained in
greater detail below in \fullref{sec:modules}.

The documentation of the framework's source code can be created via
\begin{lstlisting}[style=bash]
lualatex cocotex.dtx
\end{lstlisting}
\textbf{Note 1:} You \textit{must} use \lstinline{lualatex} in order to
create the source code documentation!

\textbf{Note 2:} The source code documentation is a technical breakdown
of the framework's source code; it is not the same document as the
more user-oriented Manual you are currently reading.


\begin{Heading}{subsection}
  \Title{Dependencies}
\end{Heading}

{\CoCoTeX}\index{Dependencies} requires a fairly recent LaTeX kernel. It is recommended to
use the latest TeXlive build, but not older than texlive 2022, since
it uses some newly added concepts like Hooks and Sockets..

The following packages are requied by the various {\CoCoTeX} modules:
\begin{description}
\item[\ttfamily coco-kernel*] requires \texttt{kvoptions-patch},
  \texttt{xkeyval}, and \texttt{etoolbox}
\item[\ttfamily coco-common*] requires \texttt{coco-kernel}, \texttt{iftex},
  \texttt{xcolor}, and \texttt{graphicx}
\item[\ttfamily coco-floats*] requires \texttt{coco-common} and
  \texttt{rotating}, \texttt{grffile}, \texttt{footnote},
  \texttt{adjustbox}, and \texttt{stfloats} and supports
  \texttt{tabularx}, \texttt{tabulary}, and
  \texttt{htmltabs}\footnote{The \texttt{htmltabs.sty} is included in
    \CoCoTeX's main GitHub Repository in the
    \texttt{externals/htmltabs/} folder}
\item[\ttfamily coco-meta*] requires \texttt{coco-common}
\item[\ttfamily coco-heading*] requires \texttt{coco-meta}, and
  \texttt{bookmark}
\item[\ttfamily coco-notes*] requires \texttt{footnote}, and \texttt{endnotes}
\item[\ttfamily coco-title*] requires \texttt{coco-meta}
\item[\ttfamily coco-accessibility] requires lua\LaTeX, \texttt{coco-kernel},
  and \texttt{ltpdfa}\footnote{\texttt{ltpfa} is included in the
    \texttt{externals/ltpdfa} folder in CoCoTeX's GitHub
    repository. Note that CoCoTeX uses only the .lua files from that
    package}. Older {\LaTeX} kernel versions require
  \texttt{atbegshi}, \texttt{xparse}, \texttt{luatexbase-attr}, and
  \texttt{atveryend}
\item[\ttfamily coco-lists] requires \texttt{coco-common}, \texttt{footnote},
  and \texttt{endnotes}
\item[\ttfamily coco-frame] requires \texttt{luatex85}, and \texttt{crop}
\item[\ttfamily coco-script] requires \texttt{coco-kernel}, \texttt{babel},
  \texttt{fontspec} (and therefore {lua\LaTeX} or Xe\LaTeX), and
  \texttt{filecontents}
\end{description}

The {\CoCoTeX} class file \texttt{cocotex.cls} includes most (namely
those indicated with an asterisk in the list above) {\CoCoTeX} modules
and requires \textit{additionally} the \texttt{index} and
\texttt{hyperref} packages. Note that all those packages might have
secondary dependencies.

{\CoCoTeX} itself is designed to run with all {\LaTeX} engines,
however, in partiular the \texttt{coco-script} and
\texttt{coco-accessibility} modules require {lua\LaTeX}, therefore
those modules are either not loaded (coco-script), or it is
loaded but not activated (coco-accessibility) by default.


\begin{Heading}{subsection}
  \Title{Usage}
\end{Heading}

{\CoCoTeX}\index{Usage} follows a modular design. It comes with several
\lstinline{.sty} files that can be used independently from
another. However, there is also a {\LaTeX} Document Class file
\lstinline{cocotex.cls} which can be used to load the whole framework
at once.

\begin{Heading}{subsubsection}
  \Title{Using cocotex.cls}
\end{Heading}

The \lstinline{cocotex.cls} serves as stand-in for the {\LaTeX}
default document classes \lstinline{article} and \lstinline{book}. It
is called with the usual {\LaTeX} commmand:
\begin{lstlisting}[style=tex]
\documentclass[<options>]{cocotex}
\end{lstlisting}

The actual document type can be set with the \lstinline{pubtype}\SeeIndex{pubtype}{Class Options} option:
\begin{lstlisting}[style=tex]
\documentclass[pubtype=<mono|article|collection|journal>]{cocotex}
\end{lstlisting}
The allowed values are:
\begin{description}
\item[\ttfamily mono] for monographs, i.e., books that are written by one or
  multiple authors as a whole,
\item[\ttfamily collection] for books that are collections of contributions of
  multiple authors, and
\item[\ttfamily article] for single journal articles,
\item[\ttfamily journal] for journals, i.e., collections of multiple journal
  articles.
\end{description}


\begin{Heading}[label=sec:modules]{subsubsection}
  \Title{Using Single Modules}
\end{Heading}

{\CoCoTeX} is designed to be used modularly. That means you can use
selected modules as packages together with \LaTeX's default or other
third-party document classes. Modules are included like any other
package, e.g.,
\begin{lstlisting}[style=tex]
\RequirePackage[<options>]{coco-floats}
\RequirePackage[<options>]{coco-headings}
\RequirePackage[<options>]{coco-title}
\end{lstlisting}

\begin{Heading}{section}
  \Title{Design Goals and Purpose}
\end{Heading}

{\CoCoTeX} is a programming framework for {\LaTeX} developers who need
to build and maintain a number of (not too) different
publisher-specific style sheets in partly or fully automatted
typesetting processes. Its original purpose is to serve as a rendering
backend for the typesetting tool \textit{xerif}\footnote{see
  \url{https://www.le-tex.de/en/xerif.html}}, but it is also usable as
a standalone extension to plain {\LaTeX}.

The following features are the main design goals of the {\CoCoTeX}
framework:
\begin{itemize}
\item Handling of different document types in the same stylesheet:
  \begin{itemize}
  \item journal articles
  \item whole journals
  \item chapters by different authors in proceedings and collections,
  \item text collections and proceedings, and
  \item monographs by (a) single author(s).
  \end{itemize}
\item Handling of recurring complex elements that are difficult to
  set-up using standard-\LaTeX, e.\,g.
  \begin{itemize}
  \item headings of all levels with authors, subtitles, quotes, etc.;
  \item a four-way distinction of material in a heading's title, its
    pendant in headers and footers, and their entry in the table(s) of
    contents, and in the PDF bookmarks; and
  \item the possibility to provide classes of text components like
    headings and floats, similar to classes in HTML/CSS; and
  \item the structured handling of meta-data, especially for
    titlepages and accessible PDFs.
  \end{itemize}
\end{itemize}

To achieve those goals, the framework introduces some concepts into
{\LaTeX} programming that are extensivley influenced by
object-oriented design principles. The name {\CoCoTeX} is derived from
two of those concepts, namely \textbf{Co}ntainers and
\textbf{Co}mponents.

The most recent versions of the {\LaTeX} kernel (Texlive 2024 and
later) has seen some quite similar concepts being introduced after
they existed before in the form of the \lstinline{xtemplate}
package. The core mechanics of {\CoCoTeX} and the relationship between
{\CoCoTeX} and {\LaTeX} Templates are explained in greater detail in
\fullref{chap:design}.


\begin{Heading}{section}
  \Title{Overview: Modules}
\end{Heading}

As mentioned earlier, {\CoCoTeX} is modular. The following modules are
included in {\CoCoTeX}:


\begin{Heading}{subsection}
  \Title{User-Level Modules That Use Containers}
\end{Heading}

\begin{description}
\item[\ttfamily coco-headings.sty] The \lstinline{headings}
  module\index{Headings}\index{Module>Headings} provides a new way to declare and
  use chapter, section and paragraph titles. It is described in
  greater detail in \fullref{chap:headings}.
\item[\ttfamily coco-floats.sty] The \lstinline{floats}\index{Module>Floats}
  module provides some extended handling for
  floating objects like tables or figures. It is described in greater
  detail in \fullref{chap:floats}.
\item[\ttfamily coco-title.sty] The
  \lstinline{title}\index{Title}\index{Module>Title} module provides meta data
  handlers for title pages. It is described in greater detail in
  \fullref{chap:title}.
\item[\ttfamily coco-lists.sty] The lists\index{Lists}
  module\index{Module>Lists} provides support for list
  environments. It is described in greater detail in
  \fullref{chap:lists}.
\end{description}


\begin{Heading}{subsection}
  \Title{User-Level Modules That Do Not Use Containers}
\end{Heading}

\begin{description}
\item[\ttfamily coco-frame.sty] \index{Frame}\index{Module>Frame}
  provides some helper tools to make the type area visible and add
  crop marks for printing. It is explained in more detail in
  \fullref{chap:frame}.
\item[\ttfamily coco-notes.sty] The \lstinline{notes}\index{Module>Notes}
  module\index{Notes} handles the easy switching between
  footnotes and endnotes, as well as the position where and in what
  way endnotes are printed. It is described in greater detail in
  \fullref{chap:notes}.
\item[\ttfamily coco-script.sty] This module provides support for
  non-latin scripts utilizing Google's Noto fonts. It is described in
  greater detail in \fullref{chap:scripts}.
\item[\ttfamily coco-accessibility.sty] The
  accessibility\index{Accessibility}
  module\index{Module>Accessibility} provides support to generate PDFs
  that conform to the PDF/UA standard and interfaces for the ltpdfa
  package. It will be described in greater detail in
  \fullref{chap:ally} with some remarks in \fullref{sec:design:ally}.
\end{description}


\begin{Heading}{subsection}
  \Title{Back End Modules}
\end{Heading}

\begin{description}
\item[\ttfamily coco-kernel.sty] The
  \lstinline{kernel}\index{Module>Kernel} module\index{Kernel} is the
  heart of the CoCoTeX framwork. As such, it is a hard dependency for
  all other modules and loaded automatically.The kernel module is
  explained in greater detail in \fullref{chap:customize}.
\item[\ttfamily coco-common.sty] The
  \lstinline{common}\index{Module>Common} module\index{Common} is a
  collection helper macros and functions, that are not per-se part of
  the {\CoCoTeX} Framework, but utilised by multiple other
  modules. The common module is loaded automatically by some of the
  other modules, but not by all. It is explained in greater detail in
  \fullref{chap:common}.
\item[\ttfamily coco-meta.sty] The \lstinline{meta}\index{Module>Meta}
  module\index{Meta} collects methods and concepts that are used by
  both the \lstinline{title} and \lstinline{headings} modules. It is
  therefore auto-loaded by both modules. It is explained in greater
  detail in \fullref{chap:headings} and \fullref{chap:title},
  respectively.
\end{description}
