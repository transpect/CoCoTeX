\begin{Heading}{chapter}
  \Title{Custom Containers}
\end{Heading}


As we already discussed in chapter~\ref{chap:overview},
Containers\Index{Container} are representations of typographical
elements that share a more or less fixed set of components that are
supposed to be rendered in a similar way.

In this section, we discuss how to declare custom Containers.

A new, custom Container can be declared with the
\DeclareMacro{\ccDeclareContainer} command:
\begin{lstlisting}
\ccDeclareContainer{<name>}{<body>}
\end{lstlisting}
where \lstinline{<name>} is the name of the Container, and
\lstinline{<body>} is a list of Container Type Declarations.

Data Types are declared with the \DeclareMacro{\ccDeclareType} command:
\begin{lstlisting}
\ccDeclareType{<name>}{<body>}
\end{lstlisting}
where \lstinline{<name>} is the name of the Data Type, and
\lstinline{<body>} a list of type-specific variable declarations. The
most commonly used Data Types are \textit{Properties},
\textit{Attributes}, and \textit{Components}, but essentially, they
can be named anything.

Another common command inside the Container body is the
\DeclareMacro{\ccInherit} command. It takes two arguments:
\InlineArg{1} is a~comma-separated list of Data Types, and
\lstinline{2}~is a~comma-separated list of Container names. The newly
defined Contianer will then inherit all Data Type declarations from
the parent Containers. For instance,
\begin{lstlisting}
\ccDeclareContainer{Parent 1}{%
  \ccDeclareType{Properties}{...}%
  \ccDeclareType{Components}{...}%
  \ccDeclareType{Junk}{...}%
}
\ccDeclareContainer{Parent 2}{%
  \ccDeclareType{Properties}{...}%
  \ccDeclareType{Components}{...}%
}
\ccDeclareContainer{Child}{%
  \ccInherit{Properties,Components}{Parent 1,Parent 2}%
}
\end{lstlisting}
means that the Container named \lstinline{Child} will inherit the
\textit{Components} and \textit{Properties} Data Types from both
Containers \lstinline{Parent 1} and \lstinline{Parent 2},
respectively, but not the Data Type \lstinline{Junk} from \lstinline{Parent 1}.

Note that both \UsageMacro{\ccDeclareType} and \UsageMacro{\ccInherit}
can only be used inside the body of a Container Declaration!
