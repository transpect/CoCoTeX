\begin{Heading}[label=chap:common]{chapter}
  \tpTitle{coco-common.sty}
\end{Heading}
\index{coco-common|textbf}

This module of the {\CoCoTeX} Framework is the core
that is used by (almost) all other modules in the framework. It
provides the basic facilities for Component, Namespace, Hook, and
Property handling.

\begin{Heading}[label=sec:use_props]{section}
  \tpTitle{Using Properties}
\end{Heading}

\begin{Heading}[label=sec:set_props]{subsection}
  \tpTitle{Setting and Using Property}\index{Properties!Setup}\index{Properties!Usage}
\end{Heading}

Properties are set up using one of either commands
\begin{lstlisting}[style=tex]
\tpSetProperty{<name>}{<val>}
\ccSetPropVal{<name>}{<val>}
\tpSetPropertyX{<name>}{<val>}
\end{lstlisting}
With
\lstinline{\ccSetPropVal}\index[tex]{SetPropVal@\texttt{\textbackslash
    ccSetPropVal}}, the second argument is expanded \textit{once}
prior to the assignment. Use this if you want to assign the current
value of a {\LaTeX} control sequence (e.g., a length register) to a
property.

\lstinline{\tpSetPropertyX}\index[tex]{tpSetPropertyX@\texttt{\textbackslash
    tpSetPropertyX}} is another variant of \lstinline{\tpSetProperty}
except that it expands the value \#2 \textit{fully} before assigning
it to the property macro with the name \#1.  Use this if you need to
use conditionals to determine the actual values of Properties that
otherwise expect fixed named or dimensional values.


When using \lstinline{\tpSetProperty}\index[tex]{SetProperty@\texttt{\textbackslash tpSetProperty}}, the second argument may not be
expanded until the property is evaluated. Use this if
\lstinline{<value>} is a string.

Values can be retrieved with the command\index[tex]{UsePropVal@\texttt{\textbackslash tpUsePropVal}}
\begin{lstlisting}[style=tex]
\tpUseProperty{<name>}
\end{lstlisting}

A user can define her your own properties, but they will most likely
have no effect until they are used or evaluated against. Be aware not
to wrongly mis-use pre-defined Properties of the {\CoCoTeX}
modules.

\begin{Heading}{subsection}
  \tpTitle{Default Property Sets}\index{Properties!Default}\index{Default Property Set|textbf}
\end{Heading}

Each Namespace has its own \textit{Default Property Set} that is
evaluated before any user-defined specifications or that of child
Namespaces\index{Namespace!Child} are evaluated.

You can extend the default Properties of a \lstinline{<Namespace>} by
using\index[tex]{AddToDefault@\texttt{\textbackslash tpAddToDefault}}
\begin{lstlisting}[style=tex]
\tpAddToDefault{<Namespace>}{<setProperties>}
\end{lstlisting}
where \lstinline{<setProperties>} is a list of \lstinline{\tpSetProperty}
assignments.\Hack{\newpage}

\begin{Heading}{section}
  \tpTitle{Evaluating Properties}
\end{Heading}

\begin{Heading}{subsection}
  \tpTitle{Conditionals}\index{Properties!Conditionals}
\end{Heading}

There are two conditionals:
\begin{lstlisting}[style=tex]
\tpIfProp{<name>}{<then>}{<else>}
\tpIfPropVal{<name>}{<value>}{<then>}{<else>}
\end{lstlisting}
The first conditional \lstinline{\tpIfProp}\index[tex]{IfProp@\texttt{\textbackslash tpIfProp}} checks if a property with
the name \lstinline{<name>} is defined \textit{or} set to a value
other than an empty pair or brackets. If the macro is defined
\lstinline{and} non-empty, the instructions in the \lstinline{<then>}
branch are executed. Otherwise, the instructions in the
\lstinline{<else>} branch are executed.

The conditional \lstinline{\tpIfPropVal}\index[tex]{IfPropVal@\texttt{\textbackslash tpIfPropVal}} also checks if the Property
is defined and also if the current Value of that Property is equal to
\lstinline{<value>}. Both \lstinline{<value>} and the Value of the
Property are fully expanded before the comparision. If both expansions
are equal, \lstinline{<then>} is executed, otherwise
\lstinline{<else>} is executed.

\begin{Heading}{subsection}
  \tpTitle{Evaluation and Inheritance of Properties}\index{Properties!Evaluation}\index{Properties!Inheritance}
\end{Heading}

The evaluation of macros is done with the macro\index[tex]{CascadeProp@\texttt{\textbackslash tpCascadeProp}}:
\begin{lstlisting}[style=tex]
\tpCascadeProps{<namespace>}{<top-level-namespace>}
\end{lstlisting}
This macro recursivly loads a \lstinline{<namespace>}'s properties, the
properties of the namespace's parent, and the default properties of the
\lstinline{<top-level-namespace>} (in reverse order).

\begin{Heading}[label=sec:common:components]{section}
  \tpTitle{Using Components}\index{Component!Usage}
\end{Heading}

The macro\index[tex]{tpUseComp@\texttt{\textbackslash tpUseComp}}
\begin{lstlisting}[style=tex]
\tpUseComp{<Component>}
\end{lstlisting}
is used to expand the contents of a Component with the name
\lstinline{<Component>}.

The conditional\index[tex]{tpIfComp@\texttt{\textbackslash tpIfComp}}
\begin{lstlisting}[style=tex]
\tpIfComp{<Component>}{<True>}{<False>}
\end{lstlisting}
can be used to check the value of a Component with the name
\lstinline{<Component>}.

If the Component's content expands to \lstinline{\relax},
\lstinline{<False>} is expanded. This is automatically the case if a
Component is \textit{not used} in a Namespace environment, or if a
Component with the name \lstinline{<Component>} is undefined.

In all other cases, \lstinline{<True>} is expanded. Is is also the
case, when the arguent of the Component macro is left empty.

\begin{Heading}[label=sec:common:components]{section}
  \tpTitle{Using Hooks}\index{Hooks!Usage}
\end{Heading}

Hooks are Namespace-specific, yet globally defined, landing points for
custom adjustments.

In contrast to Properties, which are replaced when re-declared, hooks
are cumulated, i.e., if you add material into a hook, its previous
content will remain.

The second big difference is the scope: Hooks are global, that is they
can be filled at every point in the tex code. Properties, on the other
hand, can only be altered within certain macros or environments.

A Hook is declared with the \lstinline{\tpDeclareHook} macro which
takes the Hook's name as argument:
\begin{lstlisting}
\tpDeclareHook{<Hook-Name>}
\end{lstlisting}

To add code to a hook, simply use
\begin{lstlisting}[style=tex]
\tpAddToHook[<namespace>]{<Hook-Name>}{<Appendix}
\end{lstlisting}
where \lstinline{<Hook-Name>} is the name of the Hook, and
\lstinline{<Appendix>} is the code you'd like to add to the hook. If
the optional <namespace> argument is omitted, the hook will be bound
to the currently active Namespace.

The Hook is used with the macro
\begin{lstlisting}[style=tex]
\tpUseHook{<Hook-name>}
\end{lstlisting}
Make sure that \lstinline{\tp@namespace} is set correctly or else you
may expand the wrong Hook.

A Hook can be re-used.
